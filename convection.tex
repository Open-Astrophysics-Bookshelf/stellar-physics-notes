% !TEX root = ./notes.tex
\chapter{Convection}\label{s.convection}

Hot air rises, as a glider pilot or hawk can tell you. The fluid velocities in question are very subsonic, so we still have hydrostatic equilibrium to excellent approximation. You can perform the following experiment to demonstrate this phenomenon, \emph{convection}.  Brew tea, and pour the hot tea into a saucepan that is on an unlit burner.  Using a straw with your thumb over the top to place a layer of cold milk under the warm tea in the saucepan.  The temperature difference should help keep the tea and milk from mixing.   Light the burner, and watch for the development of convection---you will know it when you see it.

\section{Criteria for onset of convection}\label{s.convection-onset}

To understand this process, let's image that we have a fluid in planar geometry and hydrostatic equilibrium,
\begin{equation}
\frac{\dif P}{\dif z} = -\rho g.
\end{equation}
Now, imagine moving a blob of fluid upwards from $z$ to $z+h$.  We move the blob slowly enough that it is in hydrostatic equilibrium with its new surroundings, $P_{b}(z+h) = P(z+h)$, where the subscript $b$ refers to ``blob.'' We do, however, move the blob quickly enough that it is not in \emph{thermal} equilibrium with its surroundings; that is we move the blob adiabatically.  The entropy of the blob at $z+h$ is the same as at $z$ and is therefore not, in general, equal to the entropy of the surrounding gas at $z+h$: $S_{b}(z+h) = S_{b}(z) = S(z) \neq S(z+h)$.  Figure~\ref{f.convective-schematic} has a cartoon of this process.

\begin{figure}[htbp]
\centerline{\includegraphics[width=\textwidth]{convective}}
\caption{\label{f.convective-schematic}Illustration of criteria for convective instability.  On the left, raising a blob a distance $h$ adiabatically and in pressure balance with its surrounding results in a higher density $V_{b} < V$.  This is stable: the blob will sink back.  On the right, the blob is less dense and hence buoyant: it will continue to rise.}
\end{figure}

In raising the blob, we had to displace some of the surrounding fluid. Archimedes tells us that if the displaced fluid is less massive than the blob, than the blob will sink.  Equivalently, if the volume of an equivalent mass of background fluid is greater than that of the blob, the blob will sink,
\begin{eqnarray}
\lefteqn{V[P(z+h),S(z+h)] - V_{b}[P(z+h),S(z+h)] =}\nonumber\\
&&  V[P(z+h),S(z+h)] - V[P(z+h),S(z)] > 0
\label{e.archimedes}
\end{eqnarray}
If this condition is violated, the blob continues to rise, and the system is unstable to convection.  Expanding the left-hand side of equation~(\ref{e.archimedes}) gives
\[
V[P(z+h),S(z)] + \tderiv{V}{S}{P}\frac{\dif S}{\dif z} - V[P(z+h),S(z)] > 0 
\]
so our condition for stability is just
\begin{equation}\label{e.convective-stability}
\left(\frac{\partial V}{\partial S}\right)_{P}\frac{\dif S}{\dif z} > 0.
\end{equation}
Noting that
\begin{eqnarray*}
\tderiv{V}{T}{P} &=& \tderiv{V}{S}{P}\tderiv{S}{T}{P}\\
 &=& \frac{C_{P}}{T}\tderiv{V}{S}{P},
 \end{eqnarray*}
 we can rewrite equation~(\ref{e.convective-stability}) as
 \[
 \frac{T}{C_{P}}\tderiv{V}{T}{P}\frac{\dif S}{\dif z} > 0.
 \]
 Now, $(\partial V/\partial T)_{P}$ is positive (gas expands on being heated), so our condition for stability is simply
 \begin{equation}\label{e.entropy-condition}
\frac{\dif S}{\dif z} > 0.
\end{equation}
In a convectively stable star, the entropy must increase with radius. Convection mixes high-entropy material outward, where it will eventually mix.  As a result, convection drives the entropy gradient toward the marginally stable configuration $\dif S/\dif r = 0$.  If a star is fully convective and mixes efficiently, the interior of the star lies along an adiabat. 

We can derive a condition for convective stability in terms of the local gradients of temperature and pressure. Writing $S = S[P(r),T(r)]$ we expand equation~(\ref{e.entropy-condition}) and use hydrostatic equilibrium to obtain
\begin{equation}\label{e.schwarzschild-1}
\frac{\dif S}{\dif r} = \tderiv{S}{P}{T} \frac{\dif P}{\dif r} + \tderiv{S}{T}{P}\frac{\dif T}{\dif r} .
\end{equation}
Now, $P$ is a monotonically decreasing function of $r$, which means we can use it as a spatial coordinate and write,
\begin{equation}\label{e.TPstar}
\frac{\dif T}{\dif r} = \TPstar \frac{\dif P}{\dif r} .
\end{equation}
Here $\dif T/\dif P|_{\star}$ is the slope of the $T(P)$ relation \emph{for the stellar interior}.  In particular, this is not a thermodynamic equality. Substituting equation~(\ref{e.TPstar}) into equation~(\ref{e.schwarzschild-1}), using hydrostatic equilibrium to eliminate $\dif P/\dif r$, and recognizing that $(\partial S/\partial T)_{P} = C_{P}/T$, we obtain
\begin{equation}\label{e.schwarzschild-2}
\frac{\dif S}{\dif r} =  -\rho g\left[\tderiv{S}{P}{T} + \frac{C_{P}}{T} \TPstar \right].
\end{equation}
Finally, we can use the identity (see Appendix~\ref{s.thermo-exercises})
\begin{equation}
\tderiv{S}{P}{T}\tderiv{T}{S}{P}\tderiv{P}{T}{S} = -1
\end{equation}
to simplify equation~(\ref{e.schwarzschild-2}),
\begin{eqnarray}
\frac{\dif S}{\dif r} &=& -\frac{\rho g}{P}C_{P}\left[\frac{P}{T}\TPstar - \frac{P}{T}\tderiv{T}{P}{S}\right]\nonumber \\
 & = & -\frac{\rho g}{P}C_{P}\left[\nabla - \nabla_{\mathrm{ad}}\right].
 \label{e.schwarzschild}
\end{eqnarray}
Here we have introduced the shorthand $\nabla\equiv \dif \ln T/\dif\ln P|_{\star}$, $\nabla_{\mathrm{ad}} \equiv \left(\partial T/\partial P\right)_{S}$.
 
\section{Efficiency of Heat Transport}

In the previous section, we found that a superadiabatic temperature gradient induces convective motions. A rising blob will be hotter than its surroundings.  As it rises, heat is conducted from blob to surroundings. As a result, convection transports heat upwards and reduces the thermal gradient. The question is by how much.  Clearly the gradient must be super-adiabatic to drive the convection in the first place. We shall see, however, that usually in stars the difference between the gradient and the adiabat are exceedingly small. In other words, convection is extraordinarily efficient at transporting heat.

To understand this, let's go back to our equations.  Let's write our density as $\rho + \Delta\rho$, where $\rho$ is found from hydrostatic balance and $\Delta\rho$ is a perturbation stemming from differences in temperature between rising and falling blobs.   We can insert this into the Navier-Stokes equation; furthermore, we will both $\Delta\rho$ and $\vu$ to be perturbations, so we will drop terms like $\vu\Delta \rho$, and arrive at a perturbed Navier-Stokes equation,
\begin{equation}\label{e.perturbed-navier-stokes}
 (\partial_{t}\vu + \vu\vdot\grad\vu) = \frac{\Delta\rho}{\rho}\vg = \left(\frac{\partial\ln\rho}{\partial\ln T}\right)_{P}\frac{\Delta T}{T}\vg.
 \end{equation}
%Heat conduction follows the equation
%\begin{equation}\label{e.heat-conduction}
%\rho C_{p}(\partial_{t} + \vu\cdot\grad)T = -\divr \bvec{F} = K\nabla^{2} T
%\end{equation}
%where $K$ is the effective thermal conductivity (includes contributions from both radiative and electronic transport of heat).
Our goal is to estimate the velocity of convective motions $u$, the departure of the temperature gradient from an adiabat $\Delta T$, and the fraction of the total heat flux carried by convective motions from these equations.

First, the velocity.  The left-hand side of equation~(\ref{e.perturbed-navier-stokes}) has a characteristic scale $]sim U^{2}/L$, whereas the right-hand side has a scale $\Delta T/T g$. (Recall that in an ideal gas, $(\partial\ln \rho/\partial\ln T) = -1$.)  If we take $L \sim c_{s}^{2}/g$, a pressure scale height, than we get an estimate of the convective velocity,
\begin{equation}\label{e.convective-velocity-estimate}
\frac{U}{c_{s}} \sim \left(\frac{\Delta T}{T}\right)^{1/2}.
\end{equation}
What is the heat flux carried by convection? Hot fluid rises and carries an excess of heat, per gram, of $c_{P}\Delta T$, giving a heat flux $\approx \rho \vu c_{P}\Delta T$. Thus to carry a given flux $F$, we have
\begin{equation}\label{e.convective-flux-estimate}
c_{s}\rho c_{P} T\left(\frac{\Delta T}{T}\right)^{3/2} \sim F.
\end{equation}
Note that in order of magnitude, $c_{P}T \sim c_{s}^{2}$, so 
\[
\frac{U}{c_{s}} \sim \left(\frac{\Delta T}{T}\right)^{1/2} \sim \left(\frac{F}{\rho c_{s}^{3}}\right)^{1/3}.
\]
For conditions in the solar interior, $F \ll \rho c_{s}^{3}$, and therefore the convective velocities are very subsonic. Indeed,
\begin{eqnarray*}
 \frac{F}{\rho c_{s}^{3}} &\sim& \frac{L_{\sun}}{4\pi \Rsun^{2}}\frac{4\pi \Rsun^{3}}{3\Msun}\left(\frac{\Rsun}{G\Msun}\right)^{3/2}\\
  &\sim& \frac{L_{\sun}}{G\Msun^{2}/\Rsun}\left(\frac{\Rsun^{3}}{G\Msun}\right)^{1/2} \\
 	&\sim& \frac{t_{\mathrm{dyn}}}{t_{\mathrm{KH}}} \ll 1.
\end{eqnarray*}
That is, the ratio of the solar flux to what could be carried for near-sonic convective motions is of the order of the dynamical timescale to the Kelvin-Helmholtz timescale.

We therefore expect that in a convective region, slow circulation will produce a temperature gradient that is very nearly adiabatic.

\section{Turbulence}
 
From the discussion of the previous section, it might seem possible, given the boundary conditions, of solving for the flow planform, that is, the velocity profile $\vu(\vx,t)$. This, however, is decidedly not the case: the flow is turbulent, with intermittent velocity fluctuations seen over a large dynamical range of spatial and temporal scales.  This affects the transport and is a vexing problem in modeling fluid dynamics.

To explore this a bit further, we need to introduce the concept of dynamical similarity.  Suppose you want to optimize a wing shape for an aircraft, and you wish to test its performance in a wind tunnel.  Why should you expect that the behavior of a model of a wing will have any relation to the full-scale one?  Let's investigate our Navier-Stokes equation in one dimension,
\[
	(\partial_{t} + u\cdot\partial_{x})u = -\frac{1}{\rho}\partial_{x}P + \nu \partial_{x}^{2}u.
\]
Let's define new variables: if $L$, $U$ and $R$ represent the characteristic length, velocity, and density scales, define $\bar{x} = x / L$, $\bar{u} = u / U$, and $\bar{\rho} = R/\rho$. This implicitly defines the time variable, $\bar{t} = t\cdot U/L$. Upon changing to these variables and writing the equation of state as $P = c_{s}^{2} \rho$ (appropriate for adiabatic flow), we obtain the dimensionless equation
\begin{equation}\label{e.scaled-NS}
	(\partial_{\bar{t}} + \bar{u}\cdot\partial_{\bar{x}})\bar{u} = -\left\{\frac{c_{s}^{2}}{U^{2}}\right\}\partial_{\bar{x}}\ln\bar{\rho} + \left\{\frac{\nu}{UL}\right\} \partial_{\bar{x}}^{2}\bar{u}.
\end{equation}
The physical characteristics of the fluid and the scales involved are contained in this mathematical equation through two dimensionless parameters:
\begin{eqnarray*}
 \Ma\equiv\frac{U}{c_{s}}&\qquad& \textrm{Mach number (measure of compressibility)}\\
 \Rey\equiv \frac{UL}{\nu} &\qquad& \textrm{Reynolds number (measure of viscous forces)}
\end{eqnarray*}
So, if we built a model wing at a certain scale and placed it in a tunnel with a certain velocity scale, by adjusting the density and temperature (and hence the sound speed and viscosity) to the same $\Ma$ and $\Rey$, then the flow pattern in our model will faithfully replicate the flow in the actual system.

We saw above that for convective motion the pressure gradient and the gravitational potential are balanced, and the flow velocity is very subsonic.  What about \Rey?  We saw earlier that the $\nu \ll u\cdot \Rsun$.  Terrestrial experiments and simulations cannot approach this regime. Experimentally, when $\Rey \gtrsim 10^{3}$, then flow becomes turbulent.  

\section{Exercises}
\begin{enumerate}
\item Assuming that $\nabla \approx \nabla_{\mathrm{ad}}$ in a convective region, sketch a plot of temperature as a function of pressure for
\begin{enumerate}
\item A star with a stable inner layer and a convective outer layer;
\item A star with a convective inner layer and and a stable outer layer.
\end{enumerate}
Indicate on these plots an adiabat.
\end{enumerate}