% !TEX TS-program = xelatex
\documentclass[11pt]{article}
\usepackage[
        pdfencoding=auto,%
        pdftitle={Notes on Stellar Astrophysics: Example Problems},%
        pdfauthor={Edward F. Brown},%
        pdfstartview=FitV,%
        colorlinks=true,%
        linkcolor=blue,%
        citecolor=black, %
        urlcolor=blue]{hyperref}
\usepackage{amssymb}
\usepackage{fontspec}
\usepackage{textcomp}
\usepackage{graphicx}
\usepackage{natbib}
\usepackage{fancyvrb}
\usepackage{aasjournals}

\defaultfontfeatures{Scale=MatchLowercase}
\setmainfont[Mapping=tex-text]{Minion Pro}
\setmonofont[Mapping=tex-text]{Courier}
\setsansfont[Mapping=tex-text]{Myriad Pro}

\newcommand*{\ee}[1]{\ensuremath{\times 10^{#1}}}
\newcommand*{\sci}[2]{\ensuremath{#1\ee{#2}}}
\newcommand{\power}[2]{\ensuremath{{#1}^{#2}}}
\newcommand*{\unitskip}{\,}
\newcommand*{\unitstyle}[1]{\ensuremath{\mathrm{#1}}}
\newcommand*{\val}[2]{\ensuremath{#1\unitskip#2}}

% sometimes you want to input a range of numbers
% example: \rng{2}{3} puts 2 to 3; \rng[--]{2}{3} puts 2--3, where the two dashes are correctly typeset as an en-dash
% rather than a minus sign
\newcommand*{\rng}[3][~to~]{\ensuremath{#2\textrm{#1}#3}}
% this puts in a range with a unit
% example: \valrng[--]{2}{3}{\times \val{10^{3}}{\kilo\gram}}
\newcommand*{\valrng}[4][~to~]{\ensuremath{\left(#2\textrm{#1}#3\right)#4}}

% prefixes
\newcommand*{\nano}{\unitstyle{n}}
\newcommand*{\milli}{\unitstyle{m}}
\newcommand*{\centi}{\unitstyle{c}}
\newcommand*{\kilo}{\unitstyle{k}}
\newcommand*{\Mega}{\unitstyle{M}}
\newcommand*{\Giga}{\unitstyle{G}}

% base units, mks
\newcommand*{\meter}{\unitstyle{m}}
\newcommand*{\kilogram}{\kilo\gram}
\newcommand*{\second}{\unitstyle{s}}

\newcommand*{\Kelvin}{\unitstyle{K}}
\newcommand*{\K}{\Kelvin}  %degrees Kelvin

% base units, cgs
\newcommand*{\cm}{\centi\meter}
\newcommand*{\gram}{\unitstyle{g}}

% derived units
\newcommand*{\grampercc}{\gram\unitskip\power{\cm}{-3}} %mass density
\newcommand*{\grampersquarecm}{\gram\unitskip\power{\cm}{-2}} %column depth
\newcommand*{\GramPerCc}{\grampercc}
\newcommand*{\GramPerSc}{\grampersquarecm}
\newcommand*{\columnunit}{\grampersquarecm}
\newcommand*{\dyne}{\unitstyle{dyn}} %dyne
\newcommand*{\erg}{\unitstyle{erg}} %ergs
\newcommand*{\ergs}{\erg}
\newcommand*{\gauss}{\unitstyle{G}} %gauss
\newcommand*{\ergspersecond}{\erg\unitskip\power{\second}{-1}}
\newcommand*{\ergspergram}{\erg\unitskip\power{\gram}{-1}}
\newcommand*{\cgsflux}{\erg\unitskip\power{\cm}{-2}\unitskip\power{\second}{-1}}

% Nuclear and atomic units
\newcommand*{\amu}{\unitstyle{u}} %atomic mass unit
\newcommand*{\angstrom}{\mbox{\AA}} %Angstrom
\newcommand*{\fermi}{\unitstyle{fm}} %fermi
\newcommand*{\eV}{\unitstyle{eV}}        %eV
\newcommand*{\keV}{\kilo\eV} %Kev
\newcommand*{\MeV}{\Mega\eV} %MeV
\newcommand*{\GeV}{\Giga\eV} %GeV
% Beam units
\newcommand*{\MeVA}{\MeV/A} % MeV per nucleon
\newcommand*{\GeVA}{\GeV/A} % GeV per nucleon

% solar and astronomical units
\newcommand*{\Msun}{\ensuremath{M_\odot}}
\newcommand*{\Myr}{\Mega\yr}
\newcommand*{\Gyr}{\Giga\yr}
\newcommand*{\parsec}{\unitstyle{pc}}
\newcommand*{\kpc}{\kilo\parsec} %kiloparsec
\newcommand*{\Jansky}{\unitstyle{Jy}}	% Jansky
\newcommand*{\mJy}{\unitstyle{\mu Jy}} %micro Jansky
\newcommand*{\Msunperyr}{\Msun\unitskip\power{\yr}{-1}}	% solar masses per year

% misc. units
\newcommand*{\minute}{\unitstyle{min}} %minute
\newcommand*{\hour}{\unitstyle{hr}} %hour
\newcommand*{\yr}{\unitstyle{yr}}        %year
\newcommand*{\km}{\kilo\meter}   %kilometers
\newcommand*{\Hz}{\unitstyle{Hz}}        %Hertz
\newcommand*{\ksec}{\kilo\second} %kilosecond

% $Id: vectors.tex 385 2008-07-13 20:07:02Z efb $
%=======================================================================
%
% vectors.tex---some basic vector operators
%
% Requires package bm.sty
%
%=======================================================================

\RequirePackage{bm}
%\RequirePackage{amssymb}
%\RequirePackage{amsbsy}
\newcommand*{\bvec}[1]{\ensuremath{\bm{#1}}} %boldface vector style
\newcommand*{\grad}{\bvec{\nabla}} %gradient
\newcommand*{\divr}{\nabla \cdot} %divergence
\newcommand*{\curl}{\bvec{\nabla \times}} %curl
\newcommand*{\lap}{\ensuremath{\nabla^2}} %Laplacian
\newcommand*{\btens}[1]{\ensuremath{\bm{\mathsf{#1}}}}
\newcommand*{\vcross}{\bvec{\times}}
\newcommand*{\vdot}{\bvec{\cdot}}
% end vectors.tex

% $Id: nuclides.tex 385 2008-07-13 20:07:02Z efb $
% nuclides.tex
% input file with macros for nuclides

% base command
\newcommand*{\nuclei}[2]{\ensuremath{\mathrm{^{#1}#2}}}

% nuclides, with most highest abundance or longest half-life as default
% for example, \carbon produces ^{12}C, \carbon[13] produces ^{13}C
%
\newcommand*{\neutron}{\ensuremath{n}}
\newcommand*{\nt}{\neutron}
\newcommand*{\proton}{\ensuremath{p}}
\newcommand*{\pt}{\proton}
\newcommand*{\hydrogen}[1][1]{\nuclei{#1}{H}}
\newcommand*{\helium}[1][4]{\nuclei{#1}{He}}
\newcommand*{\lithium}[1][7]{\nuclei{#1}{Li}}
\newcommand*{\beryllium}[1][9]{\nuclei{#1}{Be}}
\newcommand*{\boron}[1][11]{\nuclei{#1}{B}}
\newcommand*{\carbon}[1][12]{\nuclei{#1}{C}}
\newcommand*{\nitrogen}[1][14]{\nuclei{#1}{N}}
\newcommand*{\oxygen}[1][16]{\nuclei{#1}{O}}
\newcommand*{\fluorine}[1][19]{\nuclei{#1}{F}}
\newcommand*{\neon}[1][20]{\nuclei{#1}{Ne}}
\newcommand*{\sodium}[1][23]{\nuclei{#1}{Na}}
\newcommand*{\magnesium}[1][24]{\nuclei{#1}{Mg}}
\newcommand*{\aluminum}[1][27]{\nuclei{#1}{Al}}
\newcommand*{\silicon}[1][28]{\nuclei{#1}{Si}}
\newcommand*{\phosphorus}[1][31]{\nuclei{#1}{P}}
\newcommand*{\sulfur}[1][32]{\nuclei{#1}{S}}
\newcommand*{\chlorine}[1][35]{\nuclei{#1}{Cl}}
\newcommand*{\argon}[1][36]{\nuclei{#1}{Ar}}
\newcommand*{\potassium}[1][39]{\nuclei{#1}{K}}
\newcommand*{\calcium}[1][40]{\nuclei{#1}{Ca}}
\newcommand*{\scandium}[1][45]{\nuclei{#1}{Sc}}
\newcommand*{\titanium}[1][48]{\nuclei{#1}{Ti}}
\newcommand*{\vanadium}[1][51]{\nuclei{#1}{V}}
\newcommand*{\chromium}[1][52]{\nuclei{#1}{Cr}}
\newcommand*{\manganese}[1][55]{\nuclei{#1}{Mn}}
\newcommand*{\iron}[1][56]{\nuclei{#1}{Fe}}
\newcommand*{\cobalt}[1][59]{\nuclei{#1}{Co}}
\newcommand*{\nickel}[1][58]{\nuclei{#1}{Ni}}
\newcommand*{\copper}[1][63]{\nuclei{#1}{Cu}}
\newcommand*{\zinc}[1][64]{\nuclei{#1}{Zn}}
\newcommand*{\gallium}[1][69]{\nuclei{#1}{Ga}}
\newcommand*{\germanium}[1][74]{\nuclei{#1}{Ge}}
\newcommand*{\arsenic}[1][75]{\nuclei{#1}{As}}
\newcommand*{\selenium}[1][80]{\nuclei{#1}{Se}}
\newcommand*{\bromine}[1][79]{\nuclei{#1}{Br}}
\newcommand*{\krypton}[1][84]{\nuclei{#1}{Kr}}
\newcommand*{\rubidium}[1][85]{\nuclei{#1}{Rb}}
\newcommand*{\strontium}[1][88]{\nuclei{#1}{Sr}}
\newcommand*{\yttrium}[1][89]{\nuclei{#1}{Y}}
\newcommand*{\zirconium}[1][94]{\nuclei{#1}{Zr}}
\newcommand*{\niobium}[1][93]{\nuclei{#1}{Nb}}
\newcommand*{\molybdenum}[1][98]{\nuclei{#1}{Mo}}
\newcommand*{\technetium}[1][97]{\nuclei{#1}{Tc}}
\newcommand*{\ruthenium}[1][102]{\nuclei{#1}{Ru}}
\newcommand*{\rhodium}[1][103]{\nuclei{#1}{Rh }}
\newcommand*{\palladium}[1][106]{\nuclei{#1}{Pd}}
\newcommand*{\silver}[1][107]{\nuclei{#1}{Ag}}
\newcommand*{\cadmium}[1][114]{\nuclei{#1}{Cd}}
\newcommand*{\indium}[1][115]{\nuclei{#1}{In}}
\newcommand*{\tin}[1][120]{\nuclei{#1}{Sn}}
\newcommand*{\antimony}[1][121]{\nuclei{#1}{Sb}}
\newcommand*{\tellurium}[1][130]{\nuclei{#1}{Te}}
\newcommand*{\iodine}[1][127]{\nuclei{#1}{I}}
\newcommand*{\xenon}[1][132]{\nuclei{#1}{Xe}}
\newcommand*{\cesium}[1][133]{\nuclei{#1}{Cs}}
\newcommand*{\barium}[1][138]{\nuclei{#1}{Ba}}
\newcommand*{\lanthanum}[1][139]{\nuclei{#1}{La}}
\newcommand*{\cerium}[1][140]{\nuclei{#1}{Ce}}
\newcommand*{\praseodymium}[1][141]{\nuclei{#1}{Pr}}
\newcommand*{\neodymium}[1][142]{\nuclei{#1}{Nd}}
\newcommand*{\promethium}[1][147]{\nuclei{#1}{Pm}}
\newcommand*{\samarium}[1][152]{\nuclei{#1}{Sm}}
\newcommand*{\europium}[1][153]{\nuclei{#1}{Eu}}
\newcommand*{\gadolinium}[1][158]{\nuclei{#1}{Gd}}
\newcommand*{\terbium}[1][159]{\nuclei{#1}{Tb}}
\newcommand*{\dysprosium}[1][164]{\nuclei{#1}{Dy}}
\newcommand*{\holmium}[1][165]{\nuclei{#1}{Ho}}
\newcommand*{\erbium}[1][168]{\nuclei{#1}{Er}}
\newcommand*{\thulium}[1][169]{\nuclei{#1}{Tm}}
\newcommand*{\ytterbium}[1][174]{\nuclei{#1}{Yb}}
\newcommand*{\lutetium}[1][175]{\nuclei{#1}{Lu}}
\newcommand*{\hafnium}[1][180]{\nuclei{#1}{Hf}}
\newcommand*{\tantalum}[1][180]{\nuclei{#1}{Ta}}
\newcommand*{\tungsten}[1][184]{\nuclei{#1}{W}}
\newcommand*{\rhenium}[1][187]{\nuclei{#1}{Re}}
\newcommand*{\osmium}[1][192]{\nuclei{#1}{Os}}
\newcommand*{\iridium}[1][193]{\nuclei{#1}{Ir}}
\newcommand*{\platnium}[1][195]{\nuclei{#1}{Pt}}
\newcommand*{\gold}[1][197]{\nuclei{#1}{Au}}
\newcommand*{\mercury}[1][202]{\nuclei{#1}{Hg}}
\newcommand*{\thallium}[1][205]{\nuclei{#1}{Tl}}
\newcommand*{\lead}[1][208]{\nuclei{#1}{Pb}}
\newcommand*{\bisumth}[1][209]{\nuclei{#1}{Bi}}
\newcommand*{\polonium}[1][210]{\nuclei{#1}{Po}}
\newcommand*{\astatine}[1][210]{\nuclei{#1}{At}}
\newcommand*{\radon}[1][222]{\nuclei{#1}{Rn}}
\newcommand*{\francium}[1][223]{\nuclei{#1}{Fr}}
\newcommand*{\radium}[1][226]{\nuclei{#1}{Ra}}
\newcommand*{\actinium}[1][227]{\nuclei{#1}{Ac}}
\newcommand*{\thorium}[1][232]{\nuclei{#1}{Th}}
\newcommand*{\protactinium}[1][231]{\nuclei{#1}{Pa}}
\newcommand*{\uranium}[1][238]{\nuclei{#1}{U}}
\newcommand*{\neptunium}[1][237]{\nuclei{#1}{Np}}
\newcommand*{\plutonium}[1][244]{\nuclei{#1}{Pu}}
\newcommand*{\americium}[1][243]{\nuclei{#1}{Am}}
\newcommand*{\curium}[1][247]{\nuclei{#1}{Cm}}
\newcommand*{\berkelium}[1][247]{\nuclei{#1}{Bk}}
\newcommand*{\californium}[1][251]{\nuclei{#1}{Cf}}
\newcommand*{\einsteinium}[1][252]{\nuclei{#1}{Es}}
\newcommand*{\fermium}[1][257]{\nuclei{#1}{Fm}}
\newcommand*{\mendelevium}[1][258]{\nuclei{#1}{Md}}
\newcommand*{\nobelium}[1][259]{\nuclei{#1}{No}}
\newcommand*{\lawrencium}[1][262]{\nuclei{#1}{Lr}}
\newcommand*{\rutherfordium}[1][261]{\nuclei{#1}{Rf}}
\newcommand*{\dubnium}[1][268]{\nuclei{#1}{Db}}
\newcommand*{\seaborgium}[1][271]{\nuclei{#1}{Sg}}
\newcommand*{\bohrium}[1][274]{\nuclei{#1}{Bh}}
\newcommand*{\hassium}[1][270]{\nuclei{#1}{Hs}}
\newcommand*{\meitnerium}[1][278]{\nuclei{#1}{Mt}}
\newcommand*{\darmstadtium}[1][281]{\nuclei{#1}{Ds}}
\newcommand*{\roentgenium}[1][281]{\nuclei{#1}{Rg}}
\newcommand*{\copernicum}[1][285]{\nuclei{#1}{Cn}}

% symbols used in this work
\newcommand*{\ssb}{\ensuremath{\sigma_{\mathrm{SB}}}} %Stefan-Boltzmann constant
\newcommand*{\rn}{\ensuremath{r_{\mathrm{N}}}} %nuclear radius
\newcommand*{\re}{\ensuremath{r_{\mathrm{E}}}} %classical turning point
\newcommand*{\EG}{\ensuremath{E_{\mathrm{G}}}} %Gamow energy
\newcommand*{\Epk}{\ensuremath{E_{\mathrm{pk}}}} %Peak energy for reaction
\newcommand*{\TPstar}{\left.\frac{\dif T}{\dif P}\right|_{\star}} % derivative of T wrt P in star
\newcommand*{\sun}{\ensuremath{\odot}} %sun
\newcommand*{\eF}{\ensuremath{\varepsilon_{\mathrm{F}}}} %Fermi energy

% vector macros
\newcommand*{\unitvector}[1]{\ensuremath{\bvec{\hat{#1}}}}
\newcommand*{\unitn}{\unitvector{n}} %unit 'n' vector
\newcommand*{\unitk}{\unitvector{k}} %unit 'k' vector
\newcommand*{\unitj}{\unitvector{\jmath}} %unit 'j' vector
\newcommand*{\vu}{\ensuremath{\bvec{u}}} %vector 'u'
\newcommand*{\vv}{\ensuremath{\bvec{v}}} %vector 'v'
\newcommand*{\vx}{\ensuremath{\bvec{x}}} %vector 'x'
\newcommand*{\vr}{\ensuremath{\bvec{r}}} %vector 'r'
\newcommand*{\vg}{\ensuremath{\bvec{g}}} %vector 'g'
\newcommand*{\vp}{\ensuremath{\bvec{p}}} %vector 'p'
\newcommand*{\vk}{\ensuremath{\bvec{k}}} %vector 'k'
\newcommand*{\vkt}{\ensuremath{\bvec{k}_{t}}} %vector 'k_{t}'
\newcommand*{\vxi}{\ensuremath{\bvec{\xi}}} %vector 'k'
\newcommand*{\usp}{\unitskip}
\newcommand*{\nsp}{\usp}


\bibpunct{(}{)}{;}{a}{}{,}

\newcommand{\gsun}{\ensuremath{g_{\odot}}}
\newcommand{\tsix}{\ensuremath{t_{6}}}
\newcommand{\Teffsun}{\ensuremath{T_{\mathrm{eff,\odot}}}}
\newcommand{\mwd}{\ensuremath{m_{\mathrm{WD}}}}

\begin{document}
\newcommand{\thetitle}{AGB mass loss and the white dwarf mass distribution}
\newcommand{\thecourse}{AST 840}
\newcommand{\theauthor}{Edward F. Brown}

\begin{titlepage}
\begin{center}
\LARGE{\textbf{\thetitle}}\\*[1.5em]
\Large{\thecourse}\\*[2.5em]
\large{\theauthor}
\end{center}
\end{titlepage}

We start with the following equations,
\begin{eqnarray}\label{e.paczynski}
\frac{L}{\Lsun} &=& 5.9\ee{4}\left(\frac{M_{c}}{\Msun}-0.52\right)\\
\frac{\dif M_{c}}{\dif t} &=& \frac{L}{q}.\label{e.dmc}
\end{eqnarray}
For \hydrogen\ fusing to \helium\ the heat released per nucleon is $6.68\nsp\MeV/\amu = 6.44\ee{18}\nsp\ergs\usp\gram^{-1}$. It's convenient to scale these quantities to more useful units, so let's divide equation~(\ref{e.dmc}) by \Msun, and multiply the right-hand side by $\Lsun/\Lsun$; further, define $m_{c}\equiv M_{c}/\Msun$ and scale $t\to (10^{6}\nsp\yr)\cdot\tsix$.  Then we can substitute $L$ from equation~(\ref{e.paczynski}) into equation~(\ref{e.dmc}) to obtain
\begin{eqnarray}
\frac{\dif m_{c}}{\dif\tsix} &=& \frac{5.9\ee{4}\Lsun\cdot (10^{6}\nsp\yr)}{6.44\ee{18}\nsp\ergs\usp\gram^{-1}\cdot\Msun} \left(m_{c}-0.52\right) \nonumber\\
 &=& 0.57\left(m_{c}-0.52\right).\label{e.de-mass}
\end{eqnarray}
This equation is readily solved with an integrating factor, or we can just change variables to $\Delta\equiv m_{c}-0.52$ to obtain
\begin{equation}
m_{c}(\tsix) = 0.52 + \left(m_{c,0}-0.52\right)\exp(0.57\tsix).
\end{equation}
Here $m_{c,0} > 0.52$ is the initial core mass at the start of the AGB phase. The characteristic timescale in the problem is $(10^{6}\nsp\yr )\cdot(0.57)^{-1} = 1.8\nsp\Mega\yr$.

While this is going on, the high luminosity and low gravity in the envelope conspire to produce a strong wind; the star loses mass at a rate
\begin{equation}\label{e.mass-loss}
\frac{\dif M}{\dif t} = -8.0\ee{-13}\nsp\Msun\usp\yr^{-1}\left(\frac{L}{\Lsun}\frac{\gsun}{g}\frac{\Rsun}{R}\right).
\end{equation}
To make this tractable, we will assume a fixed $\Teff = 4000\nsp\K$.  Then we can write
\begin{equation}\label{e.r-l}
\left(\frac{L}{\Lsun}\frac{\gsun}{g}\frac{\Rsun}{R}\right) = \left(\frac{L}{\Lsun}\right)^{3/2}\left(\frac{\Msun}{M}\right)\left(\frac{\Teffsun}{\Teff}\right)^{2}.
\end{equation}
Defining $m \equiv M/\Msun$, changing variables $t\to\tsix$,  taking $\Teffsun/\Teff = (5780\nsp\K)/(4000\nsp\K)$, and using equation~(\ref{e.paczynski}) to eliminate $L/\Lsun$ then gives
\begin{equation}\label{e.m-deq}
m\frac{\dif m}{\dif \tsix} = -23.9(m_{c}-0.52)^{3/2}
\end{equation}
At this point, we can just integrate this equation. Since we are after a relation between $m$ and $m_{c}$, however, I am going to change the dependent variable from $\tsix$ to $\Delta = m_{c}-0.52$.  From equation~(\ref{e.de-mass}) we have $\dif\Delta/\dif\tsix = 0.57\Delta$; using this relation and writing $\dif m/\dif\tsix = (\dif m/\dif\Delta)(\dif \Delta/\dif\tsix)$, we can rewrite equation~(\ref{e.m-deq}) as
\begin{equation}\label{e.m-deq2}
m\frac{\dif m}{\dif\Delta} = -\frac{23.9}{0.57}\Delta^{1/2}. 
\end{equation}
This is readily integrated to give
\begin{eqnarray}
m^{2} &=& m_{0}^{2} - 56.0\left(\Delta^{3/2} - \Delta_{0}^{3/2}\right)\nonumber\\
 &=& m_{0}^{2} - 56.0\left[\left(m_{c}-0.52\right)^{3/2} - \left(m_{c,0}-0.52\right)^{3/2}\right].
\label{e.m-mc}
\end{eqnarray}
This equation gives the stellar mass as a function of time for an assumed initial core mass and an initial stellar mass.

At the end of the AGB phase, $m = m_{c}$, and we have an algebraic relation between the white dwarf mass $\mwd$, the core mass at the end of \helium\ burning $m_{c,0}$, and the total stellar mass at the end of the horizontal branch phase $m_{0}$,
\begin{equation}\label{e.mwd}
\mwd^{2} - m_{0}^{2} + 56.0\left[\left(\mwd-0.52\right)^{3/2} - \left(m_{c,0}-0.52\right)^{3/2}\right] = 0.
\end{equation}
It is easiest to find the root \mwd\ numerically. The simplest method is \emph{bisection}.  Pick two guesses for $\mwd$ that bracket the root; that is, one guess makes the LHS of equation~(\ref{e.mwd}) negative and the other makes the LHS positive.  Compute the midpoint of these two guesses.  Now find in which interval the root lies.  This defines a new bracket that is half the size of the initial bracket. By repeating this process, the bracket for the root can be made successively smaller until a desired tolerance is reached.  Figure~\ref{f.core-mass} shows the results for the final core mass: the left-hand panel holds $m_{0} = 2.0$ fixed and varies $m_{c,0}$, while the right-hand panel holds $m_{c,0} = 0.55$ fixed and varies $m_{0}$.

\begin{figure}[htbp]
\includegraphics[width=5in]{plots_out/core_mass_combined}
\caption{Final core mass as a function of the initial core mass $M_{c,0}$ (\emph{left}) for  a fixed initial stellar mass $M_{0} = 2.0\nsp\Msun$, and as a function of the initial stellar mass $M_{0}$ (\emph{right}) for a fixed $M_{c,0}=0.55\nsp\Msun$.\label{f.core-mass}}
\end{figure}
\clearpage

Here is the code used to implement the solution.  The module \verb|agb_end| holds the parameters of the equation and the function for equation~(\ref{e.mwd}); the program \verb|wd_mass| uses this module and \verb|bisect.f| to find the roots for a trial $m_{0}$, $m_{c,0}$, and coefficient (which is 56.0 in eq.~[\ref{e.mwd}]).
\VerbatimInput[numbers=left,numberblanklines=false]{agb.f}

The code to implement the bisection algorithm is
\VerbatimInput[numbers=left,numberblanklines=false]{rootfind.f}

The driver program to make the figure is then
\VerbatimInput[numbers=left,numberblanklines=false]{wdmass.f}

These codes were complied with \verb|gfortran| using the following options:
\begin{quote}
\verb|-ffree-form -fimplicit-none -fdefault-real-8|
\end{quote}

\end{document}