% !TEX root = ./notes.tex
\chapter{Nuclear Burning and Stellar Evolution}

\section[The pp chain]{Hydrogen burning via pp reactions: the lower main sequence}
\label{s.lower-ms}

In a contracting pre-main sequence star, the reaction $\hydrogen[2](p,\gamma)\helium[3]$ proceeds rapidly owing to the small Coulomb barrier; in fact, this reaction can occur in objects as small as $\approx 12\,M_{\mathrm{Jupiter}}$.  The small primordial abundance of deuterium, however, prevents this reaction from doing anything more than slowing contraction slightly.  The reaction $\pt +\pt$ is much slower, because there is no bound nucleus \helium[2]; the only possible way to form a nucleus is to have a weak interaction as well, giving the reaction $\pt(\pt,e^{+}\nu_{e})\hydrogen[2]$.

The weak cross section goes roughly as $\sigma_{\mathrm{weak}} \sim 10^{-20}\nsp\barn\left(E/\keV\right)$, so that
\[ \frac{\sigma_{\mathrm{weak}}}{\sigma_{\mathrm{nuc}}} \sim 10^{-23}\left(\frac{E}{\keV}\right). \]
The $S$-factor for the $\pt+\pt$ reaction is very small, and as a result the characteristic temperature for this reaction to occur is $\approx 1.5\ee{7}\nsp\K$; at this temperature, the lifetime of a proton to forming deuterium via capture of another proton is about $6\nsp\Giga\yr$.  Once a deuterium nucleus is formed, it is immediately destroyed via $\hydrogen[2](p,\gamma)\helium[3]$. The nucleus \lithium[4] is unbound with a lifetime of $10^{-22}\nsp\second$; the nucleus \beryllium[6] is likewise unbound ($\tau \sim 5\ee{-21}\nsp\second$). As a result, the next reaction that can occur is $\helium[3](\helium[3],2\pt)\helium$.  Despite having a much greater Gamow energy than $\pt + \pt$ (see Table~\ref{t.reaction}), this reaction still is much faster than $\pt+\pt$ owing to the small weak cross-section.

In addition to capturing another \helium[3], it is also possible that
\begin{eqnarray}
\helium[3] + \helium &\to& \beryllium[7] + \gamma\nonumber\\
 \beryllium[7] + e^{-} &\to& \lithium[7] +  \nu_{e}\qquad(\tau=53\nsp\unitstyle{d})\nonumber \\
 \lithium[7] + \pt &\to& 2\helium + \gamma;
 \end{eqnarray}
furthermore, at slightly higher temperatures \beryllium[7] can capture a proton instead of an electron, giving the third branch
\begin{eqnarray}
\beryllium[7] + \pt &\to& \boron[8] + \gamma\nonumber\\
\boron[8] &\to& \beryllium[8] + e^{+} + \nu_{e}\qquad(\tau = 770\nsp\milli\second)\nonumber\\
\beryllium[8] &\to& 2\helium\qquad(\tau=10^{-16}\nsp\second).
\end{eqnarray}
 The end result of these chains is the conversion of hydrogen to helium, although the amount of energy carried away by neutrinos differs from one chain to the next.

\section[The CNO cycle]{Hydrogen burning via the CNO cycle: the upper main sequence}
\label{s.upper-ms}

As we saw in the previous section, the smallness of the $\pt+\pt$ cross-section means that captures onto heavier nuclei can be competitive at stellar temperatures.  Let's get a rough estimate of how charged a nucleus can be before the Coulomb barrier makes the reaction slower than $\pt+\pt$.  Assuming $A = 2Z$, and taking the $S$-factor for $\pt+\pt$ to be $10^{-22}$ times smaller gives us the rough equation
\[ 10^{-22}\exp\left(-\frac{33.81}{T_{6}^{1/3}}\right) \approx \exp\left(-\frac{41.47 Z^{2/3}}{T_{6}^{1/3}}\right), \]
where the factors in the exponentials come from the peak energy for the reaction (see eq.~[\ref{e.exponent}]), and $T_{6}\equiv (T/10^{6}\nsp\K)$.  Solving for $Z$, we see that at $T_{6} = 10$, proton captures onto \carbon\ have a comparable cross-section to $\pt + \pt$; at $T_{6} = 20$, proton captures onto \oxygen\ have a comparable cross-section.

Thus at temperatures slightly greater than that in the solar center, the following catalytic cycle becomes possible.
\begin{center}
\begin{tabular}{rr}
reaction & $\log[(\tau/\yr) \times (\rho X_{H}/100\nsp\grampercc)]$\\
\hline
$\carbon(\mathbf{\pt},\gamma)\nitrogen[13]$ & 3.82\\
$\nitrogen[13](,e^{+}\nu_{e})\carbon[13]$ & $\tau=870\nsp\second$\\
$\carbon[13](\mathbf{\pt},\gamma)\nitrogen[14]$ & 3.21\\
$\nitrogen[14](\mathbf{\pt},\gamma)\oxygen[15]$ & 5.89 \\
$\oxygen[15](,e^{+}\nu_{e})\nitrogen[15]$ & $\tau = 178\nsp\second$\\
$\nitrogen[15](\mathbf{\pt},\mathbf{^{4}He})\carbon$ & 1.50 \\
\hline
\end{tabular}
\end{center}
As indicated by the boldfaced symbols, this cycle takes in 4 protons and releases 1 helium nucleus.
The reaction timescales are evaluated at a temperature of $20\nsp\Mega\K$.
The reaction $\nitrogen(\pt,\gamma)\oxygen[15]$ is by far the slowest step in the cycle; as a result, all of the CNO elements are quickly converted into \nitrogen\ in the stellar core, and this reaction controls the rate of heating.  At $T = 2\ee{7}\nsp\K$, $\dif \ln \varepsilon_{\mathrm{CNO}}/\dif\ln T = 18$; in contrast the $\pt+\pt$ reaction has a temperature exponent of only 4.5.
