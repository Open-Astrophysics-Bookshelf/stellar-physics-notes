% !TEX root = ../stellar-notes.tex

\section*{Preface}
These notes were written while teaching a graduate-level astronomy course on stars at Michigan State University.  The only background preparation for this course is undergraduate physics and a course on radiative processes, and so portions of these notes could be useful for upper-level undergraduates.  
The text layout uses the \code{tufte-book} (\url{https://tufte-latex.github.io/tufte-latex/}) \LaTeX\ class:  the main feature is a large right margin in which the students can take notes; this margin also holds small figures and sidenotes. Exercises are embedded throughout the text.  These range from ``reading exercises'' to longer, more challenging problems.  In addition, there are several numerical exercises that use the \mesa\ stellar evolution code, available from \url{http://mesa.sourceforge.net/}.  These numerical exercises are prefaced with the logo \raisebox{-0.015ex}{\includegraphics[height=1.4ex]{mesa_logo2}}, used by kind permission of the MESA council.

The course notes were originally meant as a supplement to the main text, \citetalt{Hansen2004Stellar-Interio}; in some editions of the course I also drew from \citetalt{Clayton1983Principles-of-S} and \citetalt{Kippenhahn1994Stellar-Structu}.  These notes therefore tend to expand upon topics not already covered there.  In the second half of the course, the students typically gave presentations on current topics in stellar evolution, and I supplemented those with readings from the MESA instrument papers\cite{Paxton2010Modules-for-Exp,Paxton2013Modules-for-Exp}.  As a result, however, my notes on topics of stellar evolution have lagged behind the rest of the text and are not yet ready for posting. 

Some of the material was inspired by three courses at UC-Berkeley in the mid-90's: ``Stars with Lars'', taught by Professor L. Bildsten; Statistical Physics, taught by Professor E.~Commins, and Fluid Mechanics, taught by Professor J. Graham.  I am also indebted to the students who took the MSU stellar physics course for their questions, feedback, and encouragement. Additional thanks go to MSU graduate students Dana Koeppe and Wei Jia Ong for reading late stages of the drafts and testing the numerical exercises.

\newthought{Please be advised that these notes are under active development;} to refer to a specific version, use the eight-character stamp labeled ``git version'' on the copyright page.
