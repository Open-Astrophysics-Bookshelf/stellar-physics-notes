% !TEX TS-program = xelatex
\documentclass[11pt]{book}
\usepackage[pdftitle={Notes on Stellar Astrophysics},%
        pdfauthor={Edward F. Brown},pdfstartview=FitV,colorlinks=true,linkcolor=blue,%
         citecolor=black, urlcolor=blue]{hyperref}
\usepackage{fontspec}
\usepackage{amssymb}
\usepackage{graphicx}
\usepackage{natbib}
\usepackage{aasjournals}

\defaultfontfeatures{Scale=MatchLowercase}
\setmainfont[Mapping=tex-text]{Adobe Garamond Pro}
\setmonofont[Mapping=tex-text]{Courier}
\setsansfont[Mapping=tex-text]{Myriad Pro}

\include{MyNotation}
% $Id: vectors.tex 385 2008-07-13 20:07:02Z efb $
%=======================================================================
%
% vectors.tex---some basic vector operators
%
% Requires package bm.sty
%
%=======================================================================

\RequirePackage{bm}
%\RequirePackage{amssymb}
%\RequirePackage{amsbsy}
\newcommand*{\bvec}[1]{\ensuremath{\bm{#1}}} %boldface vector style
\newcommand*{\grad}{\bvec{\nabla}} %gradient
\newcommand*{\divr}{\nabla \cdot} %divergence
\newcommand*{\curl}{\bvec{\nabla \times}} %curl
\newcommand*{\lap}{\ensuremath{\nabla^2}} %Laplacian
\newcommand*{\btens}[1]{\ensuremath{\bm{\mathsf{#1}}}}
\newcommand*{\vcross}{\bvec{\times}}
\newcommand*{\vdot}{\bvec{\cdot}}
% end vectors.tex

% $Id: nuclides.tex 385 2008-07-13 20:07:02Z efb $
% nuclides.tex
% input file with macros for nuclides

% base command
\newcommand*{\nuclei}[2]{\ensuremath{\mathrm{^{#1}#2}}}

% nuclides, with most highest abundance or longest half-life as default
% for example, \carbon produces ^{12}C, \carbon[13] produces ^{13}C
%
\newcommand*{\neutron}{\ensuremath{n}}
\newcommand*{\nt}{\neutron}
\newcommand*{\proton}{\ensuremath{p}}
\newcommand*{\pt}{\proton}
\newcommand*{\hydrogen}[1][1]{\nuclei{#1}{H}}
\newcommand*{\helium}[1][4]{\nuclei{#1}{He}}
\newcommand*{\lithium}[1][7]{\nuclei{#1}{Li}}
\newcommand*{\beryllium}[1][9]{\nuclei{#1}{Be}}
\newcommand*{\boron}[1][11]{\nuclei{#1}{B}}
\newcommand*{\carbon}[1][12]{\nuclei{#1}{C}}
\newcommand*{\nitrogen}[1][14]{\nuclei{#1}{N}}
\newcommand*{\oxygen}[1][16]{\nuclei{#1}{O}}
\newcommand*{\fluorine}[1][19]{\nuclei{#1}{F}}
\newcommand*{\neon}[1][20]{\nuclei{#1}{Ne}}
\newcommand*{\sodium}[1][23]{\nuclei{#1}{Na}}
\newcommand*{\magnesium}[1][24]{\nuclei{#1}{Mg}}
\newcommand*{\aluminum}[1][27]{\nuclei{#1}{Al}}
\newcommand*{\silicon}[1][28]{\nuclei{#1}{Si}}
\newcommand*{\phosphorus}[1][31]{\nuclei{#1}{P}}
\newcommand*{\sulfur}[1][32]{\nuclei{#1}{S}}
\newcommand*{\chlorine}[1][35]{\nuclei{#1}{Cl}}
\newcommand*{\argon}[1][36]{\nuclei{#1}{Ar}}
\newcommand*{\potassium}[1][39]{\nuclei{#1}{K}}
\newcommand*{\calcium}[1][40]{\nuclei{#1}{Ca}}
\newcommand*{\scandium}[1][45]{\nuclei{#1}{Sc}}
\newcommand*{\titanium}[1][48]{\nuclei{#1}{Ti}}
\newcommand*{\vanadium}[1][51]{\nuclei{#1}{V}}
\newcommand*{\chromium}[1][52]{\nuclei{#1}{Cr}}
\newcommand*{\manganese}[1][55]{\nuclei{#1}{Mn}}
\newcommand*{\iron}[1][56]{\nuclei{#1}{Fe}}
\newcommand*{\cobalt}[1][59]{\nuclei{#1}{Co}}
\newcommand*{\nickel}[1][58]{\nuclei{#1}{Ni}}
\newcommand*{\copper}[1][63]{\nuclei{#1}{Cu}}
\newcommand*{\zinc}[1][64]{\nuclei{#1}{Zn}}
\newcommand*{\gallium}[1][69]{\nuclei{#1}{Ga}}
\newcommand*{\germanium}[1][74]{\nuclei{#1}{Ge}}
\newcommand*{\arsenic}[1][75]{\nuclei{#1}{As}}
\newcommand*{\selenium}[1][80]{\nuclei{#1}{Se}}
\newcommand*{\bromine}[1][79]{\nuclei{#1}{Br}}
\newcommand*{\krypton}[1][84]{\nuclei{#1}{Kr}}
\newcommand*{\rubidium}[1][85]{\nuclei{#1}{Rb}}
\newcommand*{\strontium}[1][88]{\nuclei{#1}{Sr}}
\newcommand*{\yttrium}[1][89]{\nuclei{#1}{Y}}
\newcommand*{\zirconium}[1][94]{\nuclei{#1}{Zr}}
\newcommand*{\niobium}[1][93]{\nuclei{#1}{Nb}}
\newcommand*{\molybdenum}[1][98]{\nuclei{#1}{Mo}}
\newcommand*{\technetium}[1][97]{\nuclei{#1}{Tc}}
\newcommand*{\ruthenium}[1][102]{\nuclei{#1}{Ru}}
\newcommand*{\rhodium}[1][103]{\nuclei{#1}{Rh }}
\newcommand*{\palladium}[1][106]{\nuclei{#1}{Pd}}
\newcommand*{\silver}[1][107]{\nuclei{#1}{Ag}}
\newcommand*{\cadmium}[1][114]{\nuclei{#1}{Cd}}
\newcommand*{\indium}[1][115]{\nuclei{#1}{In}}
\newcommand*{\tin}[1][120]{\nuclei{#1}{Sn}}
\newcommand*{\antimony}[1][121]{\nuclei{#1}{Sb}}
\newcommand*{\tellurium}[1][130]{\nuclei{#1}{Te}}
\newcommand*{\iodine}[1][127]{\nuclei{#1}{I}}
\newcommand*{\xenon}[1][132]{\nuclei{#1}{Xe}}
\newcommand*{\cesium}[1][133]{\nuclei{#1}{Cs}}
\newcommand*{\barium}[1][138]{\nuclei{#1}{Ba}}
\newcommand*{\lanthanum}[1][139]{\nuclei{#1}{La}}
\newcommand*{\cerium}[1][140]{\nuclei{#1}{Ce}}
\newcommand*{\praseodymium}[1][141]{\nuclei{#1}{Pr}}
\newcommand*{\neodymium}[1][142]{\nuclei{#1}{Nd}}
\newcommand*{\promethium}[1][147]{\nuclei{#1}{Pm}}
\newcommand*{\samarium}[1][152]{\nuclei{#1}{Sm}}
\newcommand*{\europium}[1][153]{\nuclei{#1}{Eu}}
\newcommand*{\gadolinium}[1][158]{\nuclei{#1}{Gd}}
\newcommand*{\terbium}[1][159]{\nuclei{#1}{Tb}}
\newcommand*{\dysprosium}[1][164]{\nuclei{#1}{Dy}}
\newcommand*{\holmium}[1][165]{\nuclei{#1}{Ho}}
\newcommand*{\erbium}[1][168]{\nuclei{#1}{Er}}
\newcommand*{\thulium}[1][169]{\nuclei{#1}{Tm}}
\newcommand*{\ytterbium}[1][174]{\nuclei{#1}{Yb}}
\newcommand*{\lutetium}[1][175]{\nuclei{#1}{Lu}}
\newcommand*{\hafnium}[1][180]{\nuclei{#1}{Hf}}
\newcommand*{\tantalum}[1][180]{\nuclei{#1}{Ta}}
\newcommand*{\tungsten}[1][184]{\nuclei{#1}{W}}
\newcommand*{\rhenium}[1][187]{\nuclei{#1}{Re}}
\newcommand*{\osmium}[1][192]{\nuclei{#1}{Os}}
\newcommand*{\iridium}[1][193]{\nuclei{#1}{Ir}}
\newcommand*{\platnium}[1][195]{\nuclei{#1}{Pt}}
\newcommand*{\gold}[1][197]{\nuclei{#1}{Au}}
\newcommand*{\mercury}[1][202]{\nuclei{#1}{Hg}}
\newcommand*{\thallium}[1][205]{\nuclei{#1}{Tl}}
\newcommand*{\lead}[1][208]{\nuclei{#1}{Pb}}
\newcommand*{\bisumth}[1][209]{\nuclei{#1}{Bi}}
\newcommand*{\polonium}[1][210]{\nuclei{#1}{Po}}
\newcommand*{\astatine}[1][210]{\nuclei{#1}{At}}
\newcommand*{\radon}[1][222]{\nuclei{#1}{Rn}}
\newcommand*{\francium}[1][223]{\nuclei{#1}{Fr}}
\newcommand*{\radium}[1][226]{\nuclei{#1}{Ra}}
\newcommand*{\actinium}[1][227]{\nuclei{#1}{Ac}}
\newcommand*{\thorium}[1][232]{\nuclei{#1}{Th}}
\newcommand*{\protactinium}[1][231]{\nuclei{#1}{Pa}}
\newcommand*{\uranium}[1][238]{\nuclei{#1}{U}}
\newcommand*{\neptunium}[1][237]{\nuclei{#1}{Np}}
\newcommand*{\plutonium}[1][244]{\nuclei{#1}{Pu}}
\newcommand*{\americium}[1][243]{\nuclei{#1}{Am}}
\newcommand*{\curium}[1][247]{\nuclei{#1}{Cm}}
\newcommand*{\berkelium}[1][247]{\nuclei{#1}{Bk}}
\newcommand*{\californium}[1][251]{\nuclei{#1}{Cf}}
\newcommand*{\einsteinium}[1][252]{\nuclei{#1}{Es}}
\newcommand*{\fermium}[1][257]{\nuclei{#1}{Fm}}
\newcommand*{\mendelevium}[1][258]{\nuclei{#1}{Md}}
\newcommand*{\nobelium}[1][259]{\nuclei{#1}{No}}
\newcommand*{\lawrencium}[1][262]{\nuclei{#1}{Lr}}
\newcommand*{\rutherfordium}[1][261]{\nuclei{#1}{Rf}}
\newcommand*{\dubnium}[1][268]{\nuclei{#1}{Db}}
\newcommand*{\seaborgium}[1][271]{\nuclei{#1}{Sg}}
\newcommand*{\bohrium}[1][274]{\nuclei{#1}{Bh}}
\newcommand*{\hassium}[1][270]{\nuclei{#1}{Hs}}
\newcommand*{\meitnerium}[1][278]{\nuclei{#1}{Mt}}
\newcommand*{\darmstadtium}[1][281]{\nuclei{#1}{Ds}}
\newcommand*{\roentgenium}[1][281]{\nuclei{#1}{Rg}}
\newcommand*{\copernicum}[1][285]{\nuclei{#1}{Cn}}


\newcommand{\tderiv}[3]{\ensuremath{\left(\frac{\partial #1}{\partial #2}\right)_{#3}}}
\bibpunct{(}{)}{;}{a}{}{,}

\begin{document}

\newcommand{\thetitle}{Notes on Stellar Astrophysics}
\newcommand{\theauthor}{Edward F. Brown}

\begin{titlepage}
\vspace*{2.5in}
\begin{center}
\LARGE{\thetitle}\\*[2.0em]
\large{\theauthor}
\end{center}
\vspace*{3.0in}
\small{\verb$Date$}
\end{titlepage}

\tableofcontents

% !TEX root = ./notes.tex
\chapter{The sun on a blackboard}\label{ch.introduction}

To begin our study of stellar structure, let us first consider the star that we know best, our sun.  From planetary orbits and the determination of the gravitational constant $G$, we have the mass; our knowledge of the earth-sun distance and observations gives us the radius; measurements of the solar radiant flux and spectra give us the luminosity and temperature; and radiometric dating of meteorites gives the age of the solar system. In summary:
\begin{eqnarray*}
M_{\sun} &=& 1.99\ee{33}\nsp\gram\\
R_{\sun} &=& 6.96\ee{10}\nsp\cm\\
L_{\sun} &=& 3.86\ee{33}\nsp\ergspersecond\\
T_{\mathrm{eff}} &=& 5780\nsp\K\\
\tau_{\sun} &=& 4.6\nsp\Giga\yr.
\end{eqnarray*}
Moreover, the composition of the sun is well known \citep{anders.grevesse:abundances,Asplund2005The-Solar-Chemi}; the most abundant elements, with abundances $A > 8.0$ (here the abundance scale is relative to hydrogen, with $A \equiv \log_{10}[N_{\mathrm{el}}/N_{\mathrm{H}}] + 12$), are H(12.00), He(10.93), N(7.78), O(8.66), and C(8.39).

Another salient feature of our sun is its stability: the power output is remarkably constant, varying by less than 0.1\% over several solar cycle \citep{Willson1991The-suns-lumino,Frohlich2004Solar-radiative}, with inferred changes over 2,000\nsp\yr\ on a similar scale \citep{Frohlich2004Solar-radiative}.  On longer timescales, evidence for liquid water over much of Earth history suggest that the power output of the sun cannot have varied greatly over its life.  The first task, then, is to investigate the mechanical and thermal stability of a self-gravitating fluid.

\section{Fluid equation of motion}\label{s.fluid-introduction}

Over scales that are large compared to the collisional mean free paths between particles, we can treat the fluid as a continuous medium.  That is, we suppose that we can find a scale that is infinitesimal compared to the macroscopic scales, but still much larger than the scales for microscopic interactions. Thus, we can define thermodynamic quantities at a location.

Consider such a macroscopically small volume $V$. Its mass is $M = \int_{V} \rho\nsp\dif V$, where $\rho$ is the mass density.  If $\bvec{u}(\bvec{x},t)$ is the velocity, then the flux of mass into the element is
\[
-\int_{\partial V}\rho\vu\vdot \dif \bvec{S} = \frac{\partial}{\partial t}\int_{V}\rho\nsp\dif V
\]
where the right-hand side follows from mass conservation.  Using Gauss's law to transform the left-hand side into an integral over $V$ and combining terms, we have
\[
\int_{V} \left\{ \frac{\partial\rho}{\partial t} + \divr(\rho\vu)\right\}\dif V = 0.
\]
Since this equation holds for any $V$, the integrand must vanish, and we have our first equation,
\begin{equation}\label{e.mass-conv}
\partial_{t}\rho + \divr(\rho\vu) = 0.
\end{equation}
Our next equation is to get the analog of $\bvec{F} = m\bvec{a}$.  Ignoring viscous effects, the net force on our fluid element (with volume $V$) is due to the pressure over its surface $P$ and the gradient of the gravitational potential $\Phi$:
\[
\int_{V}\rho \frac{\dif^{2}\bvec{r}}{\dif t^{2}}\,\dif V = \int_{V}\bvec{F}\,\dif V =  -\int_{V}\rho\grad\Phi\nsp\dif V - \int_{\partial V}P \nsp\dif \bvec{S}.
\]
Transforming the second integral on the right-hand side to a volume integral, and assuming that $\grad \Phi$ and $\grad P$ vary on macroscopic lengthscales, we arrive at an equation for the acceleration,
\begin{equation}\label{e.accel}
\frac{\dif^{2}\vr}{\dif t^{2}} = -\grad\Phi - \frac{1}{\rho}\grad P.
\end{equation}
where $\vr(t)$ is the position of the particle so that the left-hand side is the acceleration.
Here we must be careful: $\vu(\vx,t)$ refers to velocity of the fluid at a given point in space and a given instance of time, \emph{not} to the velocity of a given particle.  A fluid element can still accelerate even if $\partial_{t}\vu = \bvec{0}$ by virtue of moving a different location. At time $t$ this particle has the velocity
\begin{equation}\label{e.rdot}
\left.\frac{\dif\vr}{\dif t}\right|_{t} = \vu(\vx = \vr|_{t},t)
\end{equation}
where we use the fact that the particle is moving along a streamline of the fluid. At a slightly later time $h$, the particle has moved to a location $\vr(t + h) \approx \vr(t) + h\vu$, and the velocity is now
\begin{equation}\label{e.rdoth}
\left.\frac{\dif\vr}{\dif t}\right|_{t+h} = \vu(\vx = \vr|_{t+h},t+h)\approx \vu + h(\vu\cdot\grad\vu + \partial_{t}\vu),
\end{equation}
where we evaluate the derivatives at time $t$. Subtracting equation~(\ref{e.rdot}) from equation~(\ref{e.rdoth}) and dividing by $h$ gives us the acceleration; inserting this into Newton's law and dividing by volume gives us \emph{Euler's} equation of motion,
\begin{equation}\label{e.euler}
\partial_{t}\vu + \vu\cdot\grad\vu = -\grad \Phi - \frac{1}{\rho}\grad P.
\end{equation}
Equations~(\ref{e.mass-conv}) and (\ref{e.euler}) form the first two equations we need to describe stellar structure.

\section{Estimates of solar properties}

From equations~(\ref{e.mass-conv}) and (\ref{e.euler}) we are in a position to estimate, in an order-of-magnitude sense, many of the stellar properties.  First, let's consider the scale for each term in equation~(\ref{e.euler}),
\begin{center}\begin{tabular}{ccccccc}
$\displaystyle \partial_{t}\vu$ & + &
$\displaystyle  \vu\cdot\grad\vu$ & = &
$\displaystyle -\grad \Phi $ & $-$ & 
$\displaystyle \frac{1}{\rho}\grad P$\\
I & & II & & III & & IV
\end{tabular}
\end{center}
For a ``characteristic'' velocity $U$ and lengthscale $R$, we see that terms I and II are both of order $\sim U^{2}/R$ (the timescale is $R/U$).  For term III, we note that $GM/R^{2} = (GM/R)/R \sim U_{\mathrm{esc}}^{2}/R$, where $U_{\mathrm{esc}}$ is the escape velocity.  Finally, for term IV, $(P/\rho)/R \sim c_{s}^{2}/R$, where $c_{s}$ is the speed of sound.  Hence the typical scales of the terms are
\[
\textrm{I} : \textrm{II} : \textrm{III} : \textrm{IV} \sim U^{2} : U^{2} : U_{\mathrm{esc}}^{2} : c_{s}^{2}
\]
It is clear that the terms on the left-hand side are quite negligible, unless we are dealing with stellar explosions; in this case we must have the two terms on the right-hand side balance, and the star is in hydrostatic balance, 
\begin{equation}\label{e.hydrostatic}
\frac{\dif P}{\dif r} = -\rho \frac{GM(r)}{r^{2}}.
\end{equation}
Note that this does not mean that $\vu$ and $\bvec{a}$ are zero; it simply means that they are not important for establishing the mechanical structure of the star.

A side benefit of our argument about the scaling of the terms is that $c_{s}\sim U_{\mathrm{esc}} \sim (G\Msun/\Rsun)^{1/2}$.  We can use this to get an estimate of the central temperature of the sun in terms of \Msun\ and \Rsun: $T_{\sun,\mathrm{center}}\sim 10^{7}\nsp\K$, assuming that the equation of state is that of an ideal gas, $P = (n_{\mathrm{ion}} + n_{e})\kB T$ (see exercise 5). 

\subsection{A closer look at hydrostatic equilibrium}
If the center of the sun is indeed at a temperature $\sim 10^{7}\nsp\K$, then most of the gas should be ionized. Now electrons are much lighter than ions, so we might worry that the charges might separate.  If that were the case, an electric field would be established.   For a pure hydrogen plasma, then, we would have \emph{two} equations of hydrostatic equilibrium, one for the electrons and one for the protons,
\begin{eqnarray}
\grad P_{p} &=& m_{p}n_{\mathrm{H}} \bvec{g}  +  Z n_{p}e\bvec{E} \\
\grad P_{e} &=& m_{e}n_{e}\bvec{g} - n_{e} e \bvec{E}.
\end{eqnarray}
Here $\bvec{g} = -g \bvec{e}_{r}$ is the gravitational acceleration and $\bvec{E}$ is the electric field. Notice that if we \emph{presume} that the plasma is charge-neutral, then $\grad (P_{p}+P_{e}) = \rho \bvec{g}$, and we can solve for the electric field $\bvec{E}$. Of course, we must have some charge separation in order to establish the electric field in the first place, but one can show that the fractional charge separation needed is self-consistently small.

\subsection{A worked example: free-fall collapse}

It's worthwhile to imagine what would happen if we suddenly turned off pressure support in the sun, say by having a demon replace each particle with a non-interacting cold particle. For spherically symmetric collapse, let's follow the motion of an observer on the surface.  The mass interior to the observer is $M = \Msun$, so her equation of motion is
\begin{equation}\label{e.free-fall-eq-motion}
\frac{\dif u}{\dif t} = -\frac{GM}{r(t)^{2}}.
\end{equation}
Multiplying both sides by $u = \dif r/\dif t$ and integrating gives
\[
\frac{1}{2} u^{2} = GM\left(\frac{1}{r} - \frac{1}{R}\right),
\]
where $R = r(t=0)$. Defining $x = r/R$ gives
\begin{equation}\label{e.free-fall-non}
\frac{\dif x}{\dif t} = \left[2 \frac{GM}{R^{3}}\left(\frac{1}{x}-1\right)\right]^{1/2}.
\end{equation}
Now, $GM/R^{3}$ has dimension $[\textrm{time}^{-2}]$; furthermore, $M/R^{3} = 4\pi\bar{\rho}/3$, where $\bar{\rho}$ is the average density at the start of collapse.  (For the sun, $\bar{\rho} = 1.4\nsp\grampercc$, just a bit denser than you.) Hence, we can define the \textbf{dynamical timescale} as $t_{\mathrm{dyn}}\equiv (G\bar{\rho})^{-1/2}$.  For the sun, $t_{\mathrm{dyn}}\approx 1\nsp\unitstyle{hr}$.  Defining $\tau = t/t_{\mathrm{dyn}}$ in equation~(\ref{e.free-fall-non}) gives us a math problem,
\[
\frac{\dif x}{\dif\tau} = \left(\frac{8\pi}{3}\right)^{1/2}\left(\frac{1}{x}-1\right)^{1/2}
\]
which can be integrated from $x = 1$ to $x=0$ to give
\[
t_{\mathrm{collapse}} = \left(\frac{3\pi}{32}\right)^{1/2}t_{\mathrm{dyn}} \approx 0.5\nsp\unitstyle{hr}
\]
as the time for the sun to collapse if all pressure support were removed.

This is another way of looking at the derivation of eq.~(\ref{e.hydrostatic}): if terms III and IV are out of balance by even a small amount, the characteristic time for the star to mechanically adjust is very rapid.

\section{Energy considerations}\label{s.energy-considerations}

For a spherically symmetric gaseous body in hydrostatic equilibrium, the mass enclosed by radius $r$ satisfies the differential equation $\dif m/\dif r = 4\pi r^{2}\rho$.  Solving for $\rho$, substituting into the equation for hydrostatic balance, eq.~(\ref{e.hydrostatic}), and rearranging terms gives
\[
4\pi r^{3} \frac{\dif P}{\dif r} = -\frac{Gm(r)}{r} \frac{\dif m}{\dif r}.
\]
Integrating both sides from $r = 0$ to $r = R$, and changing variables on the right hand side from $r$ to $m$ gives
\begin{equation}\label{e.virial-1}
\int_{0}^{R} 4\pi r^{3}\frac{\dif P}{\dif r}\nsp\dif r = -3\int_{V} P\nsp \dif V = -\int_{0}^{M}\frac{Gm}{r(m)}\nsp\dif m = E_{\mathrm{grav}},
\end{equation}
where we integrated the left-hand side by parts, used the fact $P(R) \ll P(0)$, and replaced $4\pi r^{2}\nsp\dif r$ with $\dif V$. Now the pressure is related to the internal thermal (kinetic) energy per unit volume $U$.  For a non-relativistic ideal gas, $P = 2 U/ 3$; for  a relativistic gas, such as photons, $P = U/3$.  Defining $\gamma = (P + U)/U$, we can write the total energy of our gaseous sphere as
\begin{eqnarray}
E &=& E_{\mathrm{th}} + E_{\mathrm{grav}} = \int U \nsp\dif V-3\int P\nsp\dif V \nonumber\\
  &=& \frac{1 -3\left(\gamma - 1\right) }{\gamma - 1} \int P\nsp\dif V = \frac{3(\gamma-1)-1}{3(\gamma-1)} E_{\mathrm{grav}}.
\label{e.total-energy-gas}
\end{eqnarray}
This is the just an application of the virial theorem to our star.

As a first example, consider a star with the pressure provided by a non-relativistic ideal gas. Then $\gamma = 5/3$ (this is true even if the matter is degenerate) and the total energy is 
\[
E = \frac{1}{2}E_{\mathrm{grav}} < 0.
\]
The star is bound.  As a second example, consider a star that is so luminous that radiation pressure dominates. In this case, the pressure is that of a relativistic ideal gas. Then $\gamma = 4/3$ and $E = 0$: the star is marginally bound. We must worry about the stability of very luminous stars!

Now suppose the sun were to slowly contract, such that we can still assume hydrostatic equilibrium.  How long would this take?
The time needed to radiate away the thermal energy defines the \textbf{Kelvin-Helmholtz timescale},
\begin{equation}\label{e.K-H}
t_{\mathrm{KH}} \equiv \frac{E_{\mathrm{th}}}{L} \approx \frac{G\Msun^{2}}{2\Rsun L_{\sun}} = 16 \nsp\Mega\yr.
\end{equation}
We have written ``approximately'' because we made the approximation that $E_{\mathrm{grav}}  = -G\Msun^{2}/\Rsun$; in reality it is closer to $-(3/2) G\Msun^{2}/\Rsun$.
The estimated timescale is much less than the age of the earth, and fossils indicate that the sun has not changed dramatically on this timescale.  Hence there is an energy source needed to maintain the interior in thermal steady-state. The total energy per particle, integrated over the lifetime of the sun, is
\[ \frac{\Delta E}{N} \approx \frac{L_{\sun}\times 4.6\nsp\Giga\yr}{N} \approx 0.2\nsp\MeV. \]
This is much larger than chemical reactions could provide (typical energy scale is $1\nsp\eV$). The sun must be powered by nuclear reactions.

\section{Some analytical limits}

We can use the virial theorem of the previous section to set a few limits on the interior pressure and temperature of any star.  First, the mass $m(r)$ inside a volume of radius $r$ is
\[ m(r) = 4\pi\int_{0}^{r} \rho r^{2}\,\dif r, \]
so $\dif m/\dif r = 4\pi r^{2}\rho$. Combining this with the equation of hydrostatic equilibrium gives,
\[
\frac{\dif P}{\dif m} = -\frac{G m}{4\pi r^{4}}.
\]
Integrating this equation from the center, where $P = P_{c}$, to some radius $r$ gives
\begin{equation}\label{e.virial-integration-1}
P_{c} - P(r) = \frac{G}{4\pi}\int_{0}^{r} \frac{m \,\dif m}{r^{4}}.
\end{equation}
Now, the average density enclosed in a sphere of radius $r$ is $\bar{\rho}(r) = 3m(r)/(4\pi r^{3})$; solving for $r$ and inserting in equation~(\ref{e.virial-integration-1}) gives
\begin{equation}\label{e.virial-integration-2}
P_{c} - P(r) = \left(\frac{4\pi}{3}\right)^{4/3} \frac{G}{4\pi}\int_{0}^{r} \bar{\rho}(r)^{4/3} m^{-1/3}\,\dif m.
\end{equation}
Now, the density must decrease outward if the system is to be stable (you can't have heavy fluid on top of light!) and so the average density $\bar{\rho}(r)$ must also decrease outward. Hence,
\[ \rho_{c} \ge \bar{\rho}(r) \ge \bar{\rho}(R) = \frac{3M}{4\pi R^{3}}.\]
Inserting this inequality into equation~(\ref{e.virial-integration-2}) and evaluating at $r=R$ gives a constraint on the central pressure,
\begin{equation}\label{e.pressure-limits}
\frac{3}{8\pi} \frac{GM^{2}}{R^{4}} \le P_{c} \le \frac{1}{2} \left(\frac{4\pi}{3}\right)^{1/3} \rho_{c}^{4/3}G M^{2/3}.
\end{equation}
The critical point here is to notice the order-of-magnitude scale: $P_{c}\sim GM^{2}/R^{4}$. The only assumption in setting the limits (eq.~[\ref{e.pressure-limits}]) is that the density decreases outward.

\section{Transport of energy}\label{s.energy-transport-estimate}

We saw in \S\ref{s.energy-considerations} that at the current luminosity, the sun would take $\sim 16\nsp\Mega\yr$ to radiate away its internal (thermal) energy.  This raises an interesting question: what sets the luminosity?  To develop this idea further, let us write the luminosity as 
\[ \textrm{luminosity} \sim (\textrm{radiation energy stored in sun})/(\textrm{photon escape time}). \]
To get the radiation energy stored in the sun, we multiply the energy density of a thermal distribution of photons, at the central temperature of the sun, by the volume of the sun:
\begin{equation}\label{e.photon-energy-of-sun}
E_{\gamma} = aT_{c}^{4} \times \frac{4\pi}{3}\Rsun^{3}.
\end{equation}
What about the photon escape time? As a first try, suppose the sun were transparent, so that photons could freely stream out. Then the escape time would simply be $\Rsun/c$. This gives a ridiculously large luminosity.  (Of course, a transparent sun would also not produce a thermal spectrum, since there would be no way for the photons to come into thermal equilibrium with the matter.)

Suppose now instead that each photon can only travel a short distance $\ell$ before it is scattered into some random direction.  In a random walk, the total distance the photon travels to escape is $\Rsun(\Rsun/\ell)$.  In this case, the flux from the sun would be
\begin{equation}\label{e.mfp-estimate}
F = \frac{\Lsun}{4\pi\Rsun^{2}} \sim \frac{(4\pi/3)\Rsun^{3} aT_{c}^{4}}{4\pi\Rsun^{2}} \frac{c\ell}{\Rsun^{2}} = \frac{1}{3} \ell c \frac{aT_{c}^{4}}{\Rsun}.
\end{equation}
This is very crude, but we can use it to estimate that $\ell \sim 10^{-3}\nsp\cm$.  The average distance a photon can travel before being absorbed or scattered is called its \textbf{mean free path}.  Given this value of $\ell$ estimated from eq.~(\ref{e.mfp-estimate}), we can estimate the total number of scatterings a photon must suffer in escaping; it is a very large number, and the sun is quite opaque.

\section{Summary}

In summary, we've taken the observed gross properties of the sun, the equation of motion for a fluid, and the ideal gas equation of state; from these we've deduced that the sun is in hydrostatic balance, that its interior temperature is of order $10^{7}\nsp\K$, and that it would radiate away its thermal energy and contract within about 10\nsp\Mega\yr\ if there were no nuclear reactions in its core.  We have developed now a crude picture of the sun: it is a mass of plasma that is in hydrostatic equilibrium, with pressure gradients supporting the inward pull of gravity.  It is very opaque, and thus it acts as a reservoir for photons with a thermal, Planckian distribution. Because it is opaque, the photons leak out very slowly.  This slow leakage represents a loss of thermal energy; the thermal energy is replenished by heat liberated from nuclear reactions.

What comes next is fleshing out the detailed physics implied by these considerations: an equation of state to relate the pressure to the density and temperature; photon scattering and absorption cross sections to compute the heat transport; nuclear reaction rates to determine the thermal steady-state and the gradual change in composition of the interior.

\section{Exercises}\label{s.intro-exercises}

\begin{enumerate}
\item What is the mean density of the sun? What is the luminous flux (energy/area/time) at 1~AU? What is the orbital period of a test mass just exterior to the radius of the sun?

\item Consider a planar atmophere, in which $-\grad\Phi = \bvec{g} = -g \bvec{e}_{z}$ with $g$ constant. Thus the equation of hydrostatic equilibrium (eq.~[\ref{e.hydrostatic}]) is
\begin{equation}\label{e.planar-hydrostatic}
\frac{\dif P}{\dif z} = -\rho g.
\end{equation}
Suppose we have an isothermal ideal gas, $P = \rho\kB T/(\mu\mb)$, where $T$ is the temperature, $\kB$ is Boltzmann's constant, and $\mu\mb$ is the mass of particles in the gas ($\mb$ is the atomic mass unit), so that the number of particles per unit volume is $N/V = \rho/(\mu\mb)$.  Show that for such a gas the density decreases as
\[
\rho(z) = \rho(0) \exp\left(-z/H\right)
\]
and find an expression for the \emph{scale height} $H$.  Evaluate $H$ for conditions at sea level on Earth. Does the value make sense? Now evaluate $H$ under conditions appropriate for the solar photosphere; in this case what is $H/R_{\sun}$?

\item Equation~(\ref{e.hydrostatic}) must in general be solved numerically for a real equation of state $P = P(\rho)$, but it is useful to construct a toy model to gain insight.  Suppose the sun has a density profile
\[ \rho(r) = \rho_{0}\left(1-\frac{r}{\Rsun}\right) \]
where $\rho_{0}$ is the central density. Further suppose that the equation of state is that of an ideal gas with mean molecular weight $\mu$.  Find the central density, pressure, and temperature in terms of $\Msun$, $\Rsun$, and $\mu$. How do they compare with the values for a constant density star?  Evaluate them numerically for a solar composition (hydrogen mass fraction of 0.7).  Keeping $M$ and $R$ fixed, what happens to the central temperature if the composition is transformed to pure helium? If the nuclear reaction rate depends on temperature, what would this do the luminosity, in the absence of any other changes?

\item Using equation~(\ref{e.mass-conv}), show that equation~(\ref{e.euler}) can be written as
\begin{equation}\label{e.momentum-conv}
\partial_{t}(\rho u_{i}) + \partial_{j}(\rho u_{i}u_{j}) = -\rho\partial_{i}\Phi - \partial_{i}P,
\end{equation}
where the subscripts $i$ denote components and repeated subscripts are understood to be summed over. Interpret the terms on the left-hand side in terms of conservation of momentum.

\item A side benefit of our argument about the scaling of the terms is that $c_{s}\sim U_{\mathrm{esc}} \sim (G\Msun/\Rsun)^{1/2}$.  Use this to get an estimate of the central temperature of the sun in terms of \Msun\ and \Rsun, assuming the composition is an ideal ionized hydrogen plasma.  What is the numerical value of the temperature?

\item Compute the mean kinetic (thermal) energy per hydrogen nucleus in the sun, and express it in electron volts. How does this compare to the binding energy of a hydrogen atom?


\item Consider a fully ionized hydrogen plasma in a gravitational field in planar geometry. 
\begin{enumerate}
\item Argue that in the absence of an electric field, the protons would sink to the bottom of the atmosphere. Show that if the atmosphere is to remain charge neutral, then an electric field
\[
	\bvec{E} = -\frac{1}{2}\frac{\mb}{e}\bvec{g},
\]
must be present. Compare this field to that between the proton and electron in an atom.  Could this external field be detectable, by Stark effect for example?

\item Suppose a trace ion of charge $Z'e$ and mass $A'\mb$ is	 introduced.  What is the net force on this ion?

\item In order to have an electric field, there must be some charge separation.  Quantify this: define a parameter
\[ \delta \equiv \frac{n_{e}-n_{p}}{n_{e} + n_{p}} \]
and estimate its magnitude.  \emph{Hint:} Use Poisson's equation for both the gravitational and electrostatic potentials, and the results of part (a).
\end{enumerate}

\item If we regard the sun as a large cavity filled with photons, estimate the total energy stored in the radiation field.  If the sun were suddenly to become completely transparent, what would be the resulting luminosity?

\end{enumerate}


% !TEX root = ./notes.tex
\chapter[Equation of State]{The Equation of State for an Ideal Gas}
\newcommand{\vp}{\ensuremath{\bvec{p}}}
\renewcommand{\pF}{\ensuremath{p_{\mathrm{F}}}}
\newcommand{\eF}{\ensuremath{\varepsilon_{\mathrm{F}}}}

\section{An Ideal Fermi Gas}
In statistical equilibrium, we can describe a system of non-interacting particles by a distribution function $f(p)d^{3}p$ such that the number of particles per unit volume is
\begin{equation}\label{e.dist}
n = \int d^{3}\vp f(\vp),
\end{equation}
where $\vp$ is the momentum. (Because there is no dependence on spatial coordinates, the integration over volume is trivial.) From equation~(\ref{e.dist}), we can get our other thermodynamic quantities, for example
\begin{eqnarray}
\frac{E}{V} \equiv u = \int  \varepsilon(\vp) f(\vp)\,d^{3}\vp &\qquad& \textrm{energy per unit volume}\label{e.u}\\
P = \int  (\vp\cdot\bvec{e}_{z})(\bvec{v}\cdot\bvec{e}_{z}) f(\vp)\,d^{3}\vp &\qquad& \textrm{pressure},\label{e.P}
\end{eqnarray}
where $\varepsilon$ is the particle energy and $v$ the velocity.

For \emph{fermions}, particles with half-integer spin, it can be shown that
\begin{equation}\label{e.fermion}
f(p) = \frac{g}{(2\pi\hbar)^{3}} \left[\exp\left(\frac{\varepsilon-\mu}{\kB T}\right)+1\right]^{-1}.
\end{equation}
In this equation $\varepsilon(p)$ is the energy of a particle, $\mu$ is the \emph{chemical potential}, $T$ is the temperature, and $g$ denotes the number of particles that can occupy the same energy level (for spin-1/2 particles, $g=2$).  The connection to thermodynamics is via the relations
\[
\frac{1}{T} = \left(\frac{\partial S}{\partial E}\right)_{N,V},\qquad -\frac{\mu}{T} = \left(\frac{\partial S}{\partial N}\right)_{E,V},
\]
and 
\[
TS = PV - \mu N + E;
\]
these are derived in standard texts. Let's explore what happens in various limits.  

\section{Non-degenerate limit}
First, let's take $K\equiv\exp(\mu/\kB T)\ll 1$. (In the literature, $K$ is called the \emph{fugacity}.) Then in equation~(\ref{e.fermion}), we see that the exponential term dominates.  If our system is isotropic, then $d^{3}p = 4\pi p^{2}dp$, and we'll use this substitution from now on.  We then have for the number density
\begin{equation}
n(\mu,T) = \frac{gK}{2\pi^{2}\hbar^{3}}\int_{0}^{\infty}\exp\left(-\frac{\varepsilon}{\kB T}\right) p^{2}dp.
\end{equation}
To do the integral, notice that since $2m\varepsilon = p^{2}$, we have $p^{2}\,dp = m(2m\varepsilon)^{1/2}\,d\varepsilon$; making the substitution $x = \varepsilon/(\kB T)$, we get
\begin{equation}\label{e.n-int}
n(\mu,T) = \frac{gK}{2\pi^{2}\hbar^{3}}\sqrt{2}(m\kB T)^{3/2}\int_{0}^{\infty} x^{1/2}e^{-x}\,dx.
\end{equation}
You have all struggled with this integral in your past, but to avoid unpleasant flashbacks, I will just tell you that it is $\sqrt{\pi}/2$.  So, we have our first result (but we still don't know what it means),
\begin{equation}\label{e.n-nondeg}
n(\mu,T) = K\left[g\left(\frac{m\kB T}{2\pi\hbar^{2}}\right)^{3/2}\right].
\end{equation}

Let's forge on a little further, though, and try to get the energy per unit voume $u$.  Once we have $u$, we know we can get the pressure from the relation for a non-relativistic gas, $P = 2/3\;u$.  Using equations~(\ref{e.u}) and (\ref{e.fermion}),
\begin{equation}
u(\mu,T) = \frac{gK}{2\pi^{2}\hbar^{3}}\int_{0}^{\infty}\exp\left(-\frac{\varepsilon}{\kB T}\right) \varepsilon p^{2}dp.
\end{equation}
Let's repeat our trick of changing variables from $p$ to $x = \varepsilon(p)/\kB T$; we then have
\begin{equation}
u(\mu,T) = \frac{gK}{2\pi^{2}\hbar^{3}}\sqrt{2}m(m\kB T)^{3/2}\int_{0}^{\infty} x^{3/2}e^{-x}\,dx.
\end{equation}
Did you notice that if we integrate by parts,
\[
\int_{0}^{\infty} x^{3/2}e^{-x}\,dx =  \frac{3}{2}\int_{0}^{\infty}x^{1/2}e^{-x}\,dx = \frac{3}{4}\sqrt{\pi},
\]
we get the integral we already solved in equation~(\ref{e.n-int})?  Putting everything together, we have
\begin{equation}
u(\mu,T) = \frac{3}{2}\left\{K\left[g\left(\frac{m\kB T}{2\pi\hbar^{2}}\right)^{3/2}\right] \right\}\kB T = \frac{3}{2}n\kB T.
\end{equation}
which gives us the pressure,
\begin{equation}\label{e.PVnkT}
P = n \kB T.
\end{equation}
Whoo-hoo!  We've rediscovered the ideal gas.

Now we have to understand this chemical potential $\mu$.  We can solve equation~(\ref{e.n-nondeg}) for $\mu$,
\begin{equation}\label{e.fugacity-nondeg}
\exp\left(\frac{\mu}{\kB T}\right) = K = n\left[g\left(\frac{m\kB T}{2\pi\hbar^{2}}\right)^{3/2}\right]^{-1}.
\end{equation}
Now $K$ is dimensionless, a number, so the thing in $[\;]$ must have dimensions of number density.  Let's call it $n_{Q}$.  To understand the significance of $n_{Q}$, let's calculate the uncertainty in position of a particle having energy $\kB T$; from Heisenberg, we have
\[
\Delta x \approx \frac{\hbar}{\Delta p} \sim \frac{\hbar}{\sqrt{m\kB T}}
\]
where I am dropping numerical factors and I've made the substitution $\Delta p\sim p \approx \sqrt{m\kB T}$.  Now what happens if I pack the particles so that on average there are $g$ particles per box of volume $(\Delta x)^{3}$?  In that case the density would be $n = g(\sqrt{m\kB T}/\hbar)^{3} \approx n_{Q}$.  So, what appears in the chemical potential is the ratio of the density to that density at which the particles are packed so closely that the uncertainty in their positions is the same size as the typical inter-particle spacing.
In the ideal-gas limit $K \ll 1$, which makes sense: $n\ll n_{Q}$, so the particles are very far apart compared to their thermal de Broglie wavelengths, and quantum effects ought to be unimportant.

\section{Degenerate limit for Fermions}\label{s.deg-limit-fermions}
When $n\gtrsim n_{Q}$, we can no longer use the approximation $K \ll 1$, so let's go to the opposite limit, for which $\mu \gg \kB T$.  In this case, notice from equation~(\ref{e.fermion}) that
\begin{equation}
f(p) \approx \frac{g}{(2\pi\hbar)^{3}}\left\{\begin{array}{lr} 1 & \varepsilon < \mu\\ 0 & \varepsilon> \mu\end{array}\right. .
\end{equation}
Physically, we are putting in each energy level $g$ particles, starting with the lowest energy states, until we used up all of our particles (at which point we have reached the level with energy $\varepsilon \approx \mu$).  The only levels that will be partially filled will be those lying in a thin band $\varepsilon \approx\mu \pm \kB T$.  If that is the case, we can make the following approximation.  Let's take the limit $T \to 0$, and define the \emph{Fermi energy} by $\eF = \mu|_{T\to 0}$ and the \emph{Fermi momentum} by $\pF = \sqrt{2m\eF}$.  We can then write equation~(\ref{e.dist}) as
\begin{equation}
n(\mu) = \frac{1}{\pi^{2}\hbar^{3}}\int_{0}^{\pF} p^{2} dp,
\end{equation}
where I took $g = 2$ (it is electrons we will be worried about here).  Now this is an easy integral,
\begin{equation}\label{e.n-deg}
n(\mu) = \frac{\pF^{3}}{3\pi^{2}\hbar^{3}} = \frac{(2m\eF)^{3/2}}{3\pi^{2}\hbar^{3}},
\end{equation}
or $\mu \approx \eF = (3\pi^{2}n)^{2/3} \hbar^{2}/(2m)$.  Let's get the energy per unit volume and the pressure,
\begin{equation}
u(\mu) = \frac{1}{\pi^{2}\hbar^{3}}\int_{0}^{\pF} \frac{p^{2}}{2m}\,p^{2}\,dp = \frac{\pF^{5}}{5\pi^{2}\hbar^{3}}.
\end{equation}
Comparing this with equation~(\ref{e.n-deg}), we have
\begin{eqnarray}
u &=& \frac{3}{5}n\eF,\label{e.u-deg}\\
P &=& \frac{2}{5}n\eF\label{e.P-deg}.
\end{eqnarray}
Note that to lowest order, neither $u$ nor $P$ depend on $T$.  Substituting for \eF\ in equation~(\ref{e.P-deg}) gives us the equation of state,
\begin{equation}\label{e.eos-deg}
P = \frac{2}{5}\left(3\pi^{2}\right)^{2/3}\frac{\hbar^{2}}{2m}n^{5/3}.
\end{equation}

Notice that in equation~(\ref{e.fugacity-nondeg}), $n_{Q}\propto m^{3/2}$.  This means that at any given temperature, $n_{Q}$ for electrons is $1836^{3/2} = 80,000$ times smaller than it is for protons, not to mention helium or heavier nuclei. As a result, the electrons will become degenerate ($n\gtrsim n_{Q}$) at a much lower mass density than the ions.    A common circumstance, then, is to have a mixture of degenerate electrons and ideal ions (we will deal with non-ideal corrections due to electric forces later).  

Now, let's estimate the boundary between the non-degenerate regime and the degenerate one. At a given temperature, we know in the low-density limit that the electrons obey the ideal gas law (eq.~[\ref{e.PVnkT}]) and in the high-density limit the electrons are degenerate (eq.~[\ref{e.eos-deg}]). So, let's extrapolate our two limiting expressions for the pressure and see where they meet,
\begin{equation}
n_{e}\kB T = P_{e,\mathrm{ideal}} \sim P_{e,\mathrm{deg.}} 
  = \frac{2}{5}n_{e}\eF,
\end{equation}
or $\eF \approx \kB T$.  No surprise here.  Notice that the ratio
\begin{equation}
\frac{\eF}{\kB T} \sim \frac{(3\pi^{2})^{2/3}\hbar^{2}}{2m_{e}\kB T}n_{e}^{2/3} \sim \left(\frac{n_{e}}{n_{Q}}\right)^{2/3},
\end{equation}
so marking the onset of degeneracy with $\eF\sim \kB T$ also makes sense from that aspect as well. Our boundary in the density-temperature plane between the non-degenerate and degenerate regimes is then determined by setting $\kB T = \eF$,
\begin{equation}\label{e.TF}
T = \frac{(3\pi^{2})^{2/3}\hbar^{2}}{2m_{e}\kB }\left(\frac{Y_{e}\rho}{\mb}\right)^{2/3} = 3.0\ee{5}\nsp\K \left(Y_{e}\rho\right)^{2/3}.
\end{equation}
If the temperature falls below this value, the electrons will be degenerate. Here $Y_{e}$ is the electron molar fraction, or electron abundance and \mb\ is the atomic mass unit; consult Appendix~\ref{s.composition} for details.

\section{Fermi-Dirac integrals}
This condition for the onset of degeneracy, eq.~(\ref{e.TF}), is only a rule-of-thumb; in any serious calculation we would want to calculate the electron thermal properties from the exact integrals
\begin{eqnarray}\label{e.FD12}
n(\mu,T) &=& \frac{\sqrt{2}(m\kB T)^{3/2}}{\pi^{2}\hbar^{3}}\int_{0}^{\infty} \frac{x^{1/2}\,dx}{\exp(x-\psi)+1} \\
P(\mu,T) &=& \frac{(2m\kB T)^{3/2}(\kB T)}{3\pi^{2}\hbar^{3}}\int_{0}^{\infty} \frac{x^{3/2}\,dx}{\exp(x-\psi)+1},
\end{eqnarray}
where $\psi = \mu/(\kB T)$.  These integrals cannot be done analytically, but they occur so frequently that there are many published tables and numerical approximation schemes \citep[see][]{timmes.swesty:accuracy}.  Specifically, the \emph{non-relativistic Fermi-Dirac integral of order $\nu$} is defined as
\begin{equation}\label{e.FDintegral}
F_{\nu}(\psi) = \int_{0}^{\infty}\frac{ x^{\nu}\,dx}{\exp(x-\psi)+1}.
\end{equation}
One can (numerically) invert equation~(\ref{e.FD12}) to solve for the chemical potential $\psi \kB T$.

\section{Relativistic photon gas}
Photons are \emph{bosons}---they have spin 1. For bosons, the distribution function is similar to that in equation~(\ref{e.fermion}), but with the $+1$ replaced by $-1$ in the denominator. In addition, photon number is not conserved: one can freely create and destroy photons.  This implies that their chemical potential is zero.  Also, $g=2$ for photons: there are two independent polarization modes. Putting all of these together, we can write energy per unit volume as
\begin{equation}\label{e.boson-dist}
u = \frac{1}{\pi^{2}\hbar^{3}}\int_{0}^{\infty}\varepsilon p^{2}\left[\exp\left(\frac{\varepsilon}{\kB T}\right)-1\right]^{-1}\,dp.
\end{equation}
Now, use the fact that $p = \varepsilon/c$ and change variables to $x = \varepsilon/(\kB T)$ to get
\[
 u = \frac{\kB ^{4}T^{4}}{\pi^{2}c^{3}\hbar^{3}}\int_{0}^{\infty} \frac{x^{3}\,dx}{e^{x}-1}.
\]
This integral is a classic and is equal to $\pi^{4}/15$.  Hence the energy per unit volume and the pressure are
\begin{eqnarray}\label{e.radiation-eos}
u &=& \left(\frac{\kB ^{4}\pi^{2}}{15 c^{3}\hbar^{3}} \right) T^{4} = aT^{4}\\
P &=& \frac{1}{3} aT^{4}.
\end{eqnarray}
In CGS units, $a = 7.566\ee{-15}\nsp\ergs\usp\cm^{-3}\usp\K^{-4}$.

\section{Connection to thermodynamics}
Once we have the distribution functions, we can get all of the other thermodynamic properties from the thermodynamic relations: in what follows let $N = nV$ be the total number of particles, with $V$ being the volume of the system.  The total energy is then $E = uV$, and the entropy is $S$, and we have
\begin{eqnarray}
A = E - TS, &\qquad& \textrm{Helmholtz free energy,}\\
H = E + PV, &\qquad& \textrm{Enthalpy,}\\
\mu N = G = A + PV, &\qquad& \textrm{Gibbs free energy}.
\end{eqnarray}
For example, we have in the non-degenerate limit that
\begin{equation}
\mu = \kB T\ln K = \kB T\ln\left[ \frac{n}{g}\left(\frac{2\pi\hbar^{2}}{m\kB T}\right)^{3/2}\right],
\end{equation}
and so we could write the entropy per unit mass as
\begin{eqnarray}
s \equiv \frac{S}{Nm} &=& \frac{1}{Nm}\frac{E + PV - \mu N}{T}\nonumber\\
 &=& \frac{\kB }{m}\left\{ \frac{5}{2} + \ln\left[\frac{g}{n}\left(\frac{m\kB T}{2\pi\hbar^{2}}\right)^{3/2}\right]\right\}.
 \end{eqnarray}
In this equation I have used the ideal non-degenerate values $E = (3/2) N\kB T$, $PV = N\kB T$ and have denoted the mass per particle as $m$ and the degeneracy of the spin-states as $g$.

\section{Exercises}\label{s.EOS-exercises}
\begin{enumerate}
\item For an isotropic momentum distribution, show that 
\[
P = \frac{1}{3}\int |p||v| f(\vp)\,\dif^{3}\vp.
\]
\item Now show that for a non-relativistic gas, $P = (2/3)E/V$, and that for a relativistic gas, $P = (1/3)E/V$.

\item Repeat the derivation of equation~(\ref{e.u-deg}) and (\ref{e.P-deg}) for  a relativistic Fermi gas. What is the expression for the temperature at which the gas becomes degenerate (cf.~eq.~[\ref{e.TF}]) in this case?
\end{enumerate}

% !TEX root = ./notes.tex
\chapter{Plasma Physics}

Under the conditions in a stellar interior, most of the atoms are ionized, and the stellar matter consists of positively charged nuclei and ions and negatively charged electrons. More precisely,
a \emph{plamsa} is defined as a gas of charged particles in which the kinetic energy of a typical particle is much greater than the potential energy due to its nearest neighbors. To make this quantitative, consider a gas with only one species present, with charge $q$.  Let the mean spacing between particles be $a$; clearly the number density of such particles is $n = (4\pi a^{3}/3)^{-1}$.  We may then take the quantity
\begin{equation}\label{e.Gamma-definition}
\Gamma \equiv \frac{q^{2}}{a\kB T}
\end{equation}
as indicating the relative importance of potential to kinetic energy.  In a classical plasma, $\Gamma \ll 1$.  Note, however, that systems with $\Gamma > 1$ are often (confusingly) called  \emph{strongly coupled plasmas}. The meaning is usually clear from context.

\section{Debye shielding}\label{s.plasma-shielding}

Imagine a typical charged particle in a plasma.  Very close to the particle, we expect the electrostatic potential to be that of an isolated charge $\Phi = q/r$. Far from the particle, there will be many other particles surrounding it, and we may expect that the potential to be screened. For example, a positive ion will tend to attract electrons to be somewhat closer, on average to it: we say that the ion \emph{polarizes} the plasma.  As a result of this polarization, the potential of any particular ion should go to zero much faster than $1/r$ due to the ``screening'' from the enhanced density of opposite charges around it.

Let's consider a plasma having many ion species, each with charge $Z_{i}$, and elections. About any selected ion $j$,  particles will arrange themselves according to Boltzmann's law,
\begin{equation}\label{e.ion-boltzmann}
n_{i}(r) = n_{i0}\exp\left[-\frac{Z_{i}e\Phi(r)}{\kB T}\right].
\end{equation}
Here $n_{i0}$ is the density of particle $i$ far from the charge $j$, and $r$ is the distance between particles $i$ and $j$.  (A similar equation holds for the electrons, with $Z$ replaced by $-1$.) To solve for the potential, we can use Poisson's equation,
\begin{equation}\label{e.Poisson}
\nabla^{2}\Phi = -4\pi \sum_{i} Z_{i}e n_{i}(r) +4\pi e n_{e}(r).
\end{equation}
Our assumption is that the term in the exponential of equation~(\ref{e.ion-boltzmann}) is small, so we may expand it to first order in $\Phi$ and substitute that expansion into equation~(\ref{e.Poisson}) to obtain in spherical geometry
\[
 \frac{1}{r}\frac{\partial^{2}}{\partial r^{2}}\left(r\Phi\right) = -4\pi e \left[\sum_{i} n_{i0}Z_{i}\left(1-\frac{Z_{i}e\Phi}{\kB T}\right) - n_{e0}\left(1 + \frac{e\Phi}{\kB T}\right)\right].
\]
The overall charge neutrality of the plasma implies that $n_{e0} = \sum_{i}Z_{i}n_{i0}$; using this to simplify the above equation gives
\begin{equation}\label{e.linearized-Poisson}
\frac{1}{r}\frac{\partial^{2}}{\partial r^{2}}\left(r\Phi\right) = \left[\frac{4\pi e^{2}}{\kB T}\sum_{i} n_{i0}\left(Z_{i}^{2}+Z_{i}\right)\right]\Phi \equiv \lambdaD^{-2}\Phi.
\end{equation}
The quantity in $[\, ]$ has dimensions of reciprocal length squared and we define it as $(1/\lambdaD)^{2}$ with $\lambdaD$ being called the \emph{Debye length}.

Multiplying equation~(\ref{e.linearized-Poisson}) by $r$, integrating twice, and determining the constant from the condition that as $r\to 0$, $\Phi \to Z_{j}e/r$ gives the self-consistent potential
\begin{equation}\label{e.screened-potential}
\Phi = \frac{Z_{j}e}{r}\exp\left(-\frac{r}{\lambdaD}\right).
\end{equation}
The Debye length \lambdaD\ determines the size of the screening cloud around the ion.

In order for the above derivation to be valid, we require that $\lambdaD \gg a$, where $a$ is the mean ion spacing; otherwise, there won't be any charges in our cloud to screen the potential! Equivalently, we require the number of particles in a sphere of radius \lambdaD\ to be large,
\begin{equation}\label{e.plasma-parameter}
\frac{4\pi}{3}\lambdaD^{3} \sum_{i} n_{i} \gg 1.
\end{equation}
This condition must hold if we are to treat the gas as a plasma.

\section{Corrections to the ideal gas EOS}\label{s.plasma-corrections}

In a plasma the particles are not independent: shake one particle and other nearby particles will shake as well. In statistical mechanics, this requires introducing \emph{correlation functions} to derive the equation of state.  We'll adopt the more intuitive approach of Debye and H\"uckel to get the lowest-order correction to the ideal gas EOS.  First, the total electrostatic energy in a volume $V$ is
\begin{equation}\label{e.electrostatic-energy}
E_{\mathrm{Coul}} = \frac{1}{2}V\sum_{j}Z_{j}e n_{j}\Phi_{j}.
\end{equation}
Here $\Phi_{i}$ is the potential at a particle $j$ \emph{due to all the other particles in the plasma.}  Now, we computed the total potential around a particle (eq.~[\ref{e.screened-potential}]); expanding and subtracting off the self-potential of particle $j$ gives $\Phi_{j} = -Z_{j}e/\lambdaD$.  Inserting this into equation~(\ref{e.electrostatic-energy}) and expanding gives
\begin{equation}\label{e.electrostatic-energy-2}
E_{\mathrm{Coul}} \approx - V\left(\frac{\pi}{\kB T}\right)^{1/2} e^{3} \left[\sum_{i}n_{i0} \left(Z_{i}^{2} + Z_{i}\right)\right]^{3/2}.
\end{equation}
This energy is to be added to the kinetic energy of the gas.  The effect of the electrostatic interactions is to \emph{decrease} the energy in the gas, that is, to make it more bound.

In order to get the pressure, we first must find the Helmholtz free energy $A$ (we can't directly differentiate equation~[\ref{e.electrostatic-energy-2}] because it isn't in terms of $S$ and $V$). Integrating the thermodynamical identity $E = -T^{2}(\partial/\partial T)(A/T)$ and then taking $P = -(\partial A/\partial V)_{T,N}$ gives
\begin{equation}\label{e.pressure-correction}
 P_{\mathrm{Coul}} \approx -\frac{e^{3}}{3}\left(\frac{\pi}{\kB T}\right)^{1/2}\left[\frac{\left(\langle Z^{2}\rangle+\langle Z\rangle \right)\rho}{\langle A\rangle\mb}\right]^{3/2}.
\end{equation}
The effect of Coulomb interactions is to decrease the pressure below the ideal gas value.

\section[Strongly Coupled Plasmas]{Coulomb corrections when the electrons are degenerate}\label{s.degenerate-coulomb}

The above discussion holds only when both the electrons and ions are non-degenerate.  What happens when the electrons are degenerate?  In that case the kinetic energy is of order the Fermi energy, not the temperature.  We might think to replace $\kB T$ with \eF\ in equation~(\ref{e.Gamma-definition}). Recalling the formula for \eF\ from \S~\ref{s.deg-limit-fermions}, we have the condition for weak interactions,
\begin{equation}\label{e.condition-weak-degen}
\frac{e^{2}}{a \eF} = \left(\frac{4\pi n}{3}\right)^{1/3}\left(\frac{m_{e} e^{2}}{\hbar^{2}}\right)\frac{2}{(3\pi^{2} n)^{2/3}} < 1.
\end{equation}
Here $m_{e}$ and $n$ denote, respectively, the electron mass and number density.  Do you recognize the quantity $\hbar^{2}/(m_{e}e^{2})$?  It is the Bohr radius, \abohr.  What is \abohr\ doing in this equation?  Well, we are looking for a quantum mechanical system in which the Coulomb interaction is comparable to the non-relativistic kinetic energy.  Does that sound like any system you've seen before?

Cleaning up equation~(\ref{e.condition-weak-degen}), our condition for the electrons to be weakly interacting when degenerate is
\begin{equation}
\left(\frac{2^{5/3}}{3\pi}\right)\left( n \abohr^{3}\right)^{-1/3} < 1,
\end{equation}
or, in terms of mass density $\rho$ and electron fraction $Y_{e}$, $(Y_{e}\rho) > 0.4\nsp\grampercc$.  As the density increases, the electron gas becomes more ideal, that is, the electrostatic interaction matters less and less.  

Just to complete the discussion on electrons, what if the electrons are relativistic?  In this case, $\eF = \pF c = (3\pi^{2}n)^{1/3}\hbar c$, and
\begin{equation}\label{e.condition-weak-rel-degen}
\frac{e^{2}}{a \eF} = \left(\frac{4}{9\pi}\right)^{1/3}\left(\frac{e^{2}}{\hbar c}\right) = 3.8\ee{-3}.
\end{equation}
In this case $\eF \propto n^{1/3}$ so the density dependence cancels.  You will note the appearance of the fine structure constant $\alpha_{\mathrm{F}} = e^{2}/(\hbar c)$, as you might have expected when dealing with relativistic electrons and electrostatics.

Under astrophysical conditions, we can almost always regard degenerate electrons as being ideal.  What about the ions?  They are not usually degenerate under conditions of interest. We can get a simple expression if we go to the opposite limit, in which the electrons are very degenerate.  In that case, the electrons are an ideal gas and hence have uniform density. If the temperature is low enough, the ions will have $Z^{2}e^{2}/(a\kB T) \gg 1$; in this case we might expect the ions will arrange themselves into a lattice that maximizes the inter-ionic spacing.

To get an estimate of the electrostatic energy, let's compute the energy of a charge-neutral sphere centered on a particular ion of charge $Z_{i}$.  Because the electrons have a uniform density, $Y_{e}\rho/\mb$, we can find the radius of the sphere $a$ by requiring it to have $Z_{i}$ electrons,
\begin{equation}\label{e.radius-ion-sphere}
\frac{4\pi}{3}a^{3}\left(Y_{e}\frac{\rho}{\mb}\right) = Z_{i},
\end{equation}
or $a = [3 Z_{i} \mb/(4\pi Y_{e}\rho)]^{1/3}$.  The potential energy of this sphere has two components. The first is due to electron-electron interactions,
\begin{equation}\label{e.ee-energy}
E_{ee} = \int_{0}^{a} \frac{q(r)\, \dif q}{r} = \frac{3}{5}\frac{Z_{i}^{2}e^{2}}{a},
\end{equation}
where $q(a) = Z_{i}e(r/a)^{3}$ is the charge in a sphere of radius $r < a$. The second component of the potential energy is due to the ion-electron interaction,
\begin{equation}\label{e.ei-energy}
E_{ei} = -Z_{i}e\int_{0}^{a}\frac{\dif q}{r} = -\frac{3}{2}\frac{Z_{i}^{2}e^{2}}{a}.
\end{equation}
Combining equations~(\ref{e.ee-energy}), (\ref{e.ei-energy}), and (\ref{e.radius-ion-sphere}) gives the total electrostatic energy for a single ion-sphere,
\begin{equation}\label{e.madelung-sphere}
E = -\frac{9}{10} \frac{Z_{i}^{2}e^{2}}{a} = -\frac{9}{10} Z_{i}^{5/3} e^{2} \left(\frac{4\pi}{3}\frac{Y_{e}\rho}{\mb}\right)^{1/3}.
\end{equation}
Multiplying this by $n_{i} = Y_{i}\rho/\mb$, summing over all ion species $i$, and defining $\langle Z^{5/3}\rangle = n^{-1}\sum Z_{i}^{5/3}n_{i}$ where $n = \sum n_{i}$, gives the total Coulomb energy per volume,
\begin{equation}\label{e.madelung-total}
E_{\mathrm{Coul}} = -\frac{9}{10} n\kB T \left[\frac{\langle Z^{5/3}\rangle e^{2}}{\kB T}\left(\frac{4\pi}{3}\frac{Y_{e} \rho}{\mb}\right)^{1/3}\right].
\end{equation}
Notice that the quantity in $[\nsp]$ reduces to $\Gamma$ for a single-species plasma.  If we therefore define $\Gamma \equiv [\nsp]$ for a multi-component plasma, we have $E_{\mathrm{Coul}} \approx -0.9 n\kB T \Gamma$; the pressure is then $P_{\mathrm{Coul}} = E_{\mathrm{Coul}}/3  = -0.3 n \kB T \Gamma$. This holds in the limit $\Gamma \gg 1$.

\section{Collisions}\label{s.plasma-collisions}

Without collisions, a plasma cannot reach thermodynamic equilibrium, and the rate of collisions mediates both the approach to equilibrium and the transport of quantities, such as heat, in a forced system. In this section, we'll make an estimate for the rate of electron-electron ion-ion, and electron-ion collisions.

To begin, let's imagine a light particle (electron) colliding with a much heavier, fixed particle (an ion), as illustrated in Figure~\ref{f.scattering}.  (This picture also applies to a pseudo particle of reduced mass scattering in a fixed potential.)  Let the impact parameter be $b$, and the mass of the incident particle is $\mu$.  For Coulomb interactions, the force on the particle is $(q_{1}q_{2}/r^{2})\bvec{\hat{r}}$. The incident momentum is $p_{0}$. Now by assumption, in our plasma most of the interactions are weak (potential energy is much less than kinetic), so let's treat the deflection of the particle as a perturbation.  That is, we shall assume that $p_{0} = \textrm{const}$ and that the effect of the interaction is to produce a perpendicular (to $p_{0}$) component of the momentum $p_{\bot}$.  The total change in $p_{\bot}$ is then
\[ p_{\bot} = \int_{-\infty}^{\infty}\dif t\; \frac{q_{1}q_{2}}{r^{2}}\sin\theta, \]
where $\sin\theta = b/r$ is the angle that the radial vector makes with the horizontal.  Substituting $r = b/\sin\theta$ and $\dif t = -\mu b \dif \theta/p_{0}/\sin^{2}\theta$, we have
\[ p_{\bot} = -\int_{0}^{\pi} \sin\theta\,\dif\theta\; \frac{\mu}{p_{0}}\frac{q_{1}q_{2}}{b}, \]
leading to the intuitive result
\begin{equation}\label{e.pperp}
\frac{p_{0}p_{\bot}}{2\mu} = \frac{q_{1}q_{2}}{b}.
\end{equation}
Clearly a large angle scattering occurs if $p_{\bot}\ge p_{0}$, or
\begin{equation}\label{e.b0}
b \le b_{0} \equiv \frac{2\mu q_{1}q_{2}}{p_{0}^{2}};
\end{equation}
our approach is only valid for $b \gg b_{0}$.  Note that $p_{\bot}/p_{0} = b_{0}/b$. What is the rate of large angle  scatterings? The cross section for a large angle scattering is
$ \sigma_{\mathrm{LA}} = \pi b_{0}^{2}$.  Imagine a particle incident on a volume of particles.  imagine a cylinder of length $(p_{0}/\mu)\dif t \mathcal{A}$, where $\mathcal{A}$ is the cross-section of the cylinder.  Within this cylinder there are $n\times (p_{0}/\mu)\dif t \mathcal{A}$ scatterers of cross-section $\sigma_{\mathrm{LA}}$. so the probability of the particle interacting per time $\dif t$ is
\[ \frac{(\sigma_{\mathrm{LA}}\times n\times p_{0}/\mu) \mathcal{A}}{\mathcal{A}}  = n\sigma_{\mathrm{LA}} p_{0}/\mu. \]
This defines the large-angle collision rate,
\begin{equation}\label{e.large-angle-collision-rate}
\nu_{\mathrm{LA}} = n\sigma_{\mathrm{LA}} v_{0} = \frac{4\pi \mu (q_{1}q_{2})^{2}}{p_{0}^{3}}.
\end{equation}
Note that it goes as $p_{0}^{-3}$; fast-moving particles are hard to scatter.

\begin{figure}[htbp]
\includegraphics[width=\textwidth]{scattering}
\caption{Geometry for scattering problem.}
\label{f.scattering}
\end{figure}

As mentioned earlier, we are in a weakly coupled plasma, so we expect large angle scatterings to be a rare occurrence. What happens instead is that the particle suffers a number of small deflections. Let's consider an impact parameter in the range $(b,b+\dif b)$. Each deflection will have $\bvec{\Delta p}_{\bot}$ in some random direction, so
\[ \langle \bvec{ p}_{\bot}\rangle = \sum_{i=1}^{N} \bvec{\Delta p}_{i} \approx \bvec{0}. \]
What is happening is that the component of momentum perpendicular to $p_{0}$ is executing a random walk.  Indeed, after $N$ collisions,
\begin{eqnarray*} \langle \bvec{p}_{\bot} \vdot \bvec{p}_{\bot} \rangle  &=& \left( \sum_{i=1}^{N} \bvec{\Delta p}_{i}\right)\vdot \left( \sum_{i=1}^{N} \bvec{\Delta p}_{i} \right) \\
&=& \sum_{i=1}^{N} (\Delta p_{i})^{2} + 2\sum_{i\ne j} \bvec{\Delta p}_{i}\vdot\bvec{\Delta p}_{j} = N(\Delta p_{\bot})^{2},
\end{eqnarray*}
where we have assumed that all $\bvec{\Delta p}_{\perp}$ have the same magnitude and are uncorrelated. Now $N$ is just $\Delta t \times n \times (2\pi b \,\dif b) \times (p_{0}/\mu)$: the number of particles with impact parameters between $b$ and $b+\dif b$ along the length of the particles path over a time $\Delta t$.  Dividing by $\Delta t$ and integrating over $b$, we have the rate of change of the perpendicular component of the momentum
\begin{equation}\label{e.msdev-momentum}
\frac{\dif \langle p_{\bot}^{2}\rangle }{\dif t} = \frac{8\pi n\mu (q_{1}q_{2})^{2}}{p_{0}}\int_{b_{\mathrm{min}}}^{b_{\mathrm{max}}}\;\frac{\dif b}{b}.
\end{equation}
What are $b_{\mathrm{max}}$ and $b_{\mathrm{min}}$, the maximum and minimum impact parameters? Clearly, if $b > \lambda_{\mathrm{D}}$ then the potential will be screened.  Since our approximation is only good for $b > b_{0}$, we may take $b_{\mathrm{min}} = b_{0}$. (Our rate of scattering only depends logarithmically on $b_{\mathrm{max}}/b_{\mathrm{min}}$, so these estimates are good enough for our purposes).  With this substitution,
\begin{eqnarray}
\frac{\dif \langle p_{\bot}^{2}\rangle }{\dif t} &=& \frac{8\pi n\mu (q_{1}q_{2})^{2}}{p_{0}}\ln\left(\frac{\lambda_{\mathrm{D}}}{b_{0}}\right)\nonumber \\
 &\equiv& \frac{8\pi n\mu (q_{1}q_{2})^{2}}{p_{0}}\ln\Lambda.
\label{e.deflection-rate}
\end{eqnarray}
In the literature, the quantity $\ln\Lambda$ is called the Coulomb logarithm; for a plasma such as we are considering it is $\sim \ln  \left(\Lambda_{\mathrm{D}}^{3} n\right)$, the logarithm of the number of particles in a Debye sphere (see eq.~[\ref{e.plasma-parameter}]).  For conditions typical of the solar center (hydrogen plasma, $\rho \gtrsim 1\nsp\grampercc$, $T \approx 10^{7}\nsp\K$), $\ln\Lambda \approx \left(5\textrm{--}10\right)$.  

For small-angle scattering, the concept of a collision rate is fuzzy: the particle is constantly being bombarded by many tiny collisions.  Setting $\dif \langle p_{\bot}^{2}\rangle/\dif t = p_{0}^{2}$ allows us to define a deflection rate,
\begin{eqnarray}\label{e.scattering-rate}
\nu &\approx& \frac{8\pi n \mu (q_{1}q_{2})^{2}}{p_{0}^{3}} \ln\Lambda\\
 &=& \frac{8\pi n (q_{1}q_{2})^{2}}{(3\kB T)^{3/2}\mu^{1/2}}\ln\Lambda.
\end{eqnarray}
Comparing equations~(\ref{e.scattering-rate}) and (\ref{e.large-angle-collision-rate}), we see that many small angle scatterings are more important than single large angle scattering. Note that the ion-ion collision rate will be about $\sqrt{m_{p}/m_{e}} \approx 43$ times less than the electron-electron collision rate for a given temperature.

One can define a mean free path $\ell$ from equation~(\ref{e.scattering-rate}). Consider a particle incident on a cylinder of cross-sectional area $\mathcal{A}$ and length $\ell$, as illustrated in Figure~\ref{f.mfp}. We chose $\ell$ so that the time for the particle to traverse it is $\nu^{-1}$, the timescale for deflection.  Thus $\ell = v_{0}/\nu = p_{0}/(\mu\nu)$. Note that for a large angle collision, we can write the probability for scattering as the total cross-section of scatterers per unit area,
\begin{equation}\label{e.mfp-simple}
\mathcal{P} = \frac{N\sigma}{\mathcal{A}} = \frac{n\times(\ell\mathcal{A})\sigma}{\mathcal{A}},
\end{equation}
so the particle will suffer on average a collision after traversing a distance $\ell = (n\sigma)^{-1}$. Comparing equation~(\ref{e.mfp-simple}) with our expression for $\ell$ in terms of $\nu$ allows us to define an effective cross-section for small angle scattering.

\begin{figure}[htbp]
\includegraphics[width=\textwidth]{mean-free-path}
\caption{Schematic of a particle incident on a cylinder containing $n\times\ell\times\mathcal{A}$ particles.}
\label{f.mfp}
\end{figure}

\section{Transport properties}

We now have enough machinery to make estimates of \emph{transport coefficients,} such as the viscosity and the thermal conductivity. Let's begin with the viscosity.  Suppose we have a fluid with a gradient in the velocity, a shear, as depicted in Figure~\ref{f.shear-diagram}.  Let the mean thermal velocity of a particle be $v_{0}$.  In a time $\Delta t$, a number of particles will enter the box from the top, $(1/6) n v_{0} \Delta t$, and a similar number will leave the box via the top face. On average, these particles are endowed with the fluid properties of their last scattering, so the \emph{net} momentum carried into the box across the top face is
\begin{equation}\label{e.viscosity-1}
 \frac{1}{6} n m v_{0} \Delta t \Delta A \left[v_{y}(z_{t} + \ell) - v_{y}(z_{t}-\ell)\right] \approx \frac{1}{3} n m v_{0} \Delta t\Delta A \left.\frac{\partial v_{y}}{\partial z}\right|_{z_{t}}\ell.
\end{equation}
Here $\Delta A$ is the cross-section area of our box in the $xy$ plane and $z_{t}$ is the coordinate of the top face.

\begin{figure}[htbp]
\includegraphics[width=\textwidth]{shear-diagram}
\caption{An element of fluid with a shear $\partial v_{y}/\partial z$.}\label{f.shear-diagram}
\end{figure}

A similar process occurs across the bottom face, located at coordinate $z=z_{b}$: the momentum flux across the bottom face is
\begin{equation}\label{e.viscosity-2} 
\approx -\frac{1}{3} n m v_{0} \Delta t\Delta A \left.\frac{\partial v_{y}}{\partial z}\right|_{z_{b}}\ell.
\end{equation}
Note the difference in sign: the momentum flux is positive if the $y$-velocity is larger below the box.  Putting equations~(\ref{e.viscosity-1}) and (\ref{e.viscosity-2}) together, the net change of momentum per time per unit volume $\Delta A\Delta z$ is
\begin{eqnarray}
\frac{1}{\Delta A\Delta z}\frac{\Delta p}{\Delta t} &\approx&  \frac{1}{\Delta z} \frac{1}{3} \left[\left(n m v_{0}\ell\frac{\partial v_{y}}{\partial z}\right)_{z_{t}} - \left(n m v_{0} \ell\frac{\partial v_{y}}{\partial z}\right)_{z_{b}}\right]\nonumber\\
 &\approx&   \frac{\partial}{\partial z}\left( \mu \frac{\partial v_{y}}{\partial z}\right).
\label{e.viscosity-3}
\end{eqnarray}
Here we have defined $\mu = nmv_{0}\ell/3 $ as the coefficient of dynamic viscosity. 

Dividing equation~(\ref{e.viscosity-3}) by the mass density modifies Euler's equation, eq.~(\ref{e.euler}), to the \emph{Navier-Stokes equation},
\begin{equation}\label{e.navier-stokes}
\partial_{t}\vu + \vu\cdot\grad\vu = -\grad \Phi - \frac{1}{\rho}\grad P + \frac{1}{\rho}\divr(\mu \grad\vu).
\end{equation}
In an isothermal, incompressible fluid, one can pull $\mu$ outside the divergence operator and the last term becomes
\[ \frac{\mu}{\rho}\nabla^{2}\vu \equiv \nu \nabla^{2}\vu, \]
where $\nu$ is defined as the \emph{coefficient of kinematic viscosity} (sorry for the overload of notation with the scattering frequency earlier!). Note that in order-of-magnitude
\[ \nu \sim \frac{1}{3}v_{0}\ell,\]
that is, it is roughly the thermal velocity times the mean free path.

An identical proceedure, but replacing the average momentum of a particle with its average thermal energy, yields an expression for the \emph{thermal conductivity} $K$, such that the heat flux is
\begin{equation}\label{e.flux-equation}
\bvec{F} = -K\grad T.
\end{equation}
If one writes the change in energy of a fluid element as being 
\[\rho C \partial_{t}T = \divr(K\grad T),\]
 it can be seen that in order of magnitude the thermal diffusivity $\chi \equiv K/(\rho C)$, where $C$ is the specific heat per unit mass, is $\chi \sim (1/3) v_{0} \ell$.  (From the form of the equation and dimensional analysis, it has to be like this.)  But we need to be careful here: in a plasma with ions and electrons, the ions are responsible for momentum transport, whereas electrons, being more nimble, are more effective at heat transport.  Thus the thermal diffusivity is larger than the kinematic viscosity by a factor $\sim \sqrt{m_{p}/m_{e}}\approx 43$.

\section{Exercises}\label{s.plasma-exercises}
\begin{enumerate}
\item Show that equation~(\ref{e.plasma-parameter}) is equivalent to $\Gamma \ll 1$ for a single species plasma.

\item Show that for a non-relativistic plasma, the magnetic interaction between two charged particles is much less than the electrostatic interaction.

\item Show that the net charge in the shielding cloud about an ion of charge $Ze$ is $-Ze$; the shielding cloud cancels out the ion's charge.

\item 
\begin{enumerate} 
\item\label{p.zero-pressure-iron} In the zero-temperature limit (electrons are fully degenerate) use the charge-neutral sphere approximation (p.~\pageref{e.madelung-total}) to calculate the density at which completely ionized \iron[56] has zero pressure.

\item Estimate the pressure that would be required to compress \iron[56] at the density found in part~\ref{p.zero-pressure-iron}.
\end{enumerate}

\item Now write the total pressure as the sum of electron and Coulomb pressure, and use the virial scalings that we derived for pressure and density in terms of mass and radius, to obtain a relation between mass and radius for a cold object (``a rock'').  Find the mass having the largest radius, and express this mass in terms of fundamental physical constants.  How does it compare with the mass of Jupiter?  Scale the mass and radii to that of Jupiter, and plot $R(M)$ for pure \hydrogen, pure \helium, and pure \carbon\ objects.  Also indicate on this plot the masses and radii of the Jovian planets for comparison.

\item Estimate the ratio of the pressure to viscous accelerations,
\[ \frac{|\rho^{-1}\grad P|}{|\nu\nabla^{2}\vu|} ,\]
in equation~(\ref{e.navier-stokes}).  Express your answer in terms of a characteristic lengthscale, Mach number, and mean free path.  Under what conditions are viscous effects important?

\item Estimate the plasma thermal conductivity under conditions appropriate to the solar center.

\end{enumerate}

% !TEX root = ./notes.tex
\chapter{Convection}\label{s.convection}

Hot air rises, as a glider pilot or hawk can tell you. The fluid velocities in question are very subsonic, so we still have hydrostatic equilibrium to excellent approximation. You can perform the following experiment to demonstrate this phenomenon, \emph{convection}.  Brew tea, and pour the hot tea into a saucepan that is on an unlit burner.  Using a straw with your thumb over the top to place a layer of cold milk under the warm tea in the saucepan.  The temperature difference should help keep the tea and milk from mixing.   Light the burner, and watch for the development of convection---you will know it when you see it.

\section{Criteria for onset of convection}\label{s.convection-onset}

To understand this process, let's image that we have a fluid in planar geometry and hydrostatic equilibrium,
\begin{equation}
\frac{\dif P}{\dif z} = -\rho g.
\end{equation}
Now, imagine moving a blob of fluid upwards from $z$ to $z+h$.  We move the blob slowly enough that it is in hydrostatic equilibrium with its new surroundings, $P_{b}(z+h) = P(z+h)$, where the subscript $b$ refers to ``blob.'' We do, however, move the blob quickly enough that it is not in \emph{thermal} equilibrium with its surroundings; that is we move the blob adiabatically.  The entropy of the blob at $z+h$ is the same as at $z$ and is therefore not, in general, equal to the entropy of the surrounding gas at $z+h$: $S_{b}(z+h) = S_{b}(z) = S(z) \neq S(z+h)$.  Figure~\ref{f.convective-schematic} has a cartoon of this process.

\begin{figure}[htbp]
\centerline{\includegraphics[width=\textwidth]{convective}}
\caption{\label{f.convective-schematic}Illustration of criteria for convective instability.  On the left, raising a blob a distance $h$ adiabatically and in pressure balance with its surrounding results in a higher density $V_{b} < V$.  This is stable: the blob will sink back.  On the right, the blob is less dense and hence buoyant: it will continue to rise.}
\end{figure}

In raising the blob, we had to displace some of the surrounding fluid. Archimedes tells us that if the displaced fluid is less massive than the blob, than the blob will sink.  Equivalently, if the volume of an equivalent mass of background fluid is greater than that of the blob, the blob will sink,
\begin{eqnarray}
\lefteqn{V[P(z+h),S(z+h)] - V_{b}[P(z+h),S(z+h)] =}\nonumber\\
&&  V[P(z+h),S(z+h)] - V[P(z+h),S(z)] > 0
\label{e.achimedes}
\end{eqnarray}
If this condition is violated, the blob continues to rise, and the system is unstable to convection.  Expanding the left-hand side of equation~(\ref{e.achimedes}) gives
\[
V[P(z+h),S(z)] + \tderiv{V}{S}{P}\frac{\dif S}{\dif z} - V[P(z+h),S(z)] > 0 
\]
so our condition for stability is just
\begin{equation}\label{e.convective-stability}
\left(\frac{\partial V}{\partial S}\right)_{P}\frac{\dif S}{\dif z} > 0.
\end{equation}
Noting that
\begin{eqnarray*}
\tderiv{V}{T}{P} &=& \tderiv{V}{S}{P}\tderiv{S}{T}{P}\\
 &=& \frac{C_{P}}{T}\tderiv{V}{S}{P},
 \end{eqnarray*}
 we can rewrite equation~(\ref{e.convective-stability}) as
 \[
 \frac{T}{C_{P}}\tderiv{V}{T}{P}\frac{\dif S}{\dif z} > 0.
 \]
 Now, $(\partial V/\partial T)_{P}$ is positive (gas expands on being heated), so our condition for stability is simply
 \begin{equation}\label{e.entropy-condition}
\frac{\dif S}{\dif z} > 0.
\end{equation}
In a convectively stable star, the entropy must increase with radius. Convection mixes high-entropy material outward, where it will eventually mix.  As a result, convection drives the entropy gradient toward the marginally stable configuration $\dif S/\dif r = 0$.  If a star is fully convective and mixes efficiently, the interior of the star lies along an adiabat. 

We can derive a condition for convective stability in terms of the local gradients of temperature and pressure. Writing $S = S[P(r),T(r)]$ we expand equation~(\ref{e.entropy-condition}) and use hydrostatic equilibrium to obtain
\begin{equation}\label{e.schwarzschild-1}
\frac{\dif S}{\dif r} = \tderiv{S}{P}{T} \frac{\dif P}{\dif r} + \tderiv{S}{T}{P}\frac{\dif T}{\dif r} .
\end{equation}
Now, $P$ is a monotonically decreasing function of $r$, which means we can use it as a spatial coordinate and write,
\begin{equation}\label{e.TPstar}
\frac{\dif T}{\dif r} = \TPstar \frac{\dif P}{\dif r} .
\end{equation}
Here $\dif T/\dif P|_{\star}$ is the slope of the $T(P)$ relation \emph{for the stellar interior}.  In particular, this is not a thermodynamic equality. Substituting equation~(\ref{e.TPstar}) into equation~(\ref{e.schwarzschild-1}), using hydrostatic equilibrium to eliminate $\dif P/\dif r$, and recognizing that $(\partial S/\partial T)_{P} = C_{P}/T$, we obtain
\begin{equation}\label{e.schwarzschild-2}
\frac{\dif S}{\dif r} =  -\rho g\left[\tderiv{S}{P}{T} + \frac{C_{P}}{T} \TPstar \right].
\end{equation}
Finally, we can use the identity (see Appendix~\ref{s.thermo-exercises})
\begin{equation}
\tderiv{S}{P}{T}\tderiv{T}{S}{P}\tderiv{P}{T}{S} = -1
\end{equation}
to simplify equation~(\ref{e.schwarzschild-2}),
\begin{eqnarray}
\frac{\dif S}{\dif r} &=& -\frac{\rho g}{P}C_{P}\left[\frac{P}{T}\TPstar - \frac{P}{T}\tderiv{T}{P}{S}\right]\nonumber \\
 & = & -\frac{\rho g}{P}C_{P}\left[\nabla - \nabla_{\mathrm{ad}}\right].
 \label{e.schwarzschild}
\end{eqnarray}
Here we have introduced the shorthand $\nabla\equiv \dif \ln T/\dif\ln P|_{\star}$, $\nabla_{\mathrm{ad}} \equiv \left(\partial T/\partial P\right)_{S}$.
 
\section{Efficiency of Heat Transport}

In the previous section, we found that a superadiabatic temperature gradient induces convective motions. A rising blob will be hotter than its surroundings.  As it rises, heat is conducted from blob to surroundings. As a result, convection transports heat upwards and reduces the thermal gradient. The question is by how much.  Clearly the gradient must be super-adiabatic to drive the convection in the first place. We shall see, however, that usually in stars the difference between the gradient and the adiabat are exceedingly small. In other words, convection is extraordinarily efficient at transporting heat.

To understand this, let's go back to our equations.  Let's write our density as $\rho + \Delta\rho$, where $\rho$ is found from hydrostatic balance and $\Delta\rho$ is a perturbation stemming from differences in temperature between rising and falling blobs.   We can insert this into the Navier-Stokes equation; furthermore, we will both $\Delta\rho$ and $\vu$ to be perturbations, so we will drop terms like $\vu\Delta \rho$, and arrive at a perturbed Navier-Stokes equation,
\begin{equation}\label{e.perturbed-navier-stokes}
 (\partial_{t}\vu + \vu\vdot\grad\vu) = \frac{\Delta\rho}{\rho}\vg = \left(\frac{\partial\ln\rho}{\partial\ln T}\right)_{P}\frac{\Delta T}{T}\vg.
 \end{equation}
%Heat conduction follows the equation
%\begin{equation}\label{e.heat-conduction}
%\rho C_{p}(\partial_{t} + \vu\cdot\grad)T = -\divr \bvec{F} = K\nabla^{2} T
%\end{equation}
%where $K$ is the effective thermal conductivity (includes contributions from both radiative and electronic transport of heat).
Our goal is to estimate the velocity of convective motions, the departure of the temperature gradient from an adiabat, and the fraction of the total heat flux carried by convective motions from these equations.

First, the velocity.  The left-hand side of equation~(\ref{e.perturbed-navier-stokes}) has a characteristic scale $]sim U^{2}/L$, whereas the right-hand side has a scale $\Delta T/T g$. (Recall that in an ideal gas, $(\partial\ln \rho/\partial\ln T) = -1$.)  If we take $L \sim c_{s}^{2}/g$, a pressure scale height, than we get an estimate of the convective velocity,
\begin{equation}\label{e.convective-velocity-estimate}
\frac{U}{c_{s}} \sim \left(\frac{\Delta T}{T}\right)^{1/2}.
\end{equation}
What is the heat flux carried by convection? Hot fluid rises and carries an excess of heat, per gram, of $c_{P}\Delta T$, giving a heat flux $\approx \rho \vu c_{P}\Delta T$. Thus for a given flux $F$, we have
\[
c_{s}\rho c_{P} T\left(\frac{\Delta T}{T}\right)^{3/2} \sim F.
\]
Note that in order of magnitude, $c_{P}T \sim c_{s}^{2}$, so 
\[
\frac{\Delta T}{T} \sim \left(\frac{F}{\rho c_{s}^{3}}\right)^{2/3}.
\]
For conditions in the solar interior, $F \ll \rho c_{s}^{3}$, and therefore the convective velocities are very subsonic. 

\section{Turbulence}
 
\section{Exercises}
\begin{enumerate}
\item Assuming that $\nabla \approx \nabla_{\mathrm{ad}}$ in a convective region, sketch a plot of temperature as a function of pressure for
\begin{enumerate}
\item A star with a stable inner layer and a convective outer layer;
\item A star with a convective inner layer and and a stable outer layer.
\end{enumerate}
Indicate on these plots an adiabat.
\end{enumerate}
% !TEX root = ./notes.tex
\chapter{Polytropes and the Lane-Emden Equation}

\section{Background}\label{s.LE-background}

A classic problem in stellar evolution is the construction of \emph{polytropic} stellar models.  To understand where the term polytropic comes from, let's first consider an ideal gas in hydrostatic equilibrium.  If the gas is in convective equilibrium then we know it lies along an adiabat.  In that case we have the following relations:
\begin{equation}\label{e.ideal-adiabatic-relations} 
T\rho^{1-\gamma} = \mathrm{const};\quad P\rho^{-\gamma} = \mathrm{const};\quad TP^{(1-\gamma)/\gamma} = \mathrm{const}.
\end{equation}
Here $\gamma = C_{P}/C_{\rho}$ is the ratio of specific heats. Furthermore, we have the equation of state,
\begin{equation}\label{e.ideal-eos}
P = \left(\frac{\NA\kB}{\mu}\right)\rho T,
\end{equation}
where $\mu$ is the mean molecular weight and the quantity in parenthesis is $C_{P}-C_{\rho} = \NA\kB/\mu$.

For the Earth's atmosphere, the condensation of water vapor means that one cannot hold $\dif s = 0$ in a rising plume.  This motivated work in the early 1900's, by Kelvin, Lane, Emden, and others, to consider a more general problem, in which $T\dif s = C\dif T$, where $C$ is a constant.  A configuration for which this is true is called a \emph{polytropic configuration}. An adiabat is a special case with $C = 0$. Writing the first law of thermodynamics as $T\dif s = C_{\rho}\dif T -(P/\rho^{2})\dif \rho$, substituting for $T\dif s$, and using equation~(\ref{e.ideal-eos}), we obtain
\[
\left(C_{\rho}-C\right)\frac{\dif T}{T} = \left( C_{P} - C_{\rho}\right)\frac{\dif\rho}{\rho}.
\]
This equation has a solution $T  \propto  \rho^{(C_{P}-C_{\rho})/(C_{\rho}-C)}$. Comparing this solution with equation~(\ref{e.ideal-adiabatic-relations}), we can define a \emph{polytropic exponent}, $\gamma' = (C_{P}-C)/(C_{\rho}-C)$. Then equation~(\ref{e.ideal-adiabatic-relations}) holds with $\gamma$ replaced by $\gamma'$.  The advantage of this approximation is that it relates density to pressure so that one can solve the equation of hydrostatic equilibrium without simultaneously having to solve for $T(r)$.

\section{The Lane-Emden Equation and Solution}\label{s.LE-solution}

To use the polytropic equation of state, write the pressure $P$ as
\begin{equation}\label{e.polytope}
P(r) = K\rho^{1+1/n}(r)
\end{equation}
where $n$ and $K$ are constants. Further define the dimensionless variable $\theta$ via
\begin{equation}\label{e.theta-def}
\rho(r) = \rho_{c}\theta^{n}(r),
\end{equation}
where the subscript $c$ denotes the central value at $r=0$. Note that since
\[ P(r) \propto \rho \times \rho^{1/n} \propto \rho \theta, \]
the quantity $\theta$ plays the role of a dimensionless temperature for an ideal non-degenerate gas.

Substitute equations~(\ref{e.polytope}) and (\ref{e.theta-def}) into Poisson's equation,
\begin{equation}
\nabla^{2}\Phi = 4\pi G\rho,
\end{equation}
and the equation for hydrostatic equilibrium, 
\begin{equation}
\grad P = -\rho\grad \Phi,
\end{equation}
to obtain the \emph{Lane-Emden} equation for index $n$,
\begin{equation}\label{e.LE}
\xi^{-2} \frac{d}{d\xi}\left(\xi^{2}\frac{d\theta}{d\xi}\right) = -\theta^{n}.
\end{equation}
Here $\xi = r/r_{n}$ is the dimensionless coordinate, and
\begin{equation}\label{e.LE-length-scale}
r_{n} = \left[\frac{(n+1)P_{c}}{4\pi G\rho_{c}^{2}}\right]^{1/2}
\end{equation}
is the radial length scale.

For a stellar model described by a single polytropic relation, the appropriate boundary conditions are
\begin{eqnarray}
\label{e.thetabc}\left.\theta(\xi)\right|_{\xi = 0} &=& 1,\\
\label{e.thetapbc}\left.\theta'(\xi)\right|_{\xi=0} &=& 0.
\end{eqnarray}
From the form of equation~(\ref{e.LE}), it follows that $\theta(-\xi) = \theta(\xi)$, that is, the solution is \emph{even} in $\xi$. A power-series solution of $\theta$ out to order $\xi^{6}$ is
\begin{equation}\label{e.series}
\theta(\xi) = 1 - \frac{1}{6}\xi^{2} + \frac{n}{120}\xi^{4} - \frac{n(8n-5)}{15120}\xi^{6} + \mathcal{O}(\xi^{8})
\end{equation}
There are analytical solutions for $n = 0$, 1, and 5:
\begin{eqnarray}
\theta_{0}(\xi) &=& 1-\frac{\xi^{2}}{6}\label{e.solution-0}\\
\theta_{1}(\xi) &=& \frac{\sin\xi}{\xi}\label{e.solution-1}\\
\theta_{5}(\xi) &=& \left(\frac{3}{3 + \xi^{2}}\right)^{1/2}.\label{e.solution-5}
\end{eqnarray}
Finally, the radius of the stellar model is determined by the location of the first zero, $\xi_{1}$.  For example, if $n = 0$ (eq.~[\ref{e.solution-0}]), $\xi_{1} = \sqrt{6}$. Note that if $n=5$ there is no root; the configuration extends to $\xi \to \infty$.  A sample of Lane-Emden solutions is shown in Figure~\ref{f.LE-solutions}.

\begin{figure}[htbp]
\centering{\includegraphics[width=4in]{LE_all.pdf}}
\caption{Solutions of the Lane-Emden equation for selected valued of the index $n$.}
\label{f.LE-solutions}
\end{figure}

\section{Some Useful Relations}

First, let's get the mass of our polytropic sphere.  To do this, we write the integral
\[ M = \int_{0}^{R} 4\pi r^{2} \rho\,\dif r \]
and make the substitutions $r = r_{n}\xi$, $R = r_{n}\xi_{1}$, and $\rho = \rho_{c}\theta^{n}(\xi)$ to obtain
\[
M = 4\pi r_{n}^{3}\rho_{c}\int_{0}^{\xi_{1}} \xi^{2}\theta^{n}(\xi)\,\dif\xi.
\]
Using equation~(\ref{e.LE}) , the integrand can be written as a perfect differential, so we get
\begin{equation}\label{e.LE-mass}
M = 4\pi r_{n}^{3} \rho_{c}\left(-\xi_{1}^{2} \theta_{1}'\right).
\end{equation}
Here I define the shorthand $\theta_{1}' \equiv \theta(\xi)|_{\xi=\xi_{1}}$.  Substituting $r_{n} = R/\xi_{1}$ and dividing by 3 allows us to get a formula relating the central density to the mean density,
\begin{equation}\label{e.LE-density-ratio}
\rho_{c} = \frac{3M}{4\pi R^{3}}\left(-\frac{\xi_{1}}{3\theta_{1}'}\right).
\end{equation}
For the solutions shown in Figure~\ref{f.LE-solutions}, we have the following values of $\rho_{c}/\bar{\rho}$, as shown in Table~\ref{t.LE-properties}.
As the index $n$ increases, the configuration becomes more and more concentrated toward the center.

\begin{table}[htpb]
\caption{Properties of the Lane-Emden solutions.\label{t.LE-properties}}
\begin{center}
\begin{tabular}{c|rrrrrrr}
\hline
$n$ & 0 & 1.0 & 1.5 & 2.0 & 3.0 & 4.0 & 4.5 \\
\hline\hline
$\xi_{1}$ & 2.449 & 3.142 & 3.654 & 4.353 & 6.897 & 14.972 & 31.836\\
$-\theta_{1}'$ & 0.8165 & 0.3183 & 0.2033 & 0.1272 & 0.04243 & 0.008018 & 0.001715\\
$\rho_{c}/\bar{\rho}$ & 1.00 & 3.29 & 5.99 & 11.41 & 54.18 & 622.4 & 6187.8\\
\hline
\end{tabular}
\end{center}
\end{table}

Starting from equation~(\ref{e.LE-mass}), we can substitute for $r_{n}$ using equation~(\ref{e.LE-length-scale}) and $\rho_{c}$ using equation~(\ref{e.LE-density-ratio}) to get an equation for the central pressure,
\begin{equation}\label{e.LE-PC}
P_{c} = \frac{GM^{2}}{R^{4}} \frac{1}{4\pi(n+1)(-\theta_{1}')^{2}}.
\end{equation}
For an ideal gas, $P_{c} = (\NA\kB/\mu)\rho_{c}T_{c}$ with $\mu$ being the mean molecular weight, we can solve for the central temperature,
\begin{equation}\label{e.LE-TC}
T_{c} = \left(\frac{\mu}{\NA\kB}\right)\left(\frac{GM}{R}\right)\frac{1}{(n+1)\xi_{1}(-\theta_{1}')}.
\end{equation}
Finally, starting from equation~(\ref{e.LE-PC}), substituting for $P_{c}$ using equation~(\ref{e.polytope}), and eliminating $\rho_{c}$ using equation~(\ref{e.LE-density-ratio}), we obtain a relation between mass and radius in terms of $K$ and $n$,
\begin{equation}\label{e.LE-mass-radius}
M^{1-1/n} = \left[\frac{K(n+1)}{G(4\pi)^{1/n}} \xi_{1}^{1+1/n}\left(-\theta_{1}'\right)^{1-1/n} \right] R^{1-3/n}.
\end{equation}
Alternatively, one could use this equation to fit $K$ to a star of known $M$ and $R$.

Finally, we can derive a formula for the gravitational energy of our polytropic sphere.  First, we can integrate the equation for the energy,
\[
E_{\mathrm{grav}} = -G\int_{0}^{M}\frac{m}{r}\,\dif m,
\]
by parts to obtain
\begin{equation}\label{e.polytrope-energy-1}
E_{\mathrm{grav}} = -\frac{GM^{2}}{2R} - \frac{1}{2}\int_{0}^{R}\frac{Gm^{2}}{r^{2}}\,\dif r = -\frac{GM^{2}}{2R} - \frac{1}{2}\int_{0}^{R} \frac{\dif\Phi}{\dif r} m\,\dif r.
\end{equation}
If we define our zero of energy to be such that $\Phi(R) = 0$, then we can integrate by parts again to obtain
\begin{equation}\label{e.polytrope-energy-2}
E_{\mathrm{grav}} = -\frac{GM^{2}}{2R} + \frac{1}{2}\int_{0}^{R} \Phi\,\dif m.
\end{equation}
We can rewrite the equation of hydrostatic equilibrium as
\[ \frac{\dif P}{\dif r} = -\frac{\dif\Phi}{\dif r} \rho \]
and use equation~(\ref{e.polytope}) to eliminate $P$ to obtain
\[ \frac{\dif\Phi}{\dif r} = (1+n) K \rho^{1/n} .\]
Integrating from a point in the star $r$ to $R$, and again using choosing $\Phi(R) = 0$, we obtain
\begin{equation}\label{e.polytrope-energy-3}
\Phi(r) = -(1+n)K\rho(r)^{1/n} = -(1+n)\frac{P(r)}{\rho(r)}.
\end{equation}
Inserting equation~(\ref{e.polytrope-energy-3}) into equation~(\ref{e.polytrope-energy-2}), we have
\begin{equation}\label{e.polytrope-energy-4}
E_{\mathrm{grav}} = -\frac{GM^{2}}{2R} - \frac{1+n}{2}\int_{0}^{M} \frac{P}{\rho}\,\dif m = -\frac{GM}{2R} + \frac{1+n}{6}E_{\mathrm{grav}},
\end{equation}
where we used equation~(\ref{e.virial-1}) to relate the integral of $P/\rho$ to $E_{\mathrm{grav}}$. Solving equation~(\ref{e.polytrope-energy-4}) for $E_{\mathrm{grav}}$ gives us the desired result,
\begin{equation}\label{e.polytrope-energy}
E_{\mathrm{grav}} = -\frac{3}{5-n} \frac{GM^{2}}{R}.
\end{equation}
Note that solutions with $n > 5$ have a positive gravitational energy.

\section{Eddington Standard Model}\label{s.LE-Eddington-Standard-Model}

Polytropes with index $n=3/2$ correspond to fully convective stars ($P \propto \rho^{5/3}$, the relation for an adiabat) or for white dwarfs (non-relativistic, degenerate equation of state). Another interesting case, for historical reasons, is the \emph{Eddington Standard Model}, which is a fair approximation to main-sequence stars with $M \gtrsim M_{\sun}$.  Suppose we write the equation of state as the sum of ideal gas and radiation pressure,
\begin{equation}\label{e.eos-with-rad}
 P  = \frac{\rho\kB T}{\mu \mb} + \frac{1}{3}a T^{4}.
\end{equation}
Now make the \emph{ansatz} that
\begin{equation}\label{e.beta-def}
\frac{P_{\mathrm{rad}}}{P} = \frac{aT^{4}}{3P} = 1-\beta = \mathrm{const.},
\end{equation}
that is, the radiation pressure is a fixed fraction of the total pressure everywhere.
Solving for $T$ in terms of $P$ and $\beta$,
\[ T = \left[\frac{3(1-\beta) P}{a}\right]^{1/4}, \]
and inserting this into equation~(\ref{e.eos-with-rad}) gives us a simple EOS,
\begin{equation}\label{e.beta-eos}
P = \left[\left(\frac{\kB}{\mu\mb}\right)^{4}\frac{3}{a}\right]^{1/3}\left[\frac{1-\beta}{\beta^{4}}\right]^{1/3} \rho^{4/3}.
\end{equation}
This is the equation for a polytrope of index 3.

Why is it at all reasonable to take $\beta$ as being constant? To explore this, go back to the equation for radiative diffusion
\[ F(r) = -\frac{1}{3}\frac{c}{\rho\kappa}\frac{\dif aT^{4}}{\dif r}. \]
Write the flux as $F(r) = L(r)/(4\pi r^{2})$, and since pressure decreases monotonically with radius, write
\[ 
\frac{\dif aT^{4}}{\dif r} = \frac{\dif aT^{4}}{\dif P}\frac{\dif P}{\dif r} = -\rho\frac{Gm(r)}{r^{2}}\frac{\dif aT^{4}}{\dif P}. 
\]
The equation of radiation transport then becomes
\[ L(r) = \frac{4\pi Gm(r) c}{\kappa(r)} \frac{\dif P_{\mathrm{rad}}}{\dif P}. \]
Dividing both sides by $L\cdot M/\kappa_{\mathrm{Th}}$ and rearranging terms,
\begin{equation}\label{e.Prad-P}
 \frac{\dif P_{\mathrm{rad}}}{\dif P} = \left[\frac{L\kappa_{\mathrm{Th}}}{4\pi GMc}\right]\left(\frac{\kappa(r)}{\kappa_{\mathrm{Th}}}\frac{L(r)}{L}\frac{M}{m(r)}\right).
\end{equation}
Here $L$ is the total luminosity of the star and $M$ is the total mass.  The term in $[\,]$ is a constant (the Thomson opacity $\kappa_{\mathrm{Th}}$ doesn't depend on density or temperature) and we define the \emph{Eddington luminosity} as $L_{\mathrm{Edd}}=4\pi GM c/\kappa_{\mathrm{Th}}$.  For the sun, $L_{\mathrm{Edd}} = 1.5\ee{38}\nsp\ergs\usp\second^{-1} = 3.8\ee{4}\nsp L_{\sun}$.  For the term $(\,)$ on the right-hand side, note the $L(r)/m(r)$ is basically the average energy generation rate interior to a radius $r$.  Since nuclear reactions are temperature sensitive, the heating is concentrated toward the stellar center and $L(r)/m(r)$ decreases with radius. For stars like the sun, free-free opacity is dominant, and since the free-free Rosseland opacity goes as $T^{-3.5}$, $\kappa(r)$ increases with radius.  Thus, if the energy generation rate is not too temperature dependent (the reaction $\pt + \pt \to \hydrogen[2]$ goes roughly as $T^{4.5}$ at $T=10^{7}\nsp\K$), then the term in $(\,)$  does not vary strongly with radius, and $\dif P_{\mathrm{rad}}/P$ is indeed roughly constant. 

\section{Exercises}\label{s.LE-exercises}
\begin{enumerate}
\item Derive equations~(\ref{e.LE-mass})--(\ref{e.LE-mass-radius}).

\item Explain what the mass-radius relation, eq.~(\ref{e.LE-mass-radius}), means for the cases $n=1$ and $n=3$.

\item For a fully convective star, how is the polytropic constant $K$ related to the entropy? Derive a formula for $R$ in terms of $M$ and $s$ in this case.  What happens to the star if heat is added to it?

\item Derive an expression for $\beta$ in terms of the mass of the star for the Eddington Standard Model.

\item Consider a polytrope of index $3/2$.
\begin{enumerate}
\item Using the expression for the entropy of an ideal gas (eq.~[\ref{e.sacker-tetrode}]), show that the entropy is indeed constant throughout the star.
\item Use the Lane-Emden solution to compute the specific entropy, per unit mass, in terms of the central temperature $T$ and the stellar mass $M$: $s = s(T_{c},M)$.
\item Now compute the ``gravothermal'' specific heat
\begin{equation}\label{e.cstar}
c_{\star} = T_{c}\frac{\partial s(T_{c},M)}{\partial T_{c}},
\end{equation}
and comment on its physical significance.
\end{enumerate}

\item We will see later that low-mass pre-main-sequence stars (including brown dwarfs) are fully convective and have nearly constant effective temperatures.  Use these facts to model their contraction.  Assume $T_{\mathrm{eff}} = \mathrm{const.}$ and write the equation for energy balance as $L = 4\pi R^{2} \sigma_{\mathrm{SB}} T_{\mathrm{eff}}^{4} = -\dif E/\dif t$, where $E$ refers to the total energy of the pre-main-sequence star. (Why is there a minus sign?)
\begin{enumerate}
\item Starting from this equation, derive an equation for $R(t)$. What is the characteristic timescale for a star to contract? Scale your answer to $T_{\mathrm{eff}} = 3000\nsp\K$ and $M = 0.1\nsp\Msun$, and use an expression for the energy appropriate for a fully convective star (polytrope of index $3/2$).
\item Compare your findings with more elaborate calculations.  You will find a review in \href{http://arxiv.org/abs/astro-ph/0006383}{``Theory of Low-Mass Stars and Substellar Objects,''} G. Chabrier and I. Baraffe, Ann.\ Rev.\ Astron.\ Astrophys.\ \textbf{38:} 337 (2000). 
\item Using appropriate expressions for the central density and temperature, calculate the time required for a star just below the hydrogen burning limit (about $0.07\nsp\Msun$) to contract to a radius such that $\eF(\rho_{c}) = \kB T_{c}$, where $\eF$ is the Fermi energy, and compare with the evolutionary tracks in Chabrier \& Baraffe.
\end{enumerate}
\end{enumerate}

% !TEX root = ./notes.tex
\chapter[Stellar Atmospheres]{Stellar Atmospheres}

In the atmosphere of the star, the optical depth approaches unity, and we can no longer treat the radiation field as being isotropic. Let's consider the time-independent problem ($\partial_{t}\to 0$) of a plane-parallel atmosphere. The \emph{optical depth} for an outward-directed ray is
\begin{equation}\label{e.optical-depth}
\tau_{\nu} = \int _{z}^{\infty}\!\rho\kappa_{\nu}\,\dif z'.
\end{equation}
Now the optical depth just the distance divided by the mean free path. Clearly, when $\tau_{\nu} < 1$, a photon has a good chance of reaching a distant observer without any further interactions with the stellar matter.  As a result, the intensity takes its final form around $\tau_{\nu} \approx 1$, and this defines the stellar \emph{photosphere}. To get some of the basic properties of the photosphere, rewrite eq.~(\ref{e.optical-depth}) in differential form,
\begin{equation}\label{e.dtaudz}
\frac{\dif\tau}{\dif z} = -\rho\kappa.
\end{equation}
This is for a crude estimate, so we neglect the frequency dependence for now.  We can use equation~(\ref{e.dtaudz}) along with hydrostatic balance to get an estimate of the photospheric pressure,
\begin{equation}\label{e.photo-pressure}
\frac{\dif P}{\dif \tau} = -\left(\frac{\dif \tau}{\dif z}\right)^{-1}\rho g = \frac{g}{\kappa}.
\end{equation}
Thus, at $\tau \approx 1$, the pressure is $P_{\mathrm{ph}}\approx g/\kappa$. Since the flux at the photosphere is $\sigma_{\mathrm{SB}} T_{\mathrm{eff}}^{4}$, we would expect that the local temperature is $T\approx T_{\mathrm{eff}}$.

\section{The Eddington Approximation}

To get an analytical approximation for the atmosphere, we'll first redefine our transfer equation in terms of optical depth (eq.~[\ref{e.transfer-with-source}]). Here however, we will take the optical depth to be along the $z$-direction, so we define $\mu = \unitk\cdot\unitn$, where \unitn\ is the direction along the ray. The equation of transfer then becomes
\begin{equation}\label{e.planar}
\mu\frac{\partial I_{\nu}}{\partial\tau_{\nu}} = I_{\nu}-S_{\nu},
\end{equation}
where 
\begin{equation}\label{e.source}
S_{\nu} \equiv \frac{1}{\kappa_{\nu}}\left(\frac{\varepsilon_{\nu}}{4\pi} + \kappa_{\nu}^{\mathrm{sca}}J_{\nu}\right)
\end{equation}
is the \emph{source function}. In local thermodynamical equilibrium (LTE), we can write $S_{\nu} = (1-A_{\nu})B_{\nu} + A_{\nu}J_{\nu}$, where $A_{\nu} \equiv \kappa_{\nu}^{\mathrm{sca}}/\kappa_{\nu}$ is the \emph{albedo}.  Recall that $J_{\nu} = (4\pi)^{-1}\int\dif\Omega\,I_{\nu}$ is the angle-average of $I_{\nu}$.

We noted that in thermal equilibrium, $P_{\nu} = c^{-1}\int_{-1}^{1}\dif\mu\,\mu^{2}I_{\nu} = u_{\nu}/3$. This relation holds even when the radiation is not thermal, so long as it is isotropic to terms linear in $\mu$.  To make this concrete, suppose we write
\[ I_{\nu}(\mu) = I_{\nu}^{(0)} + \mu I_{\nu}^{(1)} + \mu^{2}I_{\nu}^{(2)} + \ldots. \]
Here we are assuming that terms marked $(0)$ are much larger than terms marked $(1)$, etc.  To lowest order, the energy density, flux, and momentum flux are then
\begin{eqnarray*}
u_{\nu} &=& \frac{2\pi}{c}\int_{-1}^{1}\dif\mu\,I_{\nu}(\mu) = \frac{4\pi}{c} I_{\nu}^{(0)},\\
F_{\nu} &=& 2\pi\int_{-1}^{1}\dif\mu\,\mu\,I_{\nu}(\mu) = \frac{4\pi}{3} I_{\nu}^{(1)},\\
P_{\nu} &=& \frac{2\pi}{c}\int_{-1}^{1}\dif\mu\,\mu^{2}\,I_{\nu}(\mu) = \frac{4\pi}{3c}I_{\nu}^{(0)} = \frac{u_{\nu}}{3}.
\end{eqnarray*}
The \emph{Eddington approximation} then consists of treating the radiation field as if its anisotropy is linear in $\mu$ \emph{everywhere}, so that the above relations hold; in particular, it means assuming that $P_{\nu} = u_{\nu}/3$ everywhere.

\section[Grey Atmosphere]{A Grey Atmosphere}

Finally, to get an analytical approximation to the structure of the solar atmosphere, let's consider a grey atmosphere in LTE, i. e., one for which $\kappa_{\nu}^{\mathrm{abs}} = \kappa^{\mathrm{abs}}$ and $\kappa_{\nu}^{\mathrm{sca}} = \kappa^{\mathrm{sca}}$ are independent of frequency. Equation~(\ref{e.planar}) can then be integrated over all frequencies to become
\begin{equation}\label{e.J-grey}
\mu\frac{\partial I}{\partial\tau} = I-S.
\end{equation}
Integrating over all angles (note that we can pull the derivative wrt $\tau$ out of the integral) gives
\begin{equation}\label{e.H-grey}
\frac{1}{4\pi}\frac{\partial F}{\partial\tau} = J - S = 0.
\end{equation}
Why does the right-hand side vanish? Note that $S-J = (1-A)(B-J)$.  Clearly $S = J$ if $A = 1$ (a pure scattering atmosphere).  If $A \ne 1$, so that there is some absorption, then the condition of detailed balance, equation~(\ref{e.detail-balance}), implies that $\varepsilon_{\nu} = 4\pi\kappa^{\mathrm{abs}}B_{\nu}(T)$; inserting this into equation~(\ref{e.rad-equil}), factoring out the constant $\kappa^{\mathrm{abs}}$, and integrating over $\nu$ implies that $B - J = 0$, and hence $S - J = 0$. Note that $J = B$ does \emph{not} necessarily imply that $I_{\nu} = B_{\nu}$!

Now multiply equation~(\ref{e.J-grey}) by $\mu$ and integrate over $2\pi\,\dif\mu$ to obtain
\begin{equation}\label{e.K-grey}
c\frac{\partial P}{\partial\tau} = F,
\end{equation}
the integral over $\mu S$ vanishing because it is odd in $\mu$. Equation~(\ref{e.H-grey}) implies that $F$ is constant; hence we can integrate equation~(\ref{e.K-grey}) at once to obtain
\begin{equation}\label{e.KH}
cP = F(\tau + \tau_{0}),
\end{equation}
where $\tau_{0}$ is a constant of integration. Of course, this does help us yet; all we have done is introduce a new variable $P$, the radiation pressure. This is where the Eddington approximation comes in.  We set $P = u/3 = 4\pi J/(3c)$ in equation~(\ref{e.KH}) to obtain $4\pi J = 3F(\tau + \tau_{0})$. Since $J = S$, we can then write equation~(\ref{e.J-grey}) as
\begin{equation}
\mu\frac{\partial I}{\partial\tau} = I - \frac{3}{4\pi}F(\tau+\tau_{0}).
\end{equation}
Since $F$ is constant, this first-order differential equation is now solvable,
\begin{eqnarray}
I(\mu,\tau=0) &=& \frac{1}{\mu}\int_{0}^{\infty}\!\frac{3}{4\pi}F(\tau + \tau_{0}) e^{-\tau/\mu}\,\dif\tau,\nonumber\\
  &=& \frac{3}{4\pi}F(\mu + \tau_{0}).\label{e.I-Edd}
\end{eqnarray}
Now at $\tau = 0$, all of the flux must be outward-directed ($\mu >0$), so $I(\mu < 0,\tau = 0) = 0$ if the star is not irradiated by another source.  Note that the Eddington approximation is clearly violated here.  Still, we will see later that this approximation is not too terrible. 

To determine $\tau_{0}$, multiply $I(\mu,\tau = 0)$ by $\mu$ and integrate equation~(\ref{e.I-Edd}) over all angles to find
\begin{equation}
F = 2\pi\int_{0}^{1}\!\mu I(\mu,0)\,\dif\mu = \frac{1}{2}\int_{0}^{1}\!3F(\mu + \tau_{0})\,\mu\,\dif\mu = F\left(\frac{1}{2} + \frac{3}{4}\tau_{0}\right).
\end{equation}
We therefore find $\tau_{0} = 2/3$. Now, since we are in LTE, $P = aT^{4}/3$. Further, let us define an effective temperature by the relation $F = \sigma_{\mathrm{SB}}T_{\mathrm{eff}}^{4}$.  Substituting these definitions and the value of $\tau_{0}$ into equation~(\ref{e.KH}) gives us the atmospheric temperature structure,
\begin{equation}\label{e.Eddington}
T^{4}(\tau) = \frac{3}{4}T_{\mathrm{eff}}^{4}\left(\tau + \frac{2}{3}\right).
\end{equation}
Thus $T(\tau  = 0) = 2^{-1/4} T_{\mathrm{eff}}$ and $T(\tau = 2/3) = T_{\mathrm{eff}}$.  To get the spectral distribution, go back to equation~(\ref{e.planar}) and (assuming the atmosphere has some absorption so that the matter and radiation can come into equilibrium) insert $S_{\nu} = B_{\nu}(T)$; solving for $I_{\nu}$ at $\tau = 0$ then gives
\begin{equation}\label{e.spectral}
I_{\nu}(\mu,\tau=0) = \frac{1}{\mu}\int_{0}^{\infty}\!B_{\nu}\left[T(\tau)\right] \, e^{-\tau/\mu}\,\dif\tau.
\end{equation}
A plot of the spectral distribution for the emergent flux is shown (\emph{open circles}) in Fig.~\ref{f.spectral}. For comparison, a plot of the Planck distribution (\emph{solid line}) is also shown. Both fluxes are normalized to the total flux.  Note that $I_{\nu}(\mu,\tau=0)$ depends on angle; rays propagating at a slant will have a lower intensity.  As a result, when we observe the sun, the edge of the visible disk appears darker than the center, a phenomenon known as \emph{limb darkening}.

\begin{figure}[htbp]
\includegraphics[width=4in]{plots_out/spectral_distribution}
\caption{\label{f.spectral} Spectral distribution from a grey atmosphere. The open circles are from Chandrasekhar, \emph{Radiative Transfer}; the solid line is the Planck distribution.}
\end{figure}

\section{Line formation and the curve of growth}

A classical technique in the analysis of stellar spectra is to construct the \emph{curve of growth}, which relates the equivalent width of a line $W_{\nu}$ to the opacity in the line. This discussion follows Mihalas, \emph{Stellar Atmospheres}.

Let's first get the opacity in the line.  We saw in class that the cross-section for the transition $i\to j$ could be written as 
\begin{equation}\label{e.cross-section}
\sigma_{\nu} = \left(\frac{\pi e^{2}}{m_{e}c}\right)f_{ij}\phi_{\nu},
\end{equation}
where the first term is the classical oscillator cross-section, $f_{ij}$ is the oscillator strength and contains the quantum mechanical details of the interaction, and $\phi_{\nu}$ is the line profile.  Now recall that the opacity is given by $\kappa_{\nu} = n_{i}\sigma_{\nu}/\rho$, where $n_{i}$ denotes the number density of available atoms in state $i$ available to absorb a photon.  Furthermore, we need to allow for \emph{stimulated emission} from state $j$ to state $i$. With this added, the opacity is (I'm writing it as $\chi_{\nu}$ to distinguish it from the \emph{continuum opacity})
\begin{equation}\label{e.opacity}
\rho\chi_{\nu} = \left(\frac{\pi e^{2}}{m_{e}c}\right)f_{ij}\phi_{\nu}n_{i}\left[1 - \frac{g_{i}}{g_{j}}\frac{n_{j}}{n_{i}}\right].
\end{equation}
If we are in LTE, then the relative population of $n_{i}$ and $n_{j}$ follow a Boltzmann distribution,
\[ 1 - \frac{g_{i}}{g_{j}}\frac{n_{j}}{n_{i}} = 1- \exp\left(-\frac{h\nu}{kT}\right). \]
This ensures we have a positive opacity. If our population were inverted, i.~e., more atoms in the upper state $j$, then the opacity would be negative and we would have a \emph{laser}.

Now for the line profile.  In the case where we have doppler broadening and damping, the profile follows the \emph{Voigt} function,
\begin{equation}\label{e.voigt} \phi_{\nu} = \frac{1}{\Delta \nu_{D}}H(a,v), \end{equation}
where $\Delta\nu_{D} \equiv \nu u_{0}/c$ is the doppler width, with $u_{0}$ being the mean (thermal) velocity of the atoms, $a \equiv \Gamma/(4\pi\Delta\nu_{D})$ is the ratio of the damping width $\Gamma$ to the doppler width, and $v \equiv \Delta\nu/\Delta\nu_{D}$ is the difference in frequency from the line center in units of the doppler width.

Let's combine the line opacity with the continuum opacity and solve the equation of transfer.
For simplicity, we are going to assume pure absorption in both the continuum and the line.  Under these conditions, the source function is (see the notes on the Eddington atmosphere) $S_{\nu} = B_{\nu}$, the Planck function. For a plane-parallel atmosphere, the equation of transfer is then
\begin{equation}\label{e.cg-transfer}
\mu\frac{\dif I_{\nu}}{\dif\tau_{\nu}} = I_{\nu} - B_{\nu}
\end{equation}
where $\mu$ is the cosine of the angle of the ray with vertical. Solving equation~(\ref{e.cg-transfer}) for the emergent intensity at $\tau_{\nu} = 0$ gives
\begin{equation}\label{e.intensity}
I_{\nu}(\mu) = \frac{1}{\mu}\int_{0}^{\infty}\!B_{\nu}[T(\tau_{\nu})] \exp(-\tau_{\nu}/\mu) \,\dif\tau_{\nu}.
\end{equation}
The opacity is given by
\begin{equation}\label{e.total-opacity}
\kappa_{\nu} = \kappa_{\nu}^{C} + \chi_{\nu},
\end{equation}
where $\kappa_{\nu}^{C}$ is the continuum opacity and $\chi_{\nu} = \chi_{0}\phi_{\nu}$ is the line opacity, with 
\[
\chi_{0} = \frac{1}{\rho}\left(\frac{\pi e^{2}}{m_{e}c}\right)f_{ij}n_{i}\left(1 - e^{h\nu_{\ell}/kT}\right)
\]
being the line opacity at the line center $\nu_{\ell}$. 

As a further simplification, we can usually ignore the variation with $\nu$ in $\kappa_{\nu}^{C}$ over the width of the line. As a more suspect approximation (although it is not so bad in practice), let's assume that $\beta_{\nu} \equiv \chi_{\nu}/\kappa_{C}$ is independent of $\tau_{\nu}$. With this assumption we can write $\dif\tau_{\nu} = (1+\beta_{\nu})\dif\tau$, where $\tau = -\rho\kappa^{C}\,\dif z$. Finally, let's assume that in the line forming region, the temperature does not vary too much, so that we can expand $B_{\nu}$ to first order in $\tau$,
\[ B_{\nu}[T(\tau)] \approx B_{0} + B_{1}\tau, \]
where $B_{0}$ and $B_{1}$ are constants.
Inserting these approximations into equation~(\ref{e.intensity}), multiplying by the direction cosine $\mu$ and integrating over outward bound rays gives us the flux,
\begin{eqnarray}\label{e.flux}
F_{\nu} &=& 2\pi\int_{0}^{1}\!\int_{0}^{\infty}\!\left[B_{0}+B_{1}\tau\right]\exp\left[-\frac{\tau}{\mu}(1+\beta_{\nu})\right] \left(1+\beta_{\nu}\right) \,\dif\tau\,\dif\mu\nonumber\\
 &=& \pi\left[ B_{0} + \frac{2}{3}\frac{B_{1}}{1+\beta_{\nu}}\right].
\end{eqnarray}
Far from the line-center, $\beta_{\nu}\to 0$, implying that the continuum flux is
\[ F_{\nu}^{C} = \pi\left[B_{0} + \frac{2B_{1}}{3}\right]. \]
Hence the depth of the line is
\begin{equation}\label{e.line-depth}
A_{\nu} \equiv 1 - \frac{F_{\nu}}{F_{\nu}^{C}} = A_{0}\frac{\beta_{\nu}}{1+\beta_{\nu}},
\end{equation}
where
\begin{equation}\label{e.A0-curve-growth}
 A_{0} \equiv \frac{2B_{1}/3}{B_{0} + 2B_{1}/3}
 \end{equation}
is the depth of an infinitely opaque ($\beta_{\nu}\to\infty$) line. 

\noindent Now that we have the depth of the line $A_{\nu}$ we can compute the \emph{equivalent width},
\begin{equation}\label{e.W}
W_{\nu} \equiv \int_{0}^{\infty}\! A_{\nu}\,\dif\nu = A_{0}\int_{0}^{\infty}\!\frac{\beta_{\nu}}{1+\beta_{\nu}}\,\dif\nu.
\end{equation}
Let's change variables from $\nu$ to $v = \Delta\nu/\Delta\nu_{D} = (\nu-\nu_{\ell})/\Delta\nu_{D}$.  Since $H(a,v)$ is symmetrical about the line center, we will just integrate over $\Delta\nu >0$, giving
\begin{equation}\label{e.Wv}
 W_{\nu} = 2A_{0}\Delta\nu_{D}\int_{0}^{\infty}\!\frac{\beta_{0}H(a,v)}{1+\beta_{0}H(a,v)}\,\dif v,
 \end{equation}
with $\beta_{0} = \chi_{0}/(\kappa^{C}\Delta\nu_{D})$.

It's useful to understand the behavior of $W_{\nu}$ in various limits.  
First, at small line optical depth ($\beta_{0}\ll 1$) only the core of the line will be visible. Recall that in the core of the line, $H(a,v) \approx \exp(-v^{2})$ so we insert this into equation~(\ref{e.Wv}) and expand the denominator to give
\begin{eqnarray}\label{e.linear}
W_{\nu}^{\star} \equiv \frac{W{\nu}}{2A_{0}\Delta\nu_{D}} &=& \int_{0}^{\infty} \!\sum_{k=1}^{\infty}(-1)^{k-1}\beta_{0}^{k}e^{-kv^{2}}\,\dif v\nonumber\\
 &=& \frac{1}{2}\sqrt{\pi}\beta_{0}\left[1-\frac{\beta_{0}}{\sqrt{2}} + \frac{\beta_{0}^{2}}{\sqrt{3}} - \ldots\right].
\end{eqnarray}
Here $W_{\nu}^{\star}$ is the \emph{reduced equivalent width}.
Notice that since $\beta_{0}\propto 1/\Delta\nu_{D}$ (cf.~eq.~[\ref{e.voigt}]), the equivalent width $W_{\nu}$ is independent of $\Delta\nu_{D}$ in this \emph{linear regime}.
Physically, in the limit of small optical depth, each atom in state $i$ is able to absorb photons, and the flux removed  is just proportional to the number of atoms $n_{i}$.

As we increase $\beta_{0}$ eventually the core of the line saturates---no more absorption in the core is possible.  As a result, the equivalent width should be nearly constant until there are so many absorbers that the damping wings contribute to the removal of flux.  In the \emph{saturation regime}, the Voigt function is still given by $e^{-v^{2}}$, but we can no longer assume $\beta_{0}\ll 1$, so our expansion in equation~(\ref{e.linear}) won't work. Let's go back to our integral, eq.~(\ref{e.Wv}), change variables to $z= v^{2}$, and define $\alpha = \ln\beta_{0}$ to find
\[
W_{\nu}^{\star} = \frac{1}{2}\int_{0}^{\infty}\!\frac{z^{-1/2}}{e^{z-\alpha}+1}\,\dif z.
\]
This may not look like an improvement, but you might notice that it bears a resemblance to a Fermi-Dirac integral (see the notes on the equation of state). That means that very smart people figured out tricks to handle these integrals and all we have to do is look up what they did.  In this case we have Sommerfeld to thank. In this saturation regime,
\begin{equation}\label{e.saturation}
W_{\nu}^{\star} \approx \sqrt{\ln\beta_{0}}\left[ 1 - \frac{\pi^{2}}{24(\ln\beta_{0})^{2}} - \frac{7\pi^{4}}{384(\ln\beta_{0})^{4}}-\ldots\right].
\end{equation}
Note that the amount of flux removed is basically $2A_{0}\Delta\nu_{D}$: the line is maximally dark across the gaussian core.

Finally, if we continue to increase the line opacity, there will finally be so many absorbers that there will be significant flux removed from the wings.  Now the form of the Voigt profile is $H(a,v)\approx (a/\sqrt{\pi}) v^{-2}$, so our integral (eq.~[\ref{e.Wv}]) in this \emph{damping regime} becomes
\begin{eqnarray}\label{e.damping}
W_{\nu}^{\star} &=& \int_{0}^{\infty}\! \left(1+\frac{\sqrt{\pi}v^{2}}{\beta_{0}a}\right)^{-1}\, \dif v\nonumber\\
 &=& \frac{1}{2}\left(\pi a \beta_{0}\right)^{1/2}.
\end{eqnarray}
Note that since $a\beta_{0}\propto \Delta\nu_{D}^{-2}$, $W_{\nu}$ is again independent of the doppler width in this regime.

Now that we have this curve of growth, why is it useful? Since it only involves the equivalent width, it is possible to construct the curve of growth empirically without a high-resolution spectrum. Next, let's put some of the factors back into the quantities in the curve of growth.  First, for a set of lines, the population of the excited state depends on the Boltzmann factor $\exp(-E/kT)$. Second, we can expand out the Doppler width in both $W_{\lambda}^{\star}$ and $\beta_{0}$,
\begin{eqnarray}
\log\left(\frac{W_{\lambda}}{\Delta\lambda_{D}}\right) &=& \log\left(\frac{W_{\lambda}}{\lambda}\right) - \log\left(\frac{u_{0}}{c}\right)\label{e.ordinate}\\
\log\beta_{0} &=& \log(g_{i}f_{ij}\lambda) - \frac{E}{kT} +\log(N/\kappa^{C}) + \log C\label{e.abcissa}
\end{eqnarray}
where $C$ contains all of the constants and the continuum opacity.  The temperature $T$ is picked as a free parameter, and is picked to minimize scatter about a single curve that is assumed to fit all of the lines.  What is measured then is $\log(W_{\lambda}/\lambda)$ and $\log(g_{i}f_{ij}\lambda)$; by comparing them to theoretical curves one gets an estimate of $\log(u_{0}/c)$, the mean velocity of atoms (may be thermal or turbulent).  Since the continuum opacity $\kappa^{C}$ usually depends on the density of H, one gets from equation~(\ref{e.abcissa}) an estimate of the abundance of the line-producing element to H.

\section{Exercises}
\begin{enumerate}
\item When we observed the solar disk, light from the edges is coming at a slant through the atmosphere. This reduces the specific intensity and makes the sun appear darker around the edges (\emph{limb darkening}). Compute the reduction in intensity $I$ as a function of viewing angle.

\item This problem revisits an old argument, due to Schwarzschild, that convection is "impossible" in stellar atmospheres.  
\begin{enumerate}
\item Use Eddington's result for the run of temperature with optical depth in a stellar atmosphere, equation~(\ref{e.Eddington}), and equation~(\ref{e.photo-pressure}) 
to derive an expression for $P(\tau)$.  Then compute the temperature gradient
\begin{equation}
\frac{\dif\ln T}{\dif\ln P} \equiv \frac{P}{T}\frac{\dif T}{\dif P}
\end{equation}
in terms of the optical depth $\tau$.  The atmosphere becomes convectively unstable where (see p.~\pageref{e.schwarzschild})
\begin{equation}\label{e.conv}
 \nabla > \nabla_{\mathrm{ab}} \equiv \left(\frac{\partial\ln T}{\partial\ln P}\right)_{s}.
\end{equation}
Now suppose that along an adiabat the pressure obeys a polytropic relation, $P = K\rho^{1+1/n} \propto T^{1+n}$. Substitute this into equation~(\ref{e.conv}) to obtain a condition on $n(\tau)$ such that the atmosphere is convectively unstable.

\item What is the minimal value of $n$ such that convection happens \emph{somewhere} in the atmosphere? Would convection in fact occur under the assumptions stated?
\end{enumerate}

\item Question: why isn't $A_{0}=0$ in eq.~(\ref{e.A0-curve-growth})?

\item There are now many exoplanets in very tight orbits around a solar-type primary. Their photospheres are therefore irradiated strongly. As a first step in understanding how this irradiation affects the atmosphere, we'll stick with our Eddington approximation and grey atmosphere. (This is, as it turns out, a very bad approximation in this case, but the solution does give some insight). Assume that the spectrum of the primary star is Planckian.  
\begin{enumerate}
\item Derive an expression for the incident flux in terms of the primary's temperature $T_{\star}$, radius $R_{\star}$, and distance $D_{\star}$. 
\item Suppose that if the irradiation were not present, the planet would have an effective temperature $T_{\mathrm{int}}$ (due to gravitational contraction.  Using this and the result of part (a), estimate the effective temperature $T_{\mathrm{eff}}$ of the irradiated planet.
\item Now, you can repeat the analysis we derived in class, but your boundary condition at $\tau = 0$ will be different. Write the flux as the sum of two streams,
\[ F = \int_{0}^{1}I^{+}\mu\,\dif\mu + \int_{-1}^{0}I^{-}\mu\,\dif\mu,\]
where $I^{+}$ and $I^{-}$ are the outbound and inbound intensities, respectively.  Derive an expression for $T(\tau)$. 
\item Suppose a Jovian planet with $T_{\mathrm{int}} = 75\nsp\K$ were suddenly relocated to an orbit with a period of 5 days around a sun-like star. What would its new surface temperature be? Estimate the depth to which the atmosphere would be heated (give a value in terms of mass)?
\end{enumerate}

\item There is a subtlety involved when an atmospheric opacity is scattering-dominated, because scattering does not change the photon energy. Suppose we have an atmosphere where the Thomson scattering dominates the opacity, and the absorption of a photon is inverse bremsstrahlung (free-free), for which you can get the cross section expression from equation~(\ref{e.free-free-opacity}).  Note that we do not want the Rosseland mean here, we want to know what happens to a photon of a specific frequency. Finally, we are after scalings here, so don't get hung up on the precise value of numerical prefactors.
\begin{enumerate}
\item Is the opacity scattering-dominated at all frequencies?
\item Trace a photon of frequency $\nu$ back into the atmosphere.  How deep (in terms of the scattering optical depth) does it go before being absorbed? Is there a single well-defined photosphere for all frequencies? (\emph{Hint: the photon is taking a random walk into the star.})
\item Now, suppose the emergent intensity is still Planckian, but with a temperature that is the local temperature at the depth where the photon was last absorbed. Obtain an expression for $T$ and $\rho$ as a function of scattering optical depth, and use this to derive an approximate expression for the spectrum at high frequencies.  How does it compare to a blackbody at temperature $T_{\mathrm{eff}}$?
\emph{Hint}: you may find the article by Illarionov and Sunyaev (Astrophys. \& Space Science \textbf{19:} 61 [1972]) helpful.
\end{enumerate}
\end{enumerate}

%\section{A Constant-Flux Atmosphere}

%Consider an atmosphere (planar geometry) with constant gravity $\bvec{g} = -g \bvec{e}_{r}$ and a constant outward-directed flux $\bvec{F} = F\bvec{e}_{r}$. Let's assume that the gas can be described as a mixture of an ideal gas and radiation.  Further assume that the opacity is constant and grey (i.e., Thomson) and that the atmosphere is in LTE.

%Under what conditions does the gas pressure dominate?
%\begin{eqnarray}
%\Pgas&\gg& \Prad\nonumber\\
%\frac{\rho kT}{\mu \mb} &\gg& \frac{1}{3}aT^{4}\nonumber\\
%\rho &\gg& 0.03\mu \left(\frac{T}{10^{7}\nsp\K}\right)^{3}\nsp\grampercc.
%\end{eqnarray}
%Note the magnitude; this is easily satisfied for stars like the sun.  Now consider the equation for the flux,
%\begin{equation}
%F = -\frac{1}{3}\frac{ac}{\rho\kappa}\frac{\dif T^{4}}{\dif r}.
%\end{equation}
%Let's rewrite this using the identity $\dif/\dif r = (\dif P/\dif r)(\dif/\dif P) = -\rho g (\dif/\dif P)$ as
%\begin{equation}\label{e.F}
%F = -\frac{c}{\rho\kappa}\frac{\dif \Prad}{\dif r} = \frac{cg}{\kappa}\frac{\dif \Prad}{\dif P}.
%\end{equation}
%It is critical to remember that $P$ is the \emph{total} pressure, $P = \Prad + \Pgas$. Now for something clever.  Since $F = L/(4\pi R^{2})$ and $g = GM/R^{2}$, we can cancel $R^{2}$ from both sides of equation~(\ref{e.F}). Further, recall that $4\pi GMc/\kappa = \Ledd$, the Eddington luminosity, so that we can write
%\begin{equation}\label{e.Prad}
%\frac{\dif \Prad}{\dif P} = \frac{L}{\Ledd}.
%\end{equation}
%Since $\dif \Prad/\dif P = 4 (\Prad/T) (\dif T/\dif P)$, we can work out immediately how $\dif T/\dif P$ changes with depth.

%Since we are assuming an atmosphere with constant $L/\Ledd$, we can integrate equation~(\ref{e.Prad}) inward from the photosphere,
%\begin{equation}
%\Prad - P_{\mathrm{rad,0}} = \frac{L}{\Ledd}(P-P_{0}).
%\end{equation}
%Later on we will get the correct expression for the photospheric values (with subscript ``0''), but for now, let's go sufficiently deep that we can ignore the outer boundary, in which case,
%\begin{equation}
%\Prad \approx \frac{L}{\Ledd}P = \frac{L}{\Ledd}(\Pgas + \Prad).
%\end{equation}
%Dividing through by \Pgas, we obtain the desired result,
%\begin{equation}
%\frac{\Prad}{\Pgas} = \frac{L}{\Ledd}\left(1-\frac{L}{\Ledd}\right)^{-1}.
%\end{equation}
%The right-hand side is a constant; note that for the sun, $L/\Ledd \approx  3\times 10^{-5}$.

% !TEX root = ./notes.tex
\chapter{Non-resonant nuclear reactions}
\newcommand{\rn}{\ensuremath{r_{\mathrm{N}}}}
\newcommand{\re}{\ensuremath{r_{\mathrm{E}}}}
\newcommand{\barn}{\ensuremath{\mathrm{b}}}
\newcommand{\EG}{\ensuremath{E_{\mathrm{G}}}}
\newcommand{\Epk}{\ensuremath{E_{\mathrm{pk}}}}

The situation we are interested in is the reaction between two nuclei, $(A_{1},Z_{1})$ and $(A_{2},Z_{2})$.  The nuclear radius is $\rn\approx A^{1/3 }\times 10^{-13}\nsp\cm = A^{1/3}\nsp\fermi$, and the Coulomb energy at this distance is
\begin{equation}\label{e.cb}
\frac{Z_{1}Z_{2}e^{2}}{\rn} = \frac{Z_{1}Z_{2}\alpha \hbar c}{ \rn} \approx 1.4 Z_{1}Z_{2}A^{-1/3}\nsp\MeV\gg kT.
\end{equation}
Here $\alpha = e^{2}/(\hbar c) = 1/137$ is the fine-structure constant and $\hbar c = 197 \nsp\MeV\usp\fermi$.  Remember these numbers!  If you want to impress your friends, you can do this in your head by remembering that $e^{2} = \alpha \hbar c = (197\nsp\MeV\usp\fermi)/137 = 1.44\nsp\MeV\usp\fermi = 1440\nsp\keV\usp\fermi$.  Clearly the cross-section for a reaction between our pair of particles is controlled by the probability of tunneling through the Coulomb potential.

For a two-body system, it is convenient to transform into a center-of-mass frame.  Our problem then reduces to a one-body problem with reduced mass $m = A\mb$, with $A=A_{1}A_{2}/(A_{1}+A_{2})$ and incident energy $E = m v^{2}/2$, where $v$ is the relative velocity of the two particles.  For now, we'll neglect angular momentum ($\ell = 0$) so our scattering is s-wave.
At low energies, we can form a ``geometrical'' cross-section from the particle wavenumber $k = p/\hbar$, with
\begin{equation}\label{e.geo}
\pi k^{-2} = \pi\frac{\hbar^{2}}{(2mE)^{1/2}} = 660\nsp\barn\frac{1}{A}\left(\frac{\keV}{E}\right)^{1/2}
\end{equation}
Here the cross-section is in units of \emph{barns}, with $1\nsp\barn = 10^{-24}\nsp\cm^{2}$.  This is the first part of our nuclear cross-section $\sigma(E)$. 

The second portion of the nuclear cross-section is the probability of tunneling through the Coulomb barrier.  First, let's get the classical turning point \re\ from
\begin{eqnarray}
\frac{Z_{1}Z_{2}e^{2}}{\re} &=& E,\nonumber\\
\re &=& 1440\nsp\fermi\nsp Z_{1}Z_{2}\left(\frac{\keV}{E}\right).
\end{eqnarray}
Now the wavelength is $k^{-1} = \hbar(2A\mb E)^{-1/2} = \hbar c (2A\mb c^{2} E)^{-1/2}$ and since $\mb c^{2} = 932\nsp\MeV$ the wavelength $k^{-1} = 145\nsp\fermi\usp(\keV/E)^{1/2}$. The important point is that since $k^{-1}\ll \re$, we can solve the Schr\"odinger equation using the WKB approximation.

The WKB approximation is standard, so let me just remind you that the probability of tunneling through the barrier depends on the \emph{action},
\begin{equation}\label{e.wkb}
\mathcal{P} \propto \exp\left\{\frac{2}{\hbar}\int_{\re}^{\rn}\left[2m\left(\frac{Z_{1}Z_{2}e^{2}}{r}-E\right) \right]^{1/2}\,\dif r\right\}.
\end{equation}
To do this integral, note that $\re\gg\rn$, so we can make the approximation $\rn\to 0$ and change variables to
\[
\sin\phi = \left[2m\left(\frac{Z_{1}Z_{2}e^{2}}{r}-E\right) \right]^{1/2}\frac{r}{2mZ_{1}Z_{2}e^{2}};
\]
with this substitution we tame the integral and obtain
\begin{equation}\label{e.prob}
\mathcal{P} \propto \exp\left\{-\frac{8mZ_{1}Z_{2}e^{2}}{\hbar(2mE)^{1/2}}\int_{0}^{\pi/2}\sin^{2}\phi\,\dif\phi\right\} = \exp\left[-\left(\frac{\EG}{E}\right)^{1/2}\right],
\end{equation}
where
\begin{equation}\label{e.EGamow}
\EG\equiv 2\pi^{2} A \mb c^{2} \alpha^{2} (Z_{1}Z_{2})^{2} = 979\nsp\keV \nsp A (Z_{1}Z_{2})^{2}
\end{equation}
is the \emph{Gamow energy}.  Note the strong dependence on $Z_{1}Z_{2}$: \EG\ determines which reactions can occur at a given temperature. If you stare at the factor multiplying the integral in equation~(\ref{e.prob}), you will see that $\mathcal{P}\propto \exp(-\re/\lambda)$, the exponential of ratio of the width of the forbidden region to the wavelength of the incident particle. This makes intuitive sense.

Now we have the second part of our cross-section, the probability of getting through the Coulomb barrier.  This third part depends on the nuclear interactions.  For non-resonant reactions, this third part does not depend strongly on energy, so it is common to define the \emph{astrophysical S-factor} by writing the cross section as the product $(\textrm{geometrical})\times(\textrm{tunneling})\times(\textrm{nuclear})$, 
\begin{equation}\label{e.s-def}
\sigma(E) = \frac{1}{E}\exp\left[-\left(\frac{\EG}{E}\right)^{1/2}\right] S(E).
\end{equation}
It is easier to extrapolate the slowly varying $S(E)$ from lab energies of $> 100\nsp\keV$ down to center-of-mass energies of $\sim \keV$ than it would be to fit the rapidly varying cross-section.

Now each nucleus has a Maxwellian velocity distribution,
\begin{equation}\label{e.maxwell}
n_{1}(\bvec{v}_{1})\,\dif^{3}v = n_{1}\left(\frac{m_{1}}{2\pi kT}\right)^{3/2}\exp\left(-\frac{mv^{2}}{2kT}\right) \,\dif^{3}v,
\end{equation}
and similarly for particle 2.  Let's call a particular nucleus 1 (having velocity $\bvec{v}_{1}$) the target. By definition the cross section is 
\[ \frac{\textrm{number of reactions}/\textrm{target}/\textrm{time}}{\textrm{number of incident particles}/\textrm{area}/\textrm{time}},
\]
so to get the number of reactions per target per time we need to multiply $\sigma(E)$ by the number of incident particles per unit area per unit time.  The incident flux is just $n_{2}(\bvec{v}_{2})|\bvec{v}|\,\dif^{3}v_{2}$ where $\bvec{v}=\bvec{v}_{2}-\bvec{v}_{1}$.  Hence the reaction rate per unit volume per unit time between a pair of particles having velocities in volumes $\dif^{3}v_{1}$ and $\dif^{3}v_{2}$ about $\bvec{v}_{1}$ and $\bvec{v}_{2}$ is just
\[
\frac{1}{1+\delta_{12}} n_{1}(\bvec{v}_{1})n_{2}(\bvec{v}_{2})\sigma(E)|\bvec{v}| \,\dif^{3}v_{1}\dif^{3}v_{2}.
\]
The factor $(1+\delta_{12})^{-1}$ is equal to $1/2$ if particles 1 and 2 are identical, and is there to avoid double-counting in that case. To get the total reaction rate per unit time, we need to integrate over the joint velocity distribution $\dif^{3}v_{1}\,\dif^{3}v_{2}$,
\begin{eqnarray}\label{e.rate-joint}
\lefteqn{r_{12} = \frac{n_{1}n_{2}}{1+\delta_{12}}  \left[\frac{m_{1}m_{2}}{(2\pi kT)^{2}}\right]^{3/2}}
  \nonumber\\ &&\times\int\! \sigma(E) v\exp\left(-\frac{m_{1}v_{1}^{2}}{2kT}-\frac{m_{2}v_{2}^{2}}{2kT}\right)  \,\dif^{3} v_{1}\,\dif^{3}v_{2}.
\end{eqnarray}
Now $E$ and $v$ are the relative energies and velocity in the center-of-mass frame.  We can change variable using the relations
\begin{eqnarray*}
\bvec{v}_{1} &=& \bvec{V} - \frac{m_{2}}{m_{1}+m_{2}} \bvec{v}\\
\bvec{v}_{2} &=& \bvec{V} + \frac{m_{1}}{m_{1}+m_{2}} \bvec{v}.
\end{eqnarray*}
where $V$ is the center-of-mass velocity. It is straightforward to show that $\dif v_{1,x}\,\dif v_{2,x} = \dif V_{x}\dif v_{x}$, and likewise for the $y,z$ directions.  Furthermore, $m_{1}v_{1}^{2} + m_{2}v_{2}^{2} = (m_{1}+m_{2})V^{2} + m v^{2}$, and multiplying and dividing the integral in equation~(\ref{e.rate-joint}) by $m_{1}+m_{2}$ allows us to write
\begin{eqnarray*}
\lefteqn{r_{12} = \frac{n_{1}n_{2}}{1+\delta_{12}} \left(\frac{m_{1}+m_{2}}{2kT}\right)^{3/2}\left(\frac{m}{2kT}\right)^{3/2}}\\
&&\times \int\!\dif^{3}V \int\!\dif^{3}v \,\sigma(E)v \exp\left[-\frac{mv^{2}}{2kT}\right]
 \exp\left[-\frac{(m_{1}+m_{2})V^{2}}{2kT}\right].
\end{eqnarray*}
The integral over $\dif^{3}V$ can be factored out and is normalized to unity. Hence we have for the reaction rate between a pair of particles 1 and 2, 
\begin{eqnarray}\label{e.rate}
r_{12} &=& \frac{1}{1+\delta_{12}}n_{1}n_{2}\left\{\left(\frac{m}{2\pi kT}\right)^{3/2}\int_{0}^{\infty}\! \sigma(E) v \exp\left(-\frac{mv^{2}}{2kT}\right)  4\pi v^{2}\,\dif v\right\}.\nonumber\\
 &\equiv& \frac{1}{1+\delta_{12}}n_{1}n_{2}\langle\sigma v\rangle.
\end{eqnarray}
The term in $\{\}$ is the averaging over the joint distribution of the cross-section times the velocity, and is usually denoted as $\langle\sigma v\rangle$. 

Changing variables to $E = mv^{2}/2$ in equation~(\ref{e.rate}) and inserting the formula for the cross-section, equation~(\ref{e.s-def}), gives
\begin{equation}\label{e.integral}
\langle\sigma v\rangle = \left(\frac{8}{\pi m}\right)^{1/2}\left(\frac{1}{kT}\right)^{3/2}\int_{0}^{\infty}\!S(E)\exp\left[-\left(\frac{\EG}{E}\right)^{1/2}-\frac{E}{kT}\right]\,\dif E.
\end{equation}
Now, we've assumed that $S{E}$ varies slowly; but look at the argument of the exponential. This is a competition between a rapidly rising term $\exp[-(\EG/E)^{1/2}]$ and a rapidly falling term $\exp(-E/kT)$. As a result, the exponential will have a strong peak, and we can expand the integrand in a Taylor series about the maximum. Let 
\[
f(E) = -\left(\frac{\EG}{E}\right)^{1/2} - \frac{E}{kT}.
\]
Then we can write 
\begin{eqnarray*}
\lefteqn{\int_{0}^{\infty}\!S(E)\exp\left[-\left(\frac{\EG}{E}\right)^{1/2}-\frac{E}{kT}\right]\,\dif E}\\
&\approx&
	\int_{0}^{\infty}\! S(\Epk)\exp\left[f(\Epk) + \frac{1}{2}\left.\frac{\dif^{2} f}{\dif E^{2}}\right|_{E=\Epk}\left(E-\Epk\right)^{2}\right].
\end{eqnarray*}
Here $\Epk$ is found by solving $(\dif f/\dif E)|_{E=\Epk} = 0$. This trick allows us to turn the integral into a Gaussian! (Before the internet, all there was to do for fun were integrals.)

Solving for \Epk, we get
\[
\Epk = \frac{\EG^{1/3}(kT)^{2/3}}{2^{2/3}},
\]
and 
\[ \exp\left[f(\Epk)\right] = \exp\left[-3\left(\frac{\EG}{4kT}\right)^{1/3}\right].
\]
Further,
\[
\left.\frac{1}{2}\frac{\dif^{2}f}{\dif E^{2}}\right|_{E=\Epk} = -\frac{3}{2(2\EG)^{1/3}(kT)^{5/3}} = -\frac{3}{4\Epk kT}.
\]
Defining a variable $\Delta = 4(\Epk kT/3)^{1/2}$, our integral becomes
\begin{eqnarray}\label{e.integral2}
\lefteqn{\langle\sigma v\rangle = \left(\frac{8}{\pi m}\right)^{1/2}\left(\frac{1}{kT}\right)^{3/2}}\nonumber\\
&&\times S(\Epk)
  \exp\left[-3\left(\frac{\EG}{4kT}\right)^{1/3}\right]
  \int_{0}^{\infty}\!\exp\left[-\frac{(E-\Epk)^{2}}{(\Delta/2)^{2}}\right]\,\dif E.
\end{eqnarray}
How well does this approximation do?  Figure~\ref{f.integrand} shows the integrand (\emph{solid line}) and the approximation by a Gaussian (\emph{dashed line}).  Although the integrand is skewed to the right, the area is approximately the same.  We could correct for this by taking more terms in our expansion, but then the integral would become more difficult, not less.

\begin{figure}[htbp]
\includegraphics[width=4in]{plots_out/coulomb_integrand}
\caption{Integrand of eq.~(\protect\ref{e.integral}) (\emph{solid line}) and the Gaussian (\emph{dashed line}) constructed by expanding to second order the argument of the exponential. The parameters for $\EG$ were taken from the $p+p$ reactions ($Z_{1}Z_{2}=1$, $A = 1/2$), and the temperature is $10^{7}\nsp\K$.}
\label{f.integrand}
\end{figure}

There is another nice benefit, and that is both the Gaussian and the original integrand go to zero as $E\to 0$.  As a result, we can extend the lower bound of our integral (eq.~[\ref{e.integral2}]) to $-\infty$, and obtain
\begin{eqnarray}\label{e.rate2}
\langle\sigma v\rangle &\approx& \left(\frac{8}{\pi m}\right)^{1/2}\left(\frac{1}{kT}\right)^{3/2} S(\Epk) \exp\left[-3\left(\frac{\EG}{4kT}\right)^{1/3}\right]\frac{\Delta}{2}\nonumber\\
 &=& \frac{4\cdot 2^{1/6}}{\sqrt{3m}}\frac{\EG^{1/6}}{(kT)^{2/3}} \exp\left[-3\left(\frac{\EG}{4kT}\right)^{1/3}\right]  S(\Epk).
\end{eqnarray}
On to some numbers. Table~\ref{t.reaction} lists quantities for some common reactions. A couple of notes. First, $\Delta/\Epk$ indicates how well our Gaussian approximation works---you will see it is less than 1 in all cases. We evaluated $\Delta/\Epk$, which decreases with temperature as $T^{-1/6}$, at $T = 10^{7}\nsp\K$. Second, the quantity $n(T)$ is the exponent if we want to approximate the reaction rate as a power-law, $r\propto T^{n}$.  We compute this as 
\begin{equation}\label{e.exponent}
 n(T) = \frac{\dif\ln r}{\dif\ln T} = -\frac{2}{3} + \left(\frac{\EG}{4kT}\right)^{1/3},
 \end{equation}
 as you can easily verify for yourself. In the table, the exponent is evaluated at $T = 10^{7}\nsp\K$; obviously $n$ depends on temperature. Finally, note the size of $\EG/(4k)$.  This makes the argument of the exponential in equation~(\ref{e.rate2}) large in absolute value, and sets the temperature scale at which a given reaction comes into play.
 
\begin{table}[htbp]
\caption{\label{t.reaction} Parameters for non-resonant reactions}
\begin{tabular}{lrrrrrr}
\\ \hline
Reaction & $p+p$ & $p+\helium[3]$ & $\helium[3]+\helium[3]$ & $p+\lithium[7]$ & $p+\carbon$\\
\hline\hline
$A$ & 1/2 & 3/4 & 3/2 & 0.88 & 0.92 \\
$Z_{1}Z_{2}$ & 1 & 1 & 4 & 3 & 6 \\
$\EG$ (keV) & 489 & $2.94\ee{3}$ & $2.35\ee{4}$ & $7.70\ee{3}$ & $3.25\ee{4}$\\
$\EG/(4k)$ (K) & $1.4\ee{9}$ & $8.5\ee{9}$ & $6.8\ee{10}$ & $2.2\ee{10}$ & $9.4\ee{10}$ \\
$\Epk|_{T=10^{7}\nsp\K}$ (keV) & 4.5 & 8.2 & 16.3 & 11.3 & 18.2\\
$\Delta/\Epk|_{T=10^{7}\nsp\K}$ & 1.0 & 0.75 & 0.53 & 0.64 & 0.50 \\
$n(T = 10^{7}\nsp\K)$ & 4.6 & 8.8 & 18.3 & 12.4 & 20.5\\
\hline
\end{tabular}
\end{table}

\section{Exercises}

A simple model for the mass of a nucleus with $Z$ protons and $N=A-Z$ neutrons is
\begin{equation}
M(N, Z) = Z m_{p} + N m_{n} + a_{V} A + a_{S}A^{2/3} + a_{A}\frac{(N-Z)^{2}}{A} + a_{C}\frac{Z^{2}}{A^{1/3}}.
\end{equation}
Here masses are in MeV, so $m_{p} = 938.272\nsp\MeV$ and $m_{n} = 939.565\nsp\MeV$. Using the online fitting routine from the \href{http://128.95.95.61/~intuser/ld.html}{Institute for Nuclear Theory}, I obtained the following coefficients.

\begin{tabular}{l|rrrr}
\hline
coefficient & $a_{V}$ & $a_{S}$ & $a_{A}$ & $a_{C}$\\
\hline
value (MeV) & -15.5 & 16.6 & 22.7 & 0.71\\
\hline
\end{tabular}

\begin{enumerate}
\item Define the \emph{binding energy} as $B(N,Z) = Zm_{p} + Nm_{n} - M(N,Z)$; this is the energy released in assembling a nucleus from a collection of neutrons and protons.  
\begin{enumerate}
\item For a fixed $A$, what is the $Z_\star(A)$ such that the binding energy per nucleon, $f = B(N,Z)/A$ is maximized?  
\item\label{p.one} Plot $Z_{\star}$ vs $N$.
\item Using this $Z_{\star}$, plot $Y_{e}=Z_{\star}/A$ for $4 < A < 200$ and explain qualitatively any trends.
\item Now substitute the value of $Z_{\star}$ into the expression for $B(A,Z)$ and plot $B(A,Z_{\star})$ as a function of $A$. Explain qualitatively any trends.
\end{enumerate}
\item Define the \emph{neutron separation energy} $S_{n}$ as the energy needed to remove a neutron from a nucleus,
\begin{eqnarray}\label{e.Sn}
S_{n}(N,Z) &\equiv& \left[M(N-1,Z) + m_{n}\right] - M(N,Z) \nonumber\\
	&=& B(N,Z) - B(N-1,Z).
\end{eqnarray}
Likewise, define the \emph{proton separation energy} as
\begin{eqnarray}\label{e.Sp}
S_{p}(N,Z) &\equiv& \left[M(N,Z-1) + m_{p}\right] - M(N,Z) \nonumber\\
	&=& B(N,Z) - B(N,Z-1).
\end{eqnarray}
\begin{enumerate}
\item\label{p.two} For each $2\le Z\le 82$, find the maximum value of $N$ such that $S_{n}(A=N+Z,Z) > 0$.  Plot the values $(N,Z)$ you find.  
\item\label{p.three} For each $2\le N\le 120$, find the maximum value of $Z$ such that $S_{p}(A=N+Z,Z) > 0$. Plot the values $(N,Z)$ you find. 
\end{enumerate}

\item Compare the plots of problems \ref{p.one}, \ref{p.two}, and \ref{p.three} to a chart of the nuclides.

\item Compute the mass of H, in units of solar masses, that must be converted into \helium\ in order to supply the solar luminosity over $10^{10}\nsp\yr$.
\end{enumerate}

 

\bibliographystyle{apj}
\bibliography{master}

\appendix
% !TEX root = ./stellar-notes.tex
\chapter[Composition]{Specifying the Composition of a Multi-Component Plasma}
\label{s.composition}

In this appendix we'll look at how one specifies the composition for a multi-component plasma.  To make things concrete, let's imagine a box containing a mixture of nuclei, of many different isotopes, and electrons.  (To keep things simple, we'll assume complete ionization.)  Each isotope species $i$ has $N_{i}$ nuclei present, and is characterized by charge number $Z_{i}$ and nucleon number $A_{i}$.  Charge neutrality then specifies the number of electrons,
\begin{equation}\label{e.number-e}
N_{\mathrm{e}} = \sum_{i} Z_{i} N_{i}.
\end{equation}
The total mass of the box is
\begin{equation}\label{e.total-mass}
M = \me N_{\mathrm{e}} + \sum_{i} m_{i}N_{i},
\end{equation}
where $\me$ and $m_{i}$ are respectively the mass of an electron and a nucleus of species $i$.  Now what is $m_{i}$? Breaking a nucleus $i$ into $Z_{i}$ protons and $A_{i}-Z_{i}$ neutrons takes a certain amount of energy, the \emph{binding energy} $B_{i}$.  We can therefore write $m_{i} = Z_{i}\mpr + (A_{i}-Z_{i})\mn - B_{i}/c^{2}$, where $\mpr$ and $\mn$ are respectively the proton and neutron rest masses.

Inserting our expression for $m_{i}$ into equation~(\ref{e.total-mass}), dividing by the volume of the box $V$, and rearranging terms gives us the mass density,
\begin{equation}\label{e.rho}
\rho = \frac{M}{V} = \sum_{i} n_{i}\left[ \left(A_{i}-Z_{i}\right) \mn + Z_{i}\left(\mpr + \me\right) - B_{i}/c^{2}\right].
\end{equation}
Here $n_{i}$ is the number density of isotope species $i$, and we have used equation~(\ref{e.number-e}) to eliminate $N_{\mathrm{e}}$.  The numbers $n_{i}$ are, of course, fantastically large, so chemists and astronomers define \emph{Avogadro's constant} to be
\begin{equation}\label{e.avogadro-def}
\NA \equiv 6.0221367\ee{23}\nsp\unitstyle{mol}^{-1},
\end{equation}
that is, in one \emph{mole} of anything, there are $\NA$ items. If we multiply and divide the right-hand side of equation~(\ref{e.rho}) by $\NA$, we then have
\begin{equation}\label{e.molar-1}
\rho = \sum_{i} \left(\frac{n_{i}}{\NA}\right) \mathcal{A}_{i},
\end{equation}
where
\begin{equation}\label{e.gm-mol}
\mathcal{A}_{i} = \left[ \left(A_{i}-Z_{i}\right) \mn + Z_{i}\left(\mpr + \me\right) - B_{i}/c^{2}\right]\times\NA
\end{equation}
is the \emph{gram-molecular weight} of species $i$ with dimensions $[\mathcal{A}]\sim[\gram\cdot\mol^{-1}]$. Strictly speaking, the gram-molecular weight actually refers to the mass of a mole of the isotope in \emph{atomic} form; the right-hand side of eq.~(\ref{e.gm-mol}) is the gram-molecular weight neglecting the electronic binding energy.

Now you may wonder where the numerical value of $\NA$ came from.  It was not pulled out of thin air, but is defined so that 1\usp\mol\ of \carbon\ has a mass of exactly 12\nsp\gram.  In other words, for \carbon\, $\mathcal{A} \equiv A\nsp\gram\usp\mol^{-1}$.  In fact for all nuclei, $\mathcal{A} \approx A \usp\gram\nsp\mol^{-1}$ to better than about 1\%, as demonstrated in Table~\ref{t.gm-mol}.

\begin{table}[htbp]\caption{\label{t.gm-mol}Selected gram-molecular weights.}
\begin{center}
\begin{tabular}{r|ccc}
\hline
nuclide & $A$ & $\mathcal{A}$ & $(|\mathcal{A}-A|/A) \times 100$\\
\hline\hline
\neutron & 1 & 1.00865 & 0.865\\
\hydrogen & 1 & 1.00783 & 0.783\\
\helium & 4 & 4.00260 & 0.065\\
\oxygen & 16 & 15.99491 & 0..032\\
\silicon & 28 & 27.97693 & 0.082\\
\iron & 56 & 55.93494 & 0.116\\
\hline
\end{tabular}\end{center}
\end{table}

Because in CGS $\mathcal{A}\approx A$, it is customary to write $\mathcal{A} = A\times (1\usp\gram\nsp\mol^{-1})$, so that equation~(\ref{e.molar-1}) is
\begin{equation}\label{e.molar-2}
\rho = \sum_{i} \left(\frac{n_{i}}{\NA}\times 1\frac{\gram}{\mol}\right) A_{i}.
\end{equation}
This equation is exact if $A$ is now understood to be a real number, but the custom is to just keep it as the nucleon number. Astronomers typically then commit the sin of omission and \emph{redefine} $\NA$ in this context to mean $\NA / (1\nsp\gram\usp\mol^{-1}) = 6.022\ee{23}\nsp\gram^{-1}$. Like all sins, this can lead to grief: you can only get away with this in CGS. I prefer to use the atomic mass unit, defined as $1/12$ the mass of an atom of \carbon, so that $1\nsp\amu =  (1\nsp\gram\usp\mol^{-1})/\NA = 1.66054\ee{-24}\nsp\gram$. This puts equation~(\ref{e.molar-2}) into the more obvious form $\rho = \sum n_{i}\times A_{i}\mb$, with $\mb$ having a mass of $1\nsp\amu$.

With the redefinition of \NA, equation~(\ref{e.molar-2}) can be rewritten as
\begin{equation}\label{e.}
1 = \sum_{i}\left(\frac{n_{i}}{\NA\rho}\right)A_{i} \equiv \sum_{i}Y_{i}A_{i}
\end{equation}
where $Y_{i} \equiv n_{i}/\rho/\NA$ is the \emph{molar fraction}. It is customary to call $Y_{i}A_{i}$ the \emph{mass fraction} $X_{i}$, with $\sum X_{i} = 1$. We can then define the mean atomic mass number,
\begin{equation}\label{e.mean-A}
\bar{A} = \frac{\sum A_{i}Y_{i}}{\sum Y_{i}} = \frac{1}{\sum Y_{i}},
\end{equation}
and mean charge number
\begin{equation}\label{e.mean-Z}
\bar{Z} = \frac{\sum Z_{i}Y_{i}}{\sum Y_{i}} = \bar{A} \sum Z_{i}Y_{i}.
\end{equation}
The molar fraction of electrons is
\begin{equation}\label{e.Ye}
Y_{e} = \sum Z_{i} \frac{n_{i}}{\rho\NA} = \sum Z_{i}Y_{i} = \frac{\bar{Z}}{\bar{A}}.
\end{equation}
In stellar structure work, it is common to use the \emph{mean molecular weight}, defined so that the total number of particles, including electrons, per unit volume is
\begin{equation}\label{e.mean-molecular-weight}
\sum_{i} n_{i} + n_{e} \equiv \frac{\rho\NA}{\mu}.
\end{equation}
Yes, this is still the redefined $\NA$: $\mu$ is dimensionless. From the definition,
\[
\mu = \left(\sum_{i}Y_{i} + Y_{e}\right)^{-1} = \left[ \sum_{i}\left(Z_{i}+1\right)Y_{i} \right]^{-1};
\]
sometimes astronomers also define the mean ion molecular weight, $\mu_{I} = (\sum Y_{i})^{-1}$, and the mean electron weight, $\mu_{e} = Y_{e}^{-1}$.

\section{Exercises}
\begin{enumerate}
\item Consider a gas of \hydrogen\ and \helium\ with molar hydrogen fraction $Y_{\mathrm{H}}$.  Derive expressions for the molar fraction of \helium, $Y_{\mathrm{He}}$, $\bar{A}$, $\bar{Z}$, and $\mu$. What are the numerical value of these quantities for $Y_{\mathrm{H}} = 0.7$, i.~e., solar?
\item Assume that we can describe this plasma as an ideal gas.  What is the sound speed and the average kinetic energy of a particle, for a given mass density and temperature?
\end{enumerate}

% !TEX root = ./notes.tex
\chapter[Thermodynamical Derivatives]{Transforming Thermodynamical Derivatives}
\label{s.thermo-derivatives}

A common task in stellar physics is transforming between different derivatives with respect to different thermodynamical quantities.  For example, you may have expressions for $(\partial \kappa/\kappa T)_{\rho}$ and $(\partial \kappa/\partial \rho)_{T}$, but you need $(\partial\kappa/\partial T)_{P}$ and $(\partial\kappa/\partial P)_{T}$.  There is a straightforward way to handle transforming from $(\rho,T)$ space to $(P,T)$ space, and that is using Jacobians.  Despite the utility of this technique, it is not commonly discussed in astrophysical texts; my notes below follow \citet{landau80:_statis_physic}.

The \emph{Jacobian} is defined as the determinant of a matrix of partial derivatives,
\begin{eqnarray}
\jac{a}{b}{c}{d} &\equiv& \det\left[
	\begin{array}{lr}\tderiv{a}{c}{d} & \tderiv{a}{d}{c}\\
	\tderiv{b}{c}{d} & \tderiv{b}{d}{c} \end{array}\right] \nonumber \\
 & = & \tderiv{a}{c}{d}\tderiv{b}{d}{c} - \tderiv{a}{d}{c}\tderiv{b}{c}{d}.
 \end{eqnarray}
Because interchanging any the rows (or the columns) causes the determinant to change sign, 
\begin{equation}
\jac{b}{a}{c}{d} = -\jac{a}{b}{c}{d}
\end{equation}
and
\begin{equation}
\jac{a}{b}{d}{c} = -\jac{a}{b}{c}{d}.
\end{equation}
Further,
\begin{equation}
\jac{a}{s}{c}{s} = \tderiv{a}{c}{s}\tderiv{s}{s}{a} - \tderiv{a}{s}{c}\tderiv{s}{c}{s} = \tderiv{a}{c}{s},
\end{equation}
and
\begin{equation}
\jac{a}{b}{a}{b} = \tderiv{a}{a}{b}\tderiv{b}{b}{a} - \tderiv{a}{b}{a}\tderiv{b}{a}{b} = 1.
\end{equation}
Hence we can write thermodynamical derivative in terms of Jacobians, for example,
\begin{equation}
\tderiv{T}{P}{S} = \jac{T}{S}{P}{S}.
\end{equation}
Finally, when multiplying two Jacobians, one can ``cancel'' identical upper and lower parts,
\begin{equation}
\jac{a}{b}{c}{d}\jac{c}{d}{s}{t} =\jac{a}{b}{s}{t},
\end{equation}
as can be readily checked by expanding out both the left and right hand sides.

Here's a simple worked example of how we can use these identities.  Suppose we need the quantity $(\partial \kappa/\partial T)_{P}$, but our formula for the opacity is in terms of $\rho$ and $T$. We can express $(\partial\kappa/\partial T)_{P}$ as
\begin{eqnarray}
\tderiv{\kappa}{T}{P} &=& \jac{\kappa}{P}{T}{P}\nonumber\\
 &=& \jac{\kappa}{P}{T}{\rho}\jac{T}{\rho}{T}{P}\nonumber\\
 &=& \tderiv{\rho}{P}{T}\left[\tderiv{\kappa}{T}{\rho}\tderiv{P}{\rho}{T}-\tderiv{\kappa}{\rho}{T}\tderiv{P}{T}{\rho}\right]
 \nonumber\\
 &=& \tderiv{\kappa}{T}{\rho} - \tderiv{\kappa}{\rho}{T}\jac{\rho}{T}{P}{T}\jac{P}{\rho}{T}{\rho}\nonumber\\
 &=& \tderiv{\kappa}{T}{\rho} - \tderiv{\kappa}{\rho}{T} \frac{\chi_{T}}{\chi_{\rho}}\frac{\rho}{T}.
\end{eqnarray}
Here we used the common astrophysical notation
\begin{equation}
\chi_{T}\equiv\frac{T}{P}\tderiv{P}{T}{\rho},\qquad\chi_{\rho}\equiv\frac{\rho}{P}\tderiv{P}{\rho}{T}.
\end{equation}
The exponents $\chi_{T}$ and $\chi_{\rho}$ occur frequently in fluid dynamics; for a fixed composition the equation of state can be written as
\begin{equation}\label{e.eos}
P = P_{0}\rho^{\chi_{\rho}}T^{\chi_{T}},
\end{equation}
where $P_{0}$ s a constant.

\section{Exercises}\label{s.thermo-exercises}
\begin{enumerate}
\item Show that 
\[
 	\left(\frac{\partial T}{\partial P}\right)_{S} 
 	\left(\frac{\partial S}{\partial T}\right)_{P} 
 	\left(\frac{\partial P}{\partial S}\right)_{T} = -1
\]

\item Show that
\[ \tderiv{P}{s}{\rho} = \frac{T}{c_{\rho}}\chi_{T},\qquad \tderiv{P}{\rho}{s} = \frac{P}{\rho}\left[\chi_{\rho} + \chi_{T}\left(\Gamma_{3}-1\right)\right],
\]
where
\begin{eqnarray*}
c_{\rho} &=& T\tderiv{s}{T}{\rho}\\
\chi_{T} &=& \frac{P}{T}\tderiv{P}{T}{\rho} \equiv \tderiv{\ln P}{\ln T}{\rho}\\
\chi_{\rho} &=& \tderiv{\ln P}{\ln\rho}{T}\\
\Gamma_{3} - 1 &=& \tderiv{\ln T}{\ln\rho}{s}.
\end{eqnarray*}
\end{enumerate}


\end{document}
