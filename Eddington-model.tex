% !TEX root = ./notes.tex
\chapter{Eddington Standard Model}\label{s.LE-Eddington-Standard-Model}

Polytropes with index $n=3/2$ correspond to fully convective stars ($P \propto \rho^{5/3}$, the relation for an adiabat) or for white dwarfs (non-relativistic, degenerate equation of state). Another interesting case, for historical reasons, is the \emph{Eddington Standard Model}, which is a fair approximation to main-sequence stars with $M \gtrsim M_{\sun}$.  Suppose we write the equation of state as the sum of ideal gas and radiation pressure,
\begin{equation}\label{e.eos-with-rad}
 P  = \frac{\rho\kB T}{\mu \mb} + \frac{1}{3}a T^{4}.
\end{equation}
Now make the \emph{ansatz} that
\begin{equation}\label{e.beta-def}
\frac{P_{\mathrm{rad}}}{P} = \frac{aT^{4}}{3P} = 1-\beta = \mathrm{const.},
\end{equation}
that is, the radiation pressure is a fixed fraction of the total pressure everywhere.
Solving for $T$ in terms of $P$ and $\beta$,
\[ T = \left[\frac{3(1-\beta) P}{a}\right]^{1/4}, \]
and inserting this into equation~(\ref{e.eos-with-rad}) gives us a simple EOS,
\begin{equation}\label{e.beta-eos}
P = \left[\left(\frac{\kB}{\mu\mb}\right)^{4}\frac{3}{a}\right]^{1/3}\left[\frac{1-\beta}{\beta^{4}}\right]^{1/3} \rho^{4/3}.
\end{equation}
This is the equation for a polytrope of index 3.

Why is it at all reasonable to take $\beta$ as being constant? To explore this, go back to the equation for radiative diffusion
\[ F(r) = -\frac{1}{3}\frac{c}{\rho\kappa}\frac{\dif aT^{4}}{\dif r}. \]
Write the flux as $F(r) = L(r)/(4\pi r^{2})$, and since pressure decreases monotonically with radius, write
\[ 
\frac{\dif aT^{4}}{\dif r} = \frac{\dif aT^{4}}{\dif P}\frac{\dif P}{\dif r} = -\rho\frac{Gm(r)}{r^{2}}\frac{\dif aT^{4}}{\dif P}. 
\]
The equation of radiation transport then becomes
\[ L(r) = \frac{4\pi Gm(r) c}{\kappa(r)} \frac{\dif P_{\mathrm{rad}}}{\dif P}. \]
Dividing both sides by $L\cdot M/\kappa_{\mathrm{Th}}$ and rearranging terms,
\begin{equation}\label{e.Prad-P}
 \frac{\dif P_{\mathrm{rad}}}{\dif P} = \left[\frac{L\kappa_{\mathrm{Th}}}{4\pi GMc}\right]\left(\frac{\kappa(r)}{\kappa_{\mathrm{Th}}}\frac{L(r)}{L}\frac{M}{m(r)}\right).
\end{equation}
Here $L$ is the total luminosity of the star and $M$ is the total mass.  The term in $[\,]$ is a constant (the Thomson opacity $\kappa_{\mathrm{Th}}$ doesn't depend on density or temperature) and we define the \emph{Eddington luminosity} as $L_{\mathrm{Edd}}=4\pi GM c/\kappa_{\mathrm{Th}}$.  For the sun, $L_{\mathrm{Edd}} = 1.5\ee{38}\nsp\ergs\usp\second^{-1} = 3.8\ee{4}\nsp L_{\sun}$.  For the term $(\,)$ on the right-hand side, note the $L(r)/m(r)$ is basically the average energy generation rate interior to a radius $r$.  Since nuclear reactions are temperature sensitive, the heating is concentrated toward the stellar center and $L(r)/m(r)$ decreases with radius. For stars like the sun, free-free opacity is dominant, and since the free-free Rosseland opacity goes as $T^{-3.5}$, $\kappa(r)$ increases with radius.  Thus, if the energy generation rate is not too temperature dependent (the reaction $\pt + \pt \to \hydrogen[2]$ goes roughly as $T^{4.5}$ at $T=10^{7}\nsp\K$), then the term in $(\,)$  does not vary strongly with radius, and $\dif P_{\mathrm{rad}}/\dif P$ is indeed roughly constant. 

\section{Exercises}
\begin{enumerate}
\item Derive an expression for $\beta$ in terms of the mass of the star for the Eddington Standard Model.
\end{enumerate}
